% !TeX root = 00_nesting_paper.tex

\subsection{Graph Nesting}\label{sec:nestingdef}
The graph nesting operator uses a classifier to group all the vertices and edges that shall appear as a member of a cluster $C$. 

\begin{definition}[Nested Graph Classifier, $g_\kappa$]
Given a set of cluster labels $\mathcal{C}$, a \textbf{nested graph classifier} operator $g_\kappa$ maps a nested property graph $G_o$ into a nested property graph collection $\{G_C\}_{C\in\mathcal{C},G_C\neq \emptyset}$ of subgraphs of $G_o$. Such operator uses a classifier function $\kappa\colon \VS\cup \ES\to \wp(\mathcal{C})$ mapping each vertex or edge in either no graph or more than one non-empty subgraph. Each nested graph $G_C$ is a pair
$G_C=\Braket{\VS_C,\ES_C}$
where $\VS_C$ (and $\ES_C$) is the set of all the vertices $v$ (and edges $e$) in $G_o$ having $C\in \kappa(v)$ (and $C\in \kappa(e)$). Therefore, the nested graph classifier is defined as follows:
\[g_\kappa(G_o)=\Set{\Braket{\VS_C,\ES_C}|C\in\mathcal{C},\VS_C\neq\emptyset,\ES_C\neq\emptyset}\]
\end{definition}

The former definition is also going to express graph pattern evaluations, where $\kappa$ may be represented as a graph (cf. Neo4J). In order to represent the subgraphs in $g_\kappa(G_o)$ as either vertices and edges, we may use
the following \textsc{User-Defined Functions}:
\begin{definition}[User-Defined Functions]
An \textbf{object user defined function} $\mu_\Omega$ maps each subgraph $G_C\in g_\kappa(G_o)$ into a pair $\mu_\Omega(G_C)=(L,t)$, where $L\in\wp(\Sigma^*)$ is a set of labels and $t$ is a relational tuple.

An \textbf{edge user defined function} $\mu_E$ maps each subgraph $G_C\in g_\kappa(G_o)$ into a pair of identifiers $\mu_E(G_C)=(s,t)$ where $s,t\in\mathbb{N}^2$.
\end{definition}

While $\mu_\Omega$ may be used for transforming subgraphs to both vertices and edges, $\mu_E$ is only used to map subgraphs to edges. In order to complete such transformation,  we have to map each graph in $g_\kappa(G)$ into a new id $(\textbf{c},\textbf{i})\notin \VS\cup \ES$, for which an indexing function has to be defined as follows:
\[\iota_G(G_C)=(\max\Set{c|(c,i)\in V\cup E}+1,d(V_C\cup V_E))\]
where $d$ is an arbitrary bijection associating a  $n$-tuple in $\mathbb{N}^n$ into one single number $\mathbb{N}$ \cite{odifreddi1992}. The combination of all the previous functions allow the definition of the following nesting operator:


\begin{definition}[Graph Nesting]
Given a nested graph $G_{(c,i)}$ within a nested graph database $G$, an object user defined function $\mu_\Omega$, an edge user defined function $\mu_E$ and an indexing function $\iota_G$,
the graph nesting operator $\eta_{g_V,g_E,\mu_\Omega,\mu_E,\iota_G}^{\textbf{keep}}$ converts each subgraph in $G_C\in g_V(G_{(c,i)})$ (and $G_C\in g_E(G_{(c,i)})$) into a nested vertex (and nested edge) $\iota_G(G_C)$ and adds them in the resulting nested graph:
\[\begin{split}
\eta&{}_{g_V,g_E,\mu_\Omega,\mu_E,\iota_G}^{\textbf{keep}}(G_{(c,i)})=\\
&\big\langle \{v\in \nu(c,i) | V(v)=\emptyset\wedge\textbf{keep} \}\cup \iota_G(g_V(G_{(c,i)})),\\
&\;\{e\in \epsilon(c,i) | E(e)=\emptyset\wedge\textbf{keep} \}\cup \iota_G(g_E(G_{(c,i)}))\big\rangle\\
\end{split}\]
The vertices and edges in $G_{(c,i)}$ that appear neither in a nested vertex nor in a nested edge may be also returned if $\textbf{keep}$ is set to \texttt{true}. As a result, the nested graph database is updated by using the nested graph classifier and user defined functions as follows:
\[\begin{split}
\Big\langle&\VS\cup \iota_G(g_V(G_{(c,i)})),\;\ES\cup \iota_G(g_E(G_{(c,i)})),\\
& \lambda\oplus \bigoplus_{G_C\in g_E(G_{(c,i)})}\iota_G(G_C)\mapsto \mu_E(G_C),\\
& \ell\oplus\bigoplus_{G_C\in g_E(G_{(c,i)})\cup g_V(G_{(c,i)})}\iota_G(G_C)\mapsto\texttt{fst}\;\mu_\Omega(G_C),\\
& \omega\oplus\bigoplus_{G_C\in g_E(G_{(c,i)})\cup g_V(G_{(c,i)})}\iota_G(G_C)\mapsto\texttt{snd}\;\mu_\Omega(G_C),\\
& \nu\oplus \bigoplus_{G_C\in g_E(G_{(c,i)})\cup g_V(G_{(c,i)})}\iota_G(G_C)\mapsto V_C,\\
& \epsilon\oplus \bigoplus_{G_C\in g_E(G_{(c,i)})\cup g_V(G_{(c,i)})}\iota_G(G_C)\mapsto E_C\Big\rangle\\
\end{split}\]
where $f\oplus g$ defines the extension of the (finite) function $f$ with another (finite) function $g$ \cite{bergamimm17}.
\end{definition}


