% !TeX root = 00_nesting_paper-short.tex
% THIS IS AN EXAMPLE DOCUMENT FOR VLDB 2012
% based on ACM SIGPROC-SP.TEX VERSION 2.7
% Modified by  Gerald Weber <gerald@cs.auckland.ac.nz>
% Removed the requirement to include *bbl file in here. (AhmetSacan, Sep2012)
% Fixed the equation on page 3 to prevent line overflow. (AhmetSacan, Sep2012)

\documentclass[sigconf]{acmart}

%%\usepackage{lastpage}
\usepackage[utf8]{inputenc}
\usepackage{graphicx}
%\usepackage{balance}  % for  \balance command ON LAST PAGE  (only there!)

\usepackage{braket}

\newtheorem{definition}{Definition}
\newtheorem{example}{Example}
\newtheorem{axiom}{Axiom}

\usepackage{algorithm}
\usepackage[noend]{algpseudocode}
\makeatletter
\renewcommand{\ALG@beginalgorithmic}{\small}
\makeatother
\makeatletter

\usepackage{listings}
%\usepackage[dvipsnames]{xcolor}
\definecolor{eclipseBlue}{RGB}{42,0.0,255}
\definecolor{eclipseGreen}{RGB}{63,127,95}
\definecolor{eclipsePurple}{RGB}{127,0,85}
\lstset{basicstyle=\ttfamily\small,%
	%backgroundcolor=\color[rgb]{0.85,0.85,0.86},%
	frame=single,
	framerule=0pt,
	xleftmargin=\fboxsep,
	xrightmargin=\fboxsep,
	commentstyle=\color{eclipseGreen}, % style of comments
	keywordstyle=\color{eclipsePurple}, % style of keywords
	stringstyle=\color{eclipseBlue},
	breaklines=true,
	postbreak=\raisebox{0ex}[0ex][0ex]{\ensuremath{\color{red}\hookrightarrow\space}}
}
\lstdefinelanguage{sparql}{
	morecomment=[l][]{\#},
	morestring=[b][]\",
	morekeywords={BIND,URI,CONCAT,SELECT,CONSTRUCT,DESCRIBE,ASK,WHERE,FROM,NAMED,PREFIX,BASE,OPTIONAL,FILTER,GRAPH,LIMIT,OFFSET,SERVICE,UNION,EXISTS,NOT,BINDINGS,MINUS,a},
	sensitive=true
}
\lstdefinelanguage{cypher}{
	morekeywords={MATCH,RETURN,WHERE,DISTINCT,WITH,CREATE,COUNT,AS,UNION,ALL,is,null,NOT,AND,OR},
	sensitive=true,
	morecomment=[l]{//}, % l is for line comment
}
\lstdefinelanguage{AQL}{
	morekeywords={
		FOR,IN,COLLECT,INTO,RETURN
	},
	morestring=[b]",
	stringstyle=\color{eclipseGreen},
	morecomment=[l]{--}, % l is for line comment
}

\usepackage{multirow}
\usepackage{subcaption}
\usepackage{booktabs}
\usepackage{adjustbox}

\usepackage{float}
\newfloat{lstfloat}{htbp}{lop}
\floatname{lstfloat}{Listing}

\definecolor{webgreen}{rgb}{0,.5,0}
\newcommand{\mstr}[1]{\textup{\color{webgreen}``#1''}}

\newcommand{\nested}{\ensuremath{G_o}}
\newcommand{\ngraph}{{g}}
\DeclareMathOperator{\dom}{dom}
\DeclareMathOperator{\cod}{cod}
\newcommand{\ONTA}{\mstr{onta}}
\newcommand{\RELA}{\mstr{rela}}
\newcommand{\SRC}{\mstr{src}}
\newcommand{\DST}{\mstr{dst}}
\newcommand{\VS}{\mathcal{V}}
\newcommand{\ES}{\mathcal{E}}
\newcommand{\Keys}{K}
\newcommand{\Val}{V}
\newcommand{\valF}{F}
\newcommand{\nestF}{\nu}
\newcommand{\prov}{\epsilon}

\hypersetup{draft}


\begin{document}
\title{On Nesting Graphs}
\numberofauthors{4} 
\renewcommand{\auwidth}{0.23\linewidth}
\author{
\alignauthor
Giacomo Bergami\\
       \affaddr{University of Bologna}\\
       \affaddr{CSE Department}\\
       \affaddr{Bologna, Italy}\\
       \email{giacomo.bergami2@\\
       	unibo.it}
% 2nd. author
 \alignauthor
André Petermann\\
       \affaddr{University of Leipzig \& }\\
       \affaddr{ScaDS Dresden/Leipzig}\\
       \affaddr{Leipzig, Germany}\\
       \email{petermann@informatik.\\
       	uni-leipzig.de}
% 3rd. author
\alignauthor Erhard Rahm\\
\affaddr{University of Leipzig \& }\\
\affaddr{ScaDS Dresden/Leipzig}\\
\affaddr{Leipzig, Germany}\\
       \email{rahm@informatik.\\
       	uni-leipzig.de}
%\and  % use '\and' if you need 'another row' of author names
%% 4th. author
\alignauthor Danilo Montesi\\
       \affaddr{University of Bologna}\\
       \affaddr{CSE Department}\\
       \affaddr{Bologna, Italy}\\
       \email{danilo.montesi@\\
       	unibo.it}
}

\maketitle

\begin{abstract}
	[TODO]
%There are valuable analytical graph operations creating new vertices
%	and edges from existing graph elements, for example, approaches
%	to graph transformation and to graph summarization. However,
%	an operator allowing both to summarize a subgraph content into
%	one single vertex and to keep the information of the contained
%	vertices and edges is missing. This deficiency is not only due to
%	the operators’ definitions, but has to be sought on the current
%	graph data model definition not supporting nestings for overlapping
%	subgraphs. In this paper we propose a novel graph data structure
%	for nested property graphs. \textit{\color{red}Based thereon, we propose a generic
%	operation that can be specialized to graph transformation, graph
%	summarization as well as OLAP-operations such as roll-up and
%	drill-down.}
\end{abstract}

% !TeX root = 00_nesting_paper-Wall-Pedantic.tex

\section{Introduction}
Graphs allow flexible analyses of relationships among data objects. Thus, graph data management systems play an increasing role in present data analytics. Graphs have been already used as a fundamental data structure to represent data within different contexts such as corporate data \cite{success,Park2016355}, social networks \cite{xie,BrodkaK14} and linked data \cite{Vasilyeva13}.
Despite an increasing number of applications, a general operator that aggregates a single graph in a roll-up fashion is still missing. %partitions which the aforementioned vertices represent.
 The operation of adding structural aggregations to an existing graph is called \textit{graph nesting}.
A respective operator shall not only create a new graph of \textit{nested vertices} and \textit{nested edges}, each containing subgraphs of the original input graph, but also preserve the vertices and edges that are not affected by the actual operation. Further on, the operator must ensure that the nested elements can be freely unnested such that the original graph may be obtained back again. Vertices or edges of the original graph will be called \textit{members} of a nested vertex or edge, if they appear in its underlying subgraph.

In a resulting nested graph, edges connecting nested vertices express that members of the nested vertices are connected by an edge or, more general, by a path in the original graph.
In contrast to this general approach, current literature distinguishes between \textit{vertex summarization} and \textit{path summarization}. Thus, it is not possible to define a single algorithm that evaluates both kinds of patterns at the same time. Before outlining our proposed algorithmic solution, let's have a look on these existing approaches:

The \textit{vertex summarization} strategies group vertices in the manner of the relational \texttt{group by} operation and aggregate edges accordingly \cite{JunghannsPR17}. In this class of operations summarized edges can only be formed by edges that directly connect members of summarized vertices in the original graph. In other words, these approaches cannot freely nest edges, for example, it is not possible to aggregate paths. Since most of vertex summarization techniques are based on graph partitioning, they further provide no support for nested vertices and edges with overlapping members \cite{yin,Tian20085,jakawat}.
Exceptions are HEIDS \cite{ChengJQ16} and Graph Cube \cite{Zhao11}, which perform graph summarizations of one single graph over a collection of non pairwise disjoint subgraphs. However, the union of these underlying subgraphs must be equivalent to the original graph, i.e., it is not possible to take vertices and edges of the original graph over to the summarized graph or to represent outliers that belong to no group.

%To overcome to this graph operation limitation, this thesis proposes the \textbf{graph nesting} operator, thus providing a general graph summarization technique.
%A straightforward implementation  proves to be inefficient, because the visit of such graph collections of size $k$ (to be nested within the graph operand) implies to perform, in the worst case scenario, $|g|^k$ visits of the graph operand $g$: this results into an exponential algorithm, because the size of $k$ may vary, while $|g|$ is fixed. This implies that the graph must be always visited more than once, even if this may not be required. Even though this general operator proves to be inefficient in practice, it allows to detect a broader class of problems and of optimizable algorithms.
%In order to reduce the graph visiting cost from $|g|^k$ to $O(|g|)$, we could use a graph traversal approach: instead of pre-computing $k$ subgraphs of $g$ that are going to be later on used to nest $g$, we can directly perform the graph nesting while visiting the graph, thus allowing  not to perform additional costs for comparing the resulting graphs in a later step. The following example shows how such queries can be efficiently formulated and implemented.
By contrast, \textit{path summarization} techniques allow the aggregation of multiple paths among pairs of source and target vertices that share the same properties.
%At the time of the writing, such approaches can be performed only over (graph) query languages.
Currently, approaches to path summarization can only be found within graph query languages.
%The problem with both path and vertex summarizations is that no general class of either source and target vertices can be used as an outcome of a previous community detection \cite{xie,berlingerio11} or data cleaning and alignment phase \cite{ALIEH17} without rewriting the previously extracted data into an explicit query, thus requiring an additional pre-processing step and thus making such approach not as flexible as required by data integration scenarios. %%initial query each time after different vertex data is extracted, thus not allowing to use such query definition for general data integration scenarios.
%This problem is also reflected by 
These languages also support vertex summarization, but no combination of both approaches in a single step.
%This constraint thwarts the advantages of performing vertex and path summarizations concurrently:
Cypher, the query language of the productive graph database Neo4j, can perform distinct aggregations only within distinct \texttt{MATCH} clauses. SPARQL 1.1, the standard query language of the resource description framework (RDF), requires to combine vertex and path aggregation with a \texttt{UNION} operator, i.e., the same input graph must be visited twice.
%As a result, the query plan optimizers of such query languages do not allow to avoid to visit one same graph more than once whether possible.

This paper shows that such query language limitations can be reduced by using a graph nesting operator, which performs both vertex and path summarization queries concurrently with only a single visit of the input graph. %The following example provides an example of how such query can be formulated and performed. --> Old example
We study specific graph patterns and propose algorithmic optimizations and a specific physical model for efficient graph nesting. Hereby, we show that our algorithmic approaches reduce the time complexity of the visiting and nesting problem. Further on, our optimized data structure requires less indexing time than our competitors.

\begin{figure}[!t]
	\begin{minipage}[!t]{0.5\textwidth}
		\centering
		\includegraphics[width=.8\textwidth]{images/nesting/patterns/04bibliography}
		\subcaption{Input bibliographical network.}
		\label{fig:inputbibex2}
	\end{minipage}
	\medskip
	
	\begin{minipage}[!t]{0.5\textwidth}
		\centering
		\includegraphics[width=\textwidth]{images/nesting/patterns/042nested}
		\subcaption{Nested result: given two \texttt{Author}s $\color{orange}a$ and $\color{orange}a'$, there exist two  \texttt{coauthorship} edges, $\color{blue}a\to a'$ and $\color{blue}a'\to a$ if and only if they share some authored paper contained respectively in $\epsilon({\color{blue}a\to a'})$ and $\epsilon({\color{blue}a'\to a})$. Moreover, each author $\color{orange}a$ is associated to the set of his authored papers $\epsilon({\color{orange}a})$. }
		\label{fig:outputnested}
	\end{minipage}
	\caption{Nesting a bibliographic network: the provenance information is nested within the original node. }
	\label{fig:bibex2}
\end{figure}


\begin{example}
	\label{ex:nestingbib}
	Given a bibliographic network containing (at least) \textsc{Author}s and \textsc{Paper}s as vertices, and where \textsc{authorOf} edges connect each author to the papers he has authored (Figure \ref{fig:inputbibex2}), we want to ``roll up'' the network into a coauthorship network, where each \textsc{Author} is connected by a \textsc{coAuthor} edge with another  \textsc{Author}(2) with which he has published some papers (Figure \ref{fig:outputnested}). In particular, for each resulting \textsc{Author}(2) vertex, nest inside it  its papers as vertices, and nest inside each \textsc{coAuthor} edge all the papers coauthored by  the source and destination \textsc{Author}s. We also want to exclude \textsc{coAuthor} hooks over the same vertex.
	
	
\begin{figure}[!t]
	\centering
	\begin{minipage}[!t]{0.5\textwidth}
		\centering
		\includegraphics[width=.6\textwidth]{images/nesting/patterns/00_vertex_pattern.pdf}
		\subcaption{Vertex summarization pattern ($g_V$). Author is the vertex grouping reference $\gamma_V$.}
		\label{fig:vertexPat}
	\end{minipage} \begin{minipage}[!t]{0.4\textwidth}
		\centering
		\includegraphics[width=1\textwidth]{00_path_pattern.pdf}
		\subcaption{Path summarization pattern ($g_E$). Author$_{src}$ and Author$_{dst}$ are respectively edge grouping references $\gamma_E^{src}$ and $\gamma_E^{dst}$.}
		\label{fig:pathPat}
	\end{minipage}
	\caption{Vertex and Path summarization patterns for the query expressed in Example \ref{ex:nestingbib}. Vertex and edge grouping references are marked by a light blue circled node. As we can see, the vertex grouping reference depicts the same property expressed by edge grouping references.}
	\label{fig:patterns}
\end{figure}
	Figure \ref{fig:patterns} represents the desired vertex ($g_V$) and path ($g_E$) summarization patterns: the former will create a nested \textsc{Author}(2) vertex and the latter will create a \textsc{coAuthor} nested edge. Given that $g_V$ appears twice in $g_E$, we may also pre-istantiate the pattern $g_V$ by visting $g_E$ once. The two patterns have different key roles: while the vertex summarization retrieves all the papers that one author has published and nest them within one single matched author, the path summarizations return all the papers authored by two different authors and creates an edge between the two previously nested vertices. %This construction implies that a join between the two paths must be carried out. 
	
	
	This problem can be solved by visiting the graph only once; %by visiting the graph starting from the vertices:
	if the current vertex is a \textsc{Paper}, traverse backwards all the \textsc{authorOf} edges, thus reaching all of its \textsc{Author}s, that are going to be \textsc{coAuthor} for at least the current paper. Instead of associating the nesting content at the end of the graph visiting process, I can incrementally define the subgraph to be nested by using a separated nesting index: by visiting the two distinct \textsc{Author} vertices adjacent to the current \textsc{Paper}, the latter one shall  be contained in both final \textsc{Author}(2) vertices, thus allowing the definition of a  \textsc{coAuthor} edge. %%The remaining types of vertices and edges %All the other vertices and edges 
	%%may be discarded. %as a starting point for the graph visiting process
	By doing this, only the edges are visited twice, but the vertices are visited only once. Hereby, with these patterns we reach the optimal solution by visiting the graph only once.
	
	%This pattern comparison remarks that, in order to reduce the graph time visit, we must start from visiting the \texttt{Paper}, which is shared among the two distinct patterns, and then keep going with the graph visit by exploring the source and target vertices. 
\end{example}

%There might be other possible patterns that can be optimized, but we're going to focus just on vertex and path summarization patterns where edge grouping references are connected to each other at a 2 edge step distance (Section \ref{sec:THOSPA}). We're also going to show how such optimizations can be detected beforehand by looking at the pattern representation.

The fulfilment of the former scenario is achieved by this paper via the following three contributions:

\begin{itemize}
	\item We implement the aforementioned solution into a nested graph data model, where the logical model differs from the physical one.
	\item \textbf{Graph Nesting Operator} (Section \ref{sec:nestingdef}); we provide a general definition of the nesting operator which combines the path  with the vertex summarization approaches to nesting graphs.
	\item \textsc{{Two HOp Separated Patterns}} \textbf{(THoSP) algorithm for a specific graph nesting task} (Section \ref{sec:THOSPA}): we
	compare it to its implementation
	over both graph  (SPARQL, Cypher), relational (SQL) and document oriented (AQL) query languages. The results of such experiments shows that the sum of both indexing and query evaluation time of our proposed solution outperforms by at least one order of magnitude the aforementioned solutions evaluated on such databases with their respective query languages (Section \ref{sec:nestexpeval}). Consequently, our data model also proved to be curcial in providing an enhanced implementation of the specific graph nesting task.
	%\item A general strategy on how to extend the THoSP algorithm for patterns having vertex and edge grouping references is provided (Section \ref{sec:optimizableClass}).
	%%\item By extending the concept of binary predicates into edges, Edge Joins are introduced as a preliminary step towards the definition of Graph Nesting (Chapter \ref{cha:nesting}).
\end{itemize}

The source code for THoSP is provided at \texttt{\color{red}[Link removed for double-blind review]}.

\input{02_related.tex}
% !TeX root = 00_paper_entrypoint.tex

\section{Nested Graphs}\label{sec:model}
\begin{table}
\begin{adjustbox}{max width=.48\textwidth}
\begin{tabular}{c|l|c}
\toprule
\parbox[t]{2mm}{\multirow{6}{*}{\rotatebox[origin=c]{90}{$G$, NGDB}}} & $\mathcal{V},\mathcal{E}$ & vertex/edge indices in $\mathbb{N}^2$\\
                                 & $\lambda$ & edge to source-target vertices function\\
                                 & $\ell$ & vertex/edge multilabelling in $\Sigma^*$\\
                                 & $\omega$ & vertex/edge to tuple function\\
                                 & $\nu,\varepsilon$ & vertex/edge containment functions for $\mathcal{V}\cup\mathcal{E}$\\
                                 & $G_o$ & nested graph induced by vertex/edge $o\in\mathcal{V}\cup\mathcal{E}$\\
\midrule
\parbox[t]{2mm}{\multirow{10}{*}{\rotatebox[origin=c]{90}{$\eta$ operator dependencies}}} & $dt$ & index dovetailing function \\
                                 & $f\colon D\xrightarrow{f}C$ & function $f$ with domain $D$ and codomain $C$\\
                                 & $a\mapsto b$ & finite lambda function with domain $\{a\}$\\
                                 & $\oplus$ & finite function extension\\         
                                 & $\kappa$ & (graph) pattern, i.e. multilabelled graph classifier\\
                                 & $g_\kappa$ & nested graph classifier  over $\kappa$\\ 
                                 & $f_C$ & morphism from pattern to subgraph $G_C$\\
                                 & $\iota_{G_o}$ & indexing function for subgraphs $G_C\subseteq G_o$\\   
                                 & $\mu_\Omega$ & object user defined function\\  
                                 & $\mu_E$ & edge user defined function\\      
\bottomrule
\end{tabular}
\end{adjustbox}
\caption{Table Of Notations}
\vspace{-2em}
\end{table}
The term \textit{property graph}  \cite{angles12} usually refers to a directed, labelled and attributed multigraph. 
% In other words, if there is a schema, each vertex and edge will be represented by a relational tuple and without, by a document of key-value pairs. 
In a property graph a collection of \textit{labels} \cite{bergamimm17} is associated to every vertex and edge (e.g., \texttt{[}\mstr{Author}\texttt{]}) or \texttt{[}\mstr{coAuthorsip}\texttt{]}). Further on, vertices and edges may have arbitrary named attributes (\textit{properties}) in the form of key-value pairs (e.g., \texttt{\textbf{name}:Baldwin} or \texttt{\textbf{surname}:Oliver}). Property-value associations of vertices and edges can be represented by relational tuples; this is a common approach in literature used even when graphs have no fixed schema \cite{angles12}. We define the\textit{ nested (property) graph database} as the following extension of the property graph data model for nested information:

\begin{definition}[Nested Graph DataBase]
Given a set $\Sigma^*$ of strings,
	a \textbf{nested (property) graph database} $G$ is a tuple $G=\Braket{\VS, \ES, \lambda,\ell,\omega,\nestF,\prov}$, where $\VS$ and $\ES$ are disjoint sets, respectively referring to vertex and edge identifiers $o\equiv(c,i)\in\mathbb{N}^2$; $c$ is an incremental unique number associated to each graph. In particular, input data graphs have $c=0$ while  nested graphs created at nesting steps exhibit $c>0$. 
	
	A function $\lambda\colon \ES\to \VS^2$ maps each edge to its source and target vertex. Each vertex and edge is assigned to multiple possible labels through the labelling function $\ell:\VS\cup \ES\to \wp(\Sigma^*)$.  $\omega$ is a function mapping each vertex and edge into a relational tuple.
	
	In addition to the previous components defining a property graph, we also introduce functions representing \textit{vertex members} $\nestF\colon (\VS\cup \ES)\to\wp(\VS)$ and \textit{edge members} $\prov\colon(\VS\cup \ES)\to \wp(\ES)$. These functions induce the nesting by associating a set of vertices or edges to each vertex and edge. Each vertex or edge $o\in V\cup E$ induces a \textbf{nested (property) graph} as the following pair:
	\[G_o=\Braket{\nu(o),\Set{e\in\epsilon(o)|\lambda(e)\in (\cup_{n\geq 0}\;{\nu\epsilon}^{(n)}(\{o\}))^2}}\]
	where ${\nu\epsilon}$ returns the vertices contained in both vertices and edges ($\nu\epsilon(x)=\nu(x)\cup \nu(\epsilon(x))$). We denote $f(X){:=}\bigcup_{x\in X} f(x)$ when $X\subseteq \textup{dom}(f)$
\end{definition}


Since the member functions $\nu$ and $\epsilon$ induce the expansion of each single vertex or edge to a graph, we must avoid recursive nesting to support expanding operations.
% This condition is fundamental in order to define different levels as different abstractions over the data. 
Therefore, we additionally introduce the following constraints to be set at a nested property graph database level:

\begin{axiom}[Recursion Constraints]
	For each correctly nested property graph, each vertex $v\in \VS$ must not contain $v$ at any level of containment of $\nu$ and, any of its descendants $m$ must not contain $v$:
	\[\forall v\in \VS. \forall m\in \nu^+(v).\;\; m\neq v\wedge v\notin \cup_{n\geq 1}\;{\nu\epsilon}^{(n)}(m)\]
	Similarly to vertices, any edge shall not contain itself at any nesting level:
	\[\forall e\in \ES. \forall m\in \epsilon^+(e). m\neq e\wedge e\notin \cup_{n\geq 1}\;{\epsilon\nu}^{(n)}(m)\]
	where ${\epsilon\nu}$ returns the edges contained in both vertices and edges ($\epsilon\nu(x)=\epsilon(x)\cup \epsilon(\nu(x))$)
\end{axiom}

A vertex $v$ having a non-empty vertex or edge members is called \textbf{nested vertex}, while vertices with no members are simply referred to \textbf{simple vertices}. For edges, we respectively use the terms \textbf{nested edges} and \textbf{simple edges}. 

\begin{ex}[label=exImpl]
The property graph in Figure \ref{fig:inputbibex2} can be represented by the graph $G_{(0,11)}$, which is a nested vertex contained in the following nested graph database:
\[G=\Braket{\{(0,0),(0,1),\dots,(0,5),(0,11)\}, \{(0,6),\dots,(0,10)\},\lambda,\ell,\omega,\nu,\epsilon}\]

The nested vertex $(0,11)$  represents a \mstr{Bibliography} graph ($\ell(0,11)=[\mstr{Bibliography}]$), to which an empty tuple is associated ($\omega(0,11)=\{\}$). Its vertex ($\nu$) and edge ($\epsilon$) members are defined as follows:
\[\nu(0,11)=\{(0,0),\dots,(0,5)\}\quad\epsilon(0,11)=\{(0,6),\dots,(0,10)\}\]

The simple edge $6$ within the property graph in Figure \ref{fig:inputbibex2} ($\nu(0,6)=\epsilon(0,6)=\emptyset$) has now id $(0,6)$; it has one label, $\ell(0,6)=[\mstr{AuthorOf}]$, and it is associated to an empty tuple ($\omega(0,6)=\{\}$).
The source and target vertices are 
$\lambda(0,6)=\Braket{(0,0),\;(0,3)}$. Similar considerations can be carried out for each  remaining edge.

The simple vertex $0$ in the same Figure has id $(0,0)$ in the present example; such vertex refers to the \mstr{Author} \texttt{Abigail Conner}. This information is represented as follows:
\[\ell(0,0)=[\mstr{Author}]\quad\nu(0,0)=\epsilon(0,0)=\emptyset\] \[\omega(0,0)=\{\texttt{\textbf{name}}\colon\texttt{Abigail},\texttt{\textbf{surname}}\colon\texttt{Conner}\}\]
Similar considerations can be carried out for each remaining vertex.

\end{ex}

% !TeX root = 00_nesting_paper.tex

\subsection{Graph Nesting}\label{sec:nestingdef}
The graph nesting operator uses a classifier to group all the vertices and edges that shall appear as a member of a cluster $C$. 

\begin{definition}[Nested Graph Classifier, $g_\kappa$]
Given a set of cluster labels $\mathcal{C}$, a \textbf{nested graph classifier} operator $g_\kappa$ maps a nested property graph $G_o$ into a nested property graph collection $\{G_C\}_{C\in\mathcal{C},G_C\neq \emptyset}$ of subgraphs of $G_o$. Such operator uses a classifier function $\kappa\colon \VS\cup \ES\to \wp(\mathcal{C})$ mapping each vertex or edge in either no graph or more than one non-empty subgraph. Each nested graph $G_C$ is a pair
$G_C=\Braket{\VS_C,\ES_C}$
where $\VS_C$ (and $\ES_C$) is the set of all the vertices $v$ (and edges $e$) in $G_o$ having $C\in \kappa(v)$ (and $C\in \kappa(e)$). Therefore, the nested graph classifier is defined as follows:
\[g_\kappa(G_o)=\Set{\Braket{\VS_C,\ES_C}|C\in\mathcal{C},\VS_C\neq\emptyset,\ES_C\neq\emptyset}\]
\end{definition}

The former definition is also going to express graph pattern evaluations, where $\kappa$ may be represented as a graph (cf. Neo4J). In order to represent the subgraphs in $g_\kappa(G_o)$ as either vertices and edges, we may use
the following \textsc{User-Defined Functions}:
\begin{definition}[User-Defined Functions]
An \textbf{object user defined function} $\mu_\Omega$ maps each subgraph $G_C\in g_\kappa(G_o)$ into a pair $\mu_\Omega(G_C)=(L,t)$, where $L\in\wp(\Sigma^*)$ is a set of labels and $t$ is a relational tuple.

An \textbf{edge user defined function} $\mu_E$ maps each subgraph $G_C\in g_\kappa(G_o)$ into a pair of identifiers $\mu_E(G_C)=(s,t)$ where $s,t\in\mathbb{N}^2$.
\end{definition}

While $\mu_\Omega$ may be used for transforming subgraphs to both vertices and edges, $\mu_E$ is only used to map subgraphs to edges. In order to complete such transformation,  we have to map each graph in $g_\kappa(G)$ into a new id $(\textbf{c},\textbf{i})\notin \VS\cup \ES$, for which an indexing function has to be defined as follows:
\[\iota_G(G_C)=(\max\Set{c|(c,i)\in V\cup E}+1,d(V_C\cup V_E))\]
where $d$ is an arbitrary bijection associating a  $n$-tuple in $\mathbb{N}^n$ into one single number $\mathbb{N}$ \cite{odifreddi1992}. The combination of all the previous functions allow the definition of the following nesting operator:


\begin{definition}[Graph Nesting]
Given a nested graph $G_{(c,i)}$ within a nested graph database $G$, an object user defined function $\mu_\Omega$, an edge user defined function $\mu_E$ and an indexing function $\iota_G$,
the graph nesting operator $\eta_{g_V,g_E,\mu_\Omega,\mu_E,\iota_G}^{\textbf{keep}}$ converts each subgraph in $G_C\in g_V(G_{(c,i)})$ (and $G_C\in g_E(G_{(c,i)})$) into a nested vertex (and nested edge) $\iota_G(G_C)$ and adds them in the resulting nested graph:
\[\begin{split}
\eta&{}_{g_V,g_E,\mu_\Omega,\mu_E,\iota_G}^{\textbf{keep}}(G_{(c,i)})=\\
&\big\langle \{v\in \nu(c,i) | V(v)=\emptyset\wedge\textbf{keep} \}\cup \iota_G(g_V(G_{(c,i)})),\\
&\;\{e\in \epsilon(c,i) | E(e)=\emptyset\wedge\textbf{keep} \}\cup \iota_G(g_E(G_{(c,i)}))\big\rangle\\
\end{split}\]
The vertices and edges in $G_{(c,i)}$ that appear neither in a nested vertex nor in a nested edge may be also returned if $\textbf{keep}$ is set to \texttt{true}. As a result, the nested graph database is updated by using the nested graph classifier and user defined functions as follows:
\[\begin{split}
\Big\langle&\VS\cup \iota_G(g_V(G_{(c,i)})),\;\ES\cup \iota_G(g_E(G_{(c,i)})),\\
& \lambda\oplus \bigoplus_{G_C\in g_E(G_{(c,i)})}\iota_G(G_C)\mapsto \mu_E(G_C),\\
& \ell\oplus\bigoplus_{G_C\in g_E(G_{(c,i)})\cup g_V(G_{(c,i)})}\iota_G(G_C)\mapsto\texttt{fst}\;\mu_\Omega(G_C),\\
& \omega\oplus\bigoplus_{G_C\in g_E(G_{(c,i)})\cup g_V(G_{(c,i)})}\iota_G(G_C)\mapsto\texttt{snd}\;\mu_\Omega(G_C),\\
& \nu\oplus \bigoplus_{G_C\in g_E(G_{(c,i)})\cup g_V(G_{(c,i)})}\iota_G(G_C)\mapsto V_C,\\
& \epsilon\oplus \bigoplus_{G_C\in g_E(G_{(c,i)})\cup g_V(G_{(c,i)})}\iota_G(G_C)\mapsto E_C\Big\rangle\\
\end{split}\]
where $f\oplus g$ defines the extension of the (finite) function $f$ with another (finite) function $g$ \cite{bergamimm17}.
\end{definition}



% !TeX root = 00_nesting_paper.tex
\begin{figure}[!b]
	\centering
	\begin{minipage}[!t]{0.45\textwidth}
		\centering
		\includegraphics[width=\textwidth]{images/nesting/patterns/01_preliminar_question.pdf}
		\subcaption{Comparing the vertex summarization pattern and the path summarization patterns. We suppose that edge grouping references ($\gamma_E^{src}$ and $\gamma_E^{dst}$) correspond to the same vertex appearing as a vertex grouping reference ($\gamma_V$) Such correspondence is directly marked my the user itself providing the query by drawing morphisms (correspondences) between the vertex and the path summarization patterns (red edges). The intersections between the two patterns may be directly outlined by the user itself that provides the query (blue edges, representing other morphisms).}
		\label{fig:patcomparison}
	\end{minipage} \begin{minipage}[!t]{0.45\textwidth}
		\centering
		\includegraphics[width=.8\textwidth]{images/nesting/patterns/02_edge_analysis.pdf}
		\subcaption{Path summarization pattern sharing an $\alpha$ area of common patterns shared between the patterns, which are necessairly not the edge grouping references by definition.}
		\label{fig:edgewithIntersectionNonSRC}
	\end{minipage}%\quad \begin{minipage}[!t]{0.45\textwidth}
%		\centering
%		\includegraphics[width=.8\textwidth]{images/nesting/patterns/03_edge_analysis.pdf}
%		\subcaption{Path summarization pattern not sharing a common $\alpha$ area, although $\gamma_E^{src}$ and $\gamma_E^{dst}$ must always be present in $g_E$ by hypothesis. This constraint guarantees that the newly created edge will be associated to a nested vertex originating from the vertex summarization pattern.}
%		\label{fig:edgeWithNoIntersection}
%	\end{minipage}
	\caption{Vertex  and Path  summarization patterns.}
	\label{fig:patternAnalysis}
\end{figure}
\vfill\eject

\section{Class of optimizable graph nesting queries}\label{sec:optimizableClass}
Prior to the analysis of the THoSP algorithm that is going to be provided later on in the next subsection, we want to discuss which is the class of vertex and path summarization patterns optimizable as discussed in the chapter's introduction. As we already observed, %, the comparison of several graph collections may lead to an intractable solution which is exponential in time over the graph dimension, where the exponent is the arbitrary size of the collection generated from the original graph operand. 
the exhaustive search of graph patterns in the most general scenario must be done at any rate, because the vertex and  edge subgraphs may be extracted by an external tool of which we totally ignore its behaviour. As we also discussed in the introduction, the generation of the collections is only relevant with respect to actual data that is going to be nested and, in our case, we can only nest a subgraph of the graph resulting from the graph visiting process: consequently, within each pattern we must remark which elements are going to be nested at the final result.

First, we must provide a formal characterization of the grouping reference: we want to elect a subgraph $\gamma_p$ for each graph pattern $p$ such that each graph generated by  ${g_p}(\nested)$ expose unique elements referring to $\gamma_p$. %each for each nested graph $\nested$, a \textit{grouping reference} $\gamma_p$ of a graph pattern $g_p$ elects which elements of the subgraph $g\subseteq \nested$ returned by the graphs extraction query must appear only once per returned subgraph by $m_{g_p}(\nested)$. 
These grouping references allow to elect the subcomponents that identify an entity over which the aggregation will be performed during the graph matching process. Hereby, we can provide the following formal definition for grouping references:
\begin{definition}[Grouping reference]
	Given a graph pattern $p$ generating subgraphs ${g_p}(\nested)$ over a nested graph $\nested$, a \textbf{grouping reference} $\gamma_p$ is a subpattern $\gamma_p\subseteq g_p$ restricting the possible  generated by ${g_p}(\nested)$ such that to each vertex (or edge) in $\gamma_p$ corresponds one single vertex (or edge) in $\nested$ and that $g_p(\nested)$ contains distinct graphs. Such correspondence may be denoted using a function $f_C$ for each $G_C\in g_p(\nested)$ denoted as follows:
	\[m_{g_p}(\nested)=\Set{f_C|G:C\in g_p(\nested), p \overset{f_C}{\to} G_C}\]
\end{definition}

In particular, the elements appearing in SQL's \texttt{Group By}, AQL's \texttt{COLLECT} and Cypher's \texttt{WITH} (except from the parts where the aggregation is performed) are all grouping references of our matched graphs.

In order to reduce the computational complexity of aggregating the grouping reference, we can reduce the grouping references to one single vertex for vertex summarization patterns, and to two (distinct) vertices for path summarization patterns: in the first scenario, such vertex will identify the entity over which we can perform the nesting of all the other matched contents, while for the path summarization pattern the grouping references will identify the source ($\gamma_E^{src}$) and target ($\gamma_E^{dst}$) vertices corresponding to the vertex summarization patterns' grouping references, and hereby corresponding to the final vertices that are going to be nested in the final result. Hereby, the class of our algorithms create new nested edges only over vertices that have been previously matched as grouping references and then nested. Moreover, we can choose to mark with a specific $\ell$ label (e.g. \mstr{toNest}) each vertex and edge within each pattern in order to remark which matched vertices and edges are going to be represented in the final nesting result; this implies that UDF functions are not required by such class of problems because they can be directly represented within the graph patterns.


%\hl{Research purposes, currently investigated with the research group at Leipzig University}
%\textbf{Generalization of pattern matching approaches (algorithmical enhancement)}. Current graph query languages perform matches of different $G_H$ separately, while it can be beneficial to visit all the common subpatterns only once, and not $k$ times, thus reducing the visiting complexity from $|g|^k$ to $|g|$, where $|g|$ is the common subgraph. The application of chained generalized graph grammar rules must be done efficiently without repeating some visiting tasks more than once. The target is to find the subgraphs commonly shared by several matching strategies, whenever is possible (e.g. graph nesting as introduced in the previous chapter.)
%\textbf{Overcoming to Cypher limitations}. This algorithmical assumption implies that we do not need to extend Cypher to support nestings for some simple graph pattern matching tasks. 

Figure \ref{fig:patternAnalysis} provides an example on how  graph nesting queries based on grouping references can be optimized for both vertex ($g_V$) and path ($g_E$) summarization queries: given that the users are going to provide both the vertex and the path summarization queries, such users must directly draw the correspondences between vertex and edge pattern queries, so that the correspondences can be promptly  identified by the query plan which can better optimize the whole query execution (Figure \ref{fig:patcomparison}). After doing so, we can start to perform the general graph visiting algorithm for graph nesting (Algorithm \ref{alg:general}) by detecting which regions of both patterns are shared together in $\alpha=g_V\cap g_E$ (Figure \ref{fig:edgewithIntersectionNonSRC}); given that source and destination vertices in path summarization patterns' grouping references are distinct by definition, source and destination vertices may not be represented in $\alpha$ (Algorithm \ref{alg:general}, line \ref{obtainAlpha}). Consequently, in order to reduce the graph visiting process, we can first perform pattern matching over the input graph over $\alpha$, thus allowing a partial instantiation of the $g_V$ and $g_E$ patterns, and then iteratively extend the nesting information after each visit of $\alpha$ and  its own refinements. In particular, we can perform the algorithm  as follows:
\begin{algorithm}[!t]
	\caption{Grouping Reference Optimizable Queries (GROQ)}\label{alg:general}
	\algrenewcommand\algorithmicindent{0.5em}
	\begin{adjustbox}{max width=\textwidth}
		\begin{minipage}{1.2\linewidth}
			\begin{algorithmic}[1]
				\Procedure{GROQ}{$(g_V,\gamma_V),({g_E},\gamma_E^{src},\gamma_E^{dst}),m;\;\nested$}:
				\State {$\alpha:=g_V\cap {g_E}\backslash(\gamma_V\cup\gamma_E)$;}\label{obtainAlpha}
				\State {lV $:=$ [];}
				\If{$\alpha\neq\emptyset$}
				\For {\textbf{each graph} $g^i$ generated from $m_\alpha(\nested)$ }\label{eachAlpha}
				\State {lV $:=\{f_i\in m_{g_V;\gamma_V}(\nested)|f_i(\alpha)=g^i\}$}
				\State {GROQ$\alpha$}{($(g_V,\gamma_V),({g_E},\gamma_E^{src},\gamma_E^{dst}),m;\;\textup{lV},\nested$)}
				\EndFor
				\Else
				\State {lV $:=m_{g_V;\gamma_V}(\nested)$}\label{complete}
				\State {GROQ$\alpha$}{($(g_V,\gamma_V),({g_E},\gamma_E^{src},\gamma_E^{dst}),m;\;\textup{lV},\nested$)}
				\EndIf
				
				\EndProcedure
				\State
				\Procedure{GROQ$\alpha$}{$(g_V,\gamma_V),({g_E},\gamma_E^{src},\gamma_E^{dst}),m;\;\textup{lV},\nested$}
				\For {\textbf{each morphism} $f_i\in\; $lV}
				\State {$\{(c,i)\}:=f_i(\gamma_V)$}\label{vertexReferencePatt}
				\State {$ \omega(c+1,i)=\omega(c,i);\quad \ell(c+1,i)=\ell(c,i)$}\label{vertexLVPreserve}
				\State {$\nu(c+1,\ngraph):=\nu(c+1,\ngraph)\cup \{(c+1,i)\}$}\label{vertexGeneration}
				\State {$\nu(c+1,i):=\nu(c+1,i)\cup\Set{f_i(o)|o\in V\cap\mathcal{V},\mstr{toNest}\in\ell(o)}$}\label{vertexContent1}
				\State {$\epsilon(c+1,i):=\epsilon(c+1,i)\cup\Set{f_i(o)|o\in V\cap\mathcal{E},\mstr{toNest}\in\ell(o)}$}\label{vertexContent2}
				\EndFor
				\For {\textbf{each morphism} $f_i,f_j\in\; $lV}
				\State{lE $:=\{f_k\in m_{{g_E};\gamma_E^{src},\gamma_E^{dst}}(\nested)|f_i(\gamma_V)=f_k(\gamma_E^{src}),f_j(\gamma_V)=f_k(\gamma_E^{dst})\}$}\label{fulTraverse}
				\For {\textbf{each morphism} $f_k\in\; $lE}
				\State {$\{(c,s)\}:=f_i(\gamma_E^{src})$;\qquad $\{(c,d)\}:=f_i(\gamma_E^{dst})$}\label{pathReferencePatt1}
				\State {$j:=dt(s,d)$}\label{pathReferencePatt2}
				\State {$\omega(c+1,j):=\omega(c,s)\cup\omega(c,d);\; \xi(i_{c+1}):=\xi(i_c)\cup\ell(d_c)$}\label{fromNewElements}
				\State {$\epsilon(c+1,j):=\epsilon(c+1,j)\cup f_k(\gamma_V)$}\label{edgeGeneration}
				\State {$\lambda(c+1,j):=(({c+1},s),({c+1},d))$}
				\State {$\nu(c+1,j):=\nu(c+1,j)\cup\Set{f_k(o)|o\in E\cap \mathcal{V},\mstr{toNest}\in\ell(o))}$}
				\State {$\epsilon(c+1,j):=\epsilon(c+1,j)\cup\Set{f_k(o)|o\in E\cap\mathcal{E},\mstr{toNest}\in\ell(o)}$}
				\EndFor
				\EndFor
				\EndProcedure
				
			\end{algorithmic}
		\end{minipage}
	\end{adjustbox}
\end{algorithm}

\begin{itemize}
	\item Given a nested graph classifier $g$ for graph pattern languages, we extract all the subgraphs $g^i$ of $\nested$ generated by  $g_\alpha(\nested)$, when $\alpha$ is not empty (line \ref{eachAlpha}). If $\alpha$ is otherwise an empty pattern, we must necessarily perform a complete visit of the vertex patterns $g_V$, and perform complete instantiations of such patterns (line \ref{complete}). 
	\item We can iteratively construct the nested graph without knowing the complete information by relying on the ids of the expected elements, and we can provide the greatest subgraph of $g$ matching $\alpha$  after visiting  each possible $\alpha$ matching result, represented as a correspondence $f_i$. For this reason, the GROQ$\alpha$ subroutine may be called in both cases.
	
	\item After providing a partial instantiation of the vertex summarization patterns via $\alpha$, we find a vertex $(c,i)$ matching the grouping reference $\gamma_V$ to which we are going to nest the remaining objects: from $(c,i)$ we generate a newly derived vertex $({c+1},i)$ %start to generate a new nested vertex: this new vertex $i_{c+1}$ descends from the object $(c,i)$ matched by the grouping reference $\gamma_V$ 
	(line \ref{vertexReferencePatt}) preserving all the labels and tuples (line \ref{vertexLVPreserve}). The nesting content of $({c+1},i)$ derives from the partial instantiation of the correspondence $f_i$, by choosing the vertices and edges in $\nested$ which corresponds to vertex summarization objects marked with \mstr{toNest} (lines \ref{vertexContent1} and \ref{vertexContent2}).
	
	\item At this point we can use the same semi-instantiated correspondence in \texttt{lV} from $\alpha$ to partially instantiate the path summarization pattern, that  is now going to be fully traversed (line \ref{fulTraverse}). For each of these $f_i$ instantiations, new edges are going to be generated, inheriting the labels and tuples from the matched edges grouping references, $(c,s)$ and $(c,d)$. In particular, we can directly create associate to such edge the soruces and the targets represented by nested vertices, which will respectively be $(c+1,s)$ and $({c+1},d)$.
	
	\item The procedure is iterated until the whole graph is not visited via subsequent correspondences, and hence all the matched elements are associated from the objects $({c+1},i)$ (either vertices or edges) generated from the ones matched by the grouping reference $(c,i)$.
\end{itemize}

As we can see from the algorithm, the advantage of this approach is that the graph $g^i$ and the instantiated correspondences (as a consequence of the graph matching phase) are promptly used to define the nested information (e.g., lines \ref{vertexGeneration}-\ref{vertexContent2}). It is evident that the aforementioned algorithm provides the best performances when $\gamma_E^{src}$ and $\gamma_E^{dst}$ are separated by  one edge distance in $\alpha$   and both $g_V$ and $g_E$ create graph collections that are partitions of $\nested$. On the other hand, this class of algorithms was already discussed in literature and, consequently, an approach describing how to optimize such  scenarios can be already found in literature \cite{JunghannsPR17}. Nevertheless, this chapter focuses on another types of algorithms, which are the ones where $\alpha$ contains two edges and one vertex; this class of problems, to the best of our knowledge, has not been discussed yet in current literature with respect to their optimizations. %Therefore, the next paragraph is going to introduce one specific algorithm for $\alpha\neq\emptyset$, where in particular $\alpha$ is formed by one single vertex.
%Please note that, when $\alpha=\emptyset$, the computational complexity of the algorithm may easily become quadratic ($|\nu(\ngraph)|+|\nu(\ngraph)|^2$).


% !TeX root = 00_nesting_paper.tex
\subsection{Two HOp Separated Patterns Algorithm}\label{sec:THOSPA}
We now want to focus on a specific instance of the problem stated in Algorithm \ref{alg:general}: suppose to store a graph using adjacency lists \texttt{[similarly to the one proposed in the Graph Join algorithm]}; in particular, the previous data structure is now extended with both vertex and edge containment, plus with both ingoing and outgoing edges for each single graph vertex. The latter requirement is added in order to satisfy the possibility to visit the edges backwards, thus allowing to navigate the graph in each possible direction. The main data structure over which this algorithm relies  is presented in Figure \ref{nestedGraphVertex}: it shows that minor changes have been applied to the original data structure that was used to serialize graph within the graph join scenario. %Moreover, in this case we mark with different hash values the vertices within the data structure satisfying different predicates within the predicates. 
Given that the data structure requires a simple linear visit of the graph, no additional primary and secondary data structures are required. Nevertheless, during our serialization phase we provide both a primary index for accessing external informations (\textit{VertexIndex}) and the serialization of all the vertices' adjacency lists, which is going to be used for traversing the graph (\textit{VertexVals}). In our straightforward implementation, hash values are here used only as placeholders for the nodes' labels used within the patterns but, given that vertices are not sorted by hash value as for graph joins, we keep the hash fields for both backward compatibility and in order to make graph joins possible for nested graphs, too.


Let us now restrict $\alpha$ to one single vertex and two edges: for each vertex $v$ matched by $\alpha$ ($\alpha\vDash v$) we know that we must (possibly) visit all the edges going from $v$ towards the vertices $\gamma_E^{src}$ and $\gamma_E^{dst}$, that substantially are $\gamma_V$. Please note that if in $g_E$ there is no path connecting $\alpha$ to $\gamma_E^{src}$ or $\gamma_E^{dst}$, the problem may quickly become cubic with respect to the size of the vertices, because we must create all the possible permutations where $v$ is present alongside another element matching $\gamma_E^{src}$ or $\gamma_E^{dst}$. Therefore, having an edge as a constraint in $\alpha$ linking $v$ towards $\gamma_E^{src}$ or $\gamma_E^{dst}$ both in $g_E$ and $g_V$ can reduce all the possible computations to the actual edges traversed from $v$ meeting the grouping references. Therefore, we know whether we finished visiting our patterns after exhaustively matching all the elements within the pattern. We can now reduce the cost to check when we finished traversing all the elements reaching  $\gamma_E^{src}$ and $\gamma_E^{dst}$ from $v$  after a linear scan of all the ingoing or outgoing nodes. Hereby, the most simple graph nesting example is where $v$ is the middle node between a path between $\gamma_E^{src}$ and $\gamma_E^{dst}$ vertices. 

\begin{figure*}
	\centering
	\includegraphics[height=.7\textheight,angle =90]{/mnt/DEC4763AC47614CD/thesis/imgs/10nesting/test}
	\caption{Extending the serialized graph data structure presented for graph join for the nesting operation. In particular, the present data structure extends each vertex representation in $VertexVals$ (Figure \ref{fig:graphstructure}) in order to fully supports the nested graph data model:
		entities and relationships may now be contained into another data node (either a vertex or an edge). The first block of the serialized data structure contains the pointers towards the memory regions containing data which may vary in size. The fuchsia nodes remark the memory spaces where such data containments may be stored. Moreover, ingoing edges are stored as well as outgoing edges. }\label{nestedGraphVertex}
\end{figure*}
\begin{algorithm}[!t]
	\caption{Two HOp Separated Patterns Algorithm (THoSP)}\label{alg:THoSPAlgorithm}
	\begin{adjustbox}{max width=\textwidth}
		\begin{minipage}{1.2\linewidth}
			\algrenewcommand\algorithmicindent{1em}
			\begin{algorithmic}[1]
				\Procedure{PartitionHashJoin}{$(g_V,\gamma_V),({g_E},\gamma_E^{src},\gamma_E^{dst});\; \nested$}
				\State \textsc{File} $AdjFile$ = \textsc{Open}($\nested$);
				\State \textsc{File} $Nesting$ = \textsc{Open}(\textbf{new});
				\State \textsc{Adjacency} $toSerialize$ = \par \textbf{new} \textsc{Map<Vertex,<Edge,Vertex>>}();
				\State {$\alpha:=g_V\cap {g_E}\backslash(\gamma_V\cup\gamma_E)$;}
				\For{\textbf{each vertex} $v$ in $AdjFile$}
				\If{$\alpha\vDash v$} 
				\For {\textbf{each} $\gamma_V\vDash (u,e,v)$} 
				\State{$u':=dt(1,dt(0,u))$} 
				\State{$NestingIndex$.write($\Braket{u',u}$)} 
				\State{$NestingIndex$.write($\Braket{u',e}$)}
				\State{$NestingIndex$.write($\Braket{u',v}$)}
				\If{$\gamma_E\vDash (v,e,u,e',w)$}
				\State{$w':=dt(1,dt(0,w))$} 
				\State{$\varepsilon:=dt(1,dt(u,w))$}
				\State{$NestingIndex$.write($\Braket{\varepsilon,u}$)}
				\State{$NestingIndex$.write($\Braket{\varepsilon,e}$)}
				\State{$NestingIndex$.write($\Braket{\varepsilon,w}$)}
				\State{$NestingIndex$.write($\Braket{\varepsilon,e'}$)}
				\State{$NestingIndex$.write($\Braket{\varepsilon,v}$)} 
				\State{$toSerialize$.put($u'$,$\Braket{\varepsilon,w'}$)}
				\EndIf
				\EndFor
				\EndIf
				\EndFor
				$AdjFile$.serialize($toSerialize$);
				
				\EndProcedure
			\end{algorithmic}
		\end{minipage}
	\end{adjustbox}
\end{algorithm}
\begin{table*}[!t]
	\centering
	\begin{tabular}{@{}cr|rr@{}}
		\toprule
		{\textbf{Operands' Vertices}} & Matched Graphs  & {\textbf{General Nesting} (ms)} & {\textbf{THoSP} (ms)}  \\	
		\midrule
		$10$ & $3$ &  0.57       & 0.11\\
		$10^2$ & $58$  & 0.73        & 0.14\\
		$10^3$  & $968$  & 2.78   & 0.46\\
		$10^4$ & $8,683$   & 152.11   & 4.07\\
		$10^5$ & $88,885$   & 14,015.00 & 43.81 \\
		$10^6$  & $902,020$  &  1,579,190.00      & 563.02\\
		$10^7$ & $8,991,417$   &  $>$1H      & 8,202.93\\
		$10^8$ & $89,146,891$   &  $>$1H      & 91,834.20\\
		\bottomrule
	\end{tabular}
	%\end{minipage}
	\caption{Comparing the performances of the THoSP algorithm with the naive General Nesting algorithm. This comparison shows that the previously defined algorithm has a worse performance than the THoSP one. }
	\label{tab:comparisonTwo}
\end{table*}


%The main data structure over which this algorithm relies  is presented in Figure \ref{nestedGraphVertex}: it shows that minor changes have been applied to the original data structure that was used to serialize graph within the graph join scenario. %%Moreover, in this case we mark with different hash values the vertices within the data structure satisfying different predicates within the predicates. 
%Given that the data structure requires a simple linear visit of the graph, no additional primary and secondary data structures are required. Nevertheless, during our serialization phase we provide both a primary index for accessing external informations (\textit{VertexIndex}) and the serialization of all the vertices' adjacency lists, which is going to be used for traversing the graph (\textit{VertexVals}). In our straightforward implementation, hash values are here used only as placeholders for the nodes' labels used within the patterns but, given that vertices are not sorted by hash value as for graph joins, we keep the hash fields for both backward compatibility and in order to make graph joins possible for nested graphs, too.

Finally, Algorithm \ref{alg:THoSPAlgorithm} provides the desired implementation of the THoSP algorithm: we can observe that THoSP does not include the data serialization preprocessing step because the data indexing provided at that step is not required by the present algorithm. The main memory is used to create the graph (represented as an adjacency list) that is going to be later on serialized using the same data structure used for providing the result for graph joins, that is an adjacency list where only the vertices' and edges' id appear. This choice is also done both for backward compatibility with respect to the graph join data structure \cite{bergamimm17} and for representing the nesting containment as a separate data structure. We can easily observe that this approach may slow down the whole algorithm, that can be quickened by directly storing the graph representation in secondary memory by using linear hashing. The nesting data structure is stored in a $NestingIndex$ file as a set of pairs $\Braket{u,v}$, where $u$ represents the containing object and $v$ represents the content. By doing so, we omit the \texttt{Group By} cost which affects the previously seen query languages, thus allowing to an overall better performance. Even in this case, a join is performed between the two nested patterns: this is evident from the two nested for loops appearing in the algorithm. 

Table \ref{tab:comparisonTwo} provides a comparison between the general Nesting Algorithm \texttt{[Should we put the previous CIKM version on arXive, so that here we just focus on the main\\ algorithm?]} and over the THoSP implementation of the query provided in our running example, under the assumptions that are going to be soon introduced in the next section. In particular, while THoSP increases linearly alongside the data size, the general nesting algorithm grows quadratically, thus quickly leading to a intractable time evaluation for big data scenarios. Hereby, the THoSP algorithm is going to be used in comparisons with other problem-specific queries on different query languages and data structures.
\input{07_evaluation.tex}
% !TeX root = 00_nesting_paper.tex
\section{Conclusions}
This paper introduces an algorithm for graph nesting jointly with graph traversal queries. Moreover, a class of graph traversal queries that can be optimized is identified, among which a graph algorithm (THoSP) is proposed. THoSP generates nesting vertices of two graph vertices appearing at most at two edge distance within the input graph. By running our algorithm over our graph data structures, we also know that our choice of providing the containment as an external index provides better performances, thus avoiding an additional data grouping phase. Moreover, this algorithm shows that the definition of a separate index for providing the algorithmic result of graph nesting is beneficial, because it allows to return the nested graph as a simple graph having additional nesting informations. This solution was possible due to the assumptions of our proposed graph data model, where  it is possible to refer any time to the elements that are going to be created later on within the computation, by  deterministically knowing  the id belonging to the to-be-created nested vertex or edge.  Therefore, this paper shows that the representation of nested graph may lead to the solution of current graph querying problems in a tractable way. Nevertheless, we believe that further studies will have to be done on the class of  graph nesting problems, thus extending our work on THoSP.
%\appendix
%\input{09_languages.tex}

\bibliographystyle{abbrv}
\bibliography{mbib} 
\balance

\end{document}
