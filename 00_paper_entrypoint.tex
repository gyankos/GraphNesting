% !TeX root = 00_nesting_paper-short.tex
% THIS IS AN EXAMPLE DOCUMENT FOR VLDB 2012
% based on ACM SIGPROC-SP.TEX VERSION 2.7
% Modified by  Gerald Weber <gerald@cs.auckland.ac.nz>
% Removed the requirement to include *bbl file in here. (AhmetSacan, Sep2012)
% Fixed the equation on page 3 to prevent line overflow. (AhmetSacan, Sep2012)

\documentclass[sigconf]{acmart}

%%\usepackage{lastpage}
\usepackage[utf8]{inputenc}
\usepackage{graphicx}
%\usepackage{balance}  % for  \balance command ON LAST PAGE  (only there!)

\usepackage{braket}

\newtheorem{definition}{Definition}
\newtheorem{example}{Example}
\newtheorem{axiom}{Axiom}

\usepackage{algorithm}
\usepackage[noend]{algpseudocode}
\makeatletter
\renewcommand{\ALG@beginalgorithmic}{\small}
\makeatother
\makeatletter

\usepackage{listings}
%\usepackage[dvipsnames]{xcolor}
\definecolor{eclipseBlue}{RGB}{42,0.0,255}
\definecolor{eclipseGreen}{RGB}{63,127,95}
\definecolor{eclipsePurple}{RGB}{127,0,85}
\lstset{basicstyle=\ttfamily\small,%
	%backgroundcolor=\color[rgb]{0.85,0.85,0.86},%
	frame=single,
	framerule=0pt,
	xleftmargin=\fboxsep,
	xrightmargin=\fboxsep,
	commentstyle=\color{eclipseGreen}, % style of comments
	keywordstyle=\color{eclipsePurple}, % style of keywords
	stringstyle=\color{eclipseBlue},
	breaklines=true,
	postbreak=\raisebox{0ex}[0ex][0ex]{\ensuremath{\color{red}\hookrightarrow\space}}
}
\lstdefinelanguage{sparql}{
	morecomment=[l][]{\#},
	morestring=[b][]\",
	morekeywords={BIND,URI,CONCAT,SELECT,CONSTRUCT,DESCRIBE,ASK,WHERE,FROM,NAMED,PREFIX,BASE,OPTIONAL,FILTER,GRAPH,LIMIT,OFFSET,SERVICE,UNION,EXISTS,NOT,BINDINGS,MINUS,a},
	sensitive=true
}
\lstdefinelanguage{cypher}{
	morekeywords={MATCH,RETURN,WHERE,DISTINCT,WITH,CREATE,COUNT,AS,UNION,ALL,is,null,NOT,AND,OR},
	sensitive=true,
	morecomment=[l]{//}, % l is for line comment
}
\lstdefinelanguage{AQL}{
	morekeywords={
		FOR,IN,COLLECT,INTO,RETURN
	},
	morestring=[b]",
	stringstyle=\color{eclipseGreen},
	morecomment=[l]{--}, % l is for line comment
}

\usepackage{multirow}
\usepackage{subcaption}
\usepackage{booktabs}
\usepackage{adjustbox}

\usepackage{float}
\newfloat{lstfloat}{htbp}{lop}
\floatname{lstfloat}{Listing}

\definecolor{webgreen}{rgb}{0,.5,0}
\newcommand{\mstr}[1]{\textup{\color{webgreen}``#1''}}

\newcommand{\nested}{\ensuremath{G_o}}
\newcommand{\ngraph}{{g}}
\DeclareMathOperator{\dom}{dom}
\DeclareMathOperator{\cod}{cod}
\newcommand{\ONTA}{\mstr{onta}}
\newcommand{\RELA}{\mstr{rela}}
\newcommand{\SRC}{\mstr{src}}
\newcommand{\DST}{\mstr{dst}}
\newcommand{\VS}{\mathcal{V}}
\newcommand{\ES}{\mathcal{E}}
\newcommand{\Keys}{K}
\newcommand{\Val}{V}
\newcommand{\valF}{F}
\newcommand{\nestF}{\nu}
\newcommand{\prov}{\epsilon}

\hypersetup{draft}
\begin{abstract}
  Despite the growing popularity of techniques related to graph summarization, a general operator for the flexible nesting of graphs is still missing.
  We propose a novel nested graph data model and a powerful graph nesting operator. In contrast to existing approaches, our approach is able to summarize vertices and paths among vertex groups within a single query. Further on, our model supports partial nestings under the preservation of original graph elements as well as the full recovery of the original graph. We propose an efficient nesting algorithm (THoSP) that is able to perform vertex and path nestings in a single visit of the input graph. Results of an experimental evaluation show that THoSP outperforms equivalent implementations based on graph (Cypher, SPARQL), relational (SQL) and document oriented (ArangoDB) databases.
\end{abstract}

\begin{document}
\fancyhead{}
\title{THoSP: an Algorithm for Nesting Property Graphs}
\author{Giacomo Bergami}
\affiliation{\institution{University of Bologna}}
\email{giacomo.bergami2@unibo.it}

\author{André Petermann}
\affiliation{\institution{University of Leipzig}}
\email{petermann@informatik.uni-leipzig.de}

\author{Danilo Montesi}
\affiliation{\institution{University of Bologna}}
\email{danilo.montesi@unibo.it}
\begin{CCSXML}
	<ccs2012>
	<concept>
	<concept_id>10002951.10002952.10002953.10010146</concept_id>
	<concept_desc>Information systems~Graph-based database models</concept_desc>
	<concept_significance>300</concept_significance>
	</concept>
	<concept>
	<concept_id>10002951.10002952.10003190.10003192.10003210</concept_id>
	<concept_desc>Information systems~Query optimization</concept_desc>
	<concept_significance>300</concept_significance>
	</concept>
	</ccs2012>
\end{CCSXML}

\ccsdesc[300]{Information systems~Query optimization}
\ccsdesc[300]{Information systems~Graph-based database models}
\keywords{Graph Nesting, Nested graphs, Nested Property Graphs}

\fancyhead{}
\maketitle

% !TeX root = 00_nesting_paper-Wall-Pedantic.tex

\section{Introduction}
Graphs allow flexible analyses of relationships among data objects. Thus, graph data management systems play an increasing role in present data analytics. Graphs have been already used as a fundamental data structure to represent data within different contexts such as corporate data \cite{success,Park2016355}, social networks \cite{xie,BrodkaK14} and linked data \cite{Vasilyeva13}.
Despite an increasing number of applications, a general operator that aggregates a single graph in a roll-up fashion is still missing. %partitions which the aforementioned vertices represent.
 The operation of adding structural aggregations to an existing graph is called \textit{graph nesting}.
A respective operator shall not only create a new graph of \textit{nested vertices} and \textit{nested edges}, each containing subgraphs of the original input graph, but also preserve the vertices and edges that are not affected by the actual operation. Further on, the operator must ensure that the nested elements can be freely unnested such that the original graph may be obtained back again. Vertices or edges of the original graph will be called \textit{members} of a nested vertex or edge, if they appear in its underlying subgraph.

In a resulting nested graph, edges connecting nested vertices express that members of the nested vertices are connected by an edge or, more general, by a path in the original graph.
In contrast to this general approach, current literature distinguishes between \textit{vertex summarization} and \textit{path summarization}. Thus, it is not possible to define a single algorithm that evaluates both kinds of patterns at the same time. Before outlining our proposed algorithmic solution, let's have a look on these existing approaches:

The \textit{vertex summarization} strategies group vertices in the manner of the relational \texttt{group by} operation and aggregate edges accordingly \cite{JunghannsPR17}. In this class of operations summarized edges can only be formed by edges that directly connect members of summarized vertices in the original graph. In other words, these approaches cannot freely nest edges, for example, it is not possible to aggregate paths. Since most of vertex summarization techniques are based on graph partitioning, they further provide no support for nested vertices and edges with overlapping members \cite{yin,Tian20085,jakawat}.
Exceptions are HEIDS \cite{ChengJQ16} and Graph Cube \cite{Zhao11}, which perform graph summarizations of one single graph over a collection of non pairwise disjoint subgraphs. However, the union of these underlying subgraphs must be equivalent to the original graph, i.e., it is not possible to take vertices and edges of the original graph over to the summarized graph or to represent outliers that belong to no group.

%To overcome to this graph operation limitation, this thesis proposes the \textbf{graph nesting} operator, thus providing a general graph summarization technique.
%A straightforward implementation  proves to be inefficient, because the visit of such graph collections of size $k$ (to be nested within the graph operand) implies to perform, in the worst case scenario, $|g|^k$ visits of the graph operand $g$: this results into an exponential algorithm, because the size of $k$ may vary, while $|g|$ is fixed. This implies that the graph must be always visited more than once, even if this may not be required. Even though this general operator proves to be inefficient in practice, it allows to detect a broader class of problems and of optimizable algorithms.
%In order to reduce the graph visiting cost from $|g|^k$ to $O(|g|)$, we could use a graph traversal approach: instead of pre-computing $k$ subgraphs of $g$ that are going to be later on used to nest $g$, we can directly perform the graph nesting while visiting the graph, thus allowing  not to perform additional costs for comparing the resulting graphs in a later step. The following example shows how such queries can be efficiently formulated and implemented.
By contrast, \textit{path summarization} techniques allow the aggregation of multiple paths among pairs of source and target vertices that share the same properties.
%At the time of the writing, such approaches can be performed only over (graph) query languages.
Currently, approaches to path summarization can only be found within graph query languages.
%The problem with both path and vertex summarizations is that no general class of either source and target vertices can be used as an outcome of a previous community detection \cite{xie,berlingerio11} or data cleaning and alignment phase \cite{ALIEH17} without rewriting the previously extracted data into an explicit query, thus requiring an additional pre-processing step and thus making such approach not as flexible as required by data integration scenarios. %%initial query each time after different vertex data is extracted, thus not allowing to use such query definition for general data integration scenarios.
%This problem is also reflected by 
These languages also support vertex summarization, but no combination of both approaches in a single step.
%This constraint thwarts the advantages of performing vertex and path summarizations concurrently:
Cypher, the query language of the productive graph database Neo4j, can perform distinct aggregations only within distinct \texttt{MATCH} clauses. SPARQL 1.1, the standard query language of the resource description framework (RDF), requires to combine vertex and path aggregation with a \texttt{UNION} operator, i.e., the same input graph must be visited twice.
%As a result, the query plan optimizers of such query languages do not allow to avoid to visit one same graph more than once whether possible.

This paper shows that such query language limitations can be reduced by using a graph nesting operator, which performs both vertex and path summarization queries concurrently with only a single visit of the input graph. %The following example provides an example of how such query can be formulated and performed. --> Old example
We study specific graph patterns and propose algorithmic optimizations and a specific physical model for efficient graph nesting. Hereby, we show that our algorithmic approaches reduce the time complexity of the visiting and nesting problem. Further on, our optimized data structure requires less indexing time than our competitors.

\begin{figure}[!t]
	\begin{minipage}[!t]{0.5\textwidth}
		\centering
		\includegraphics[width=.8\textwidth]{images/nesting/patterns/04bibliography}
		\subcaption{Input bibliographical network.}
		\label{fig:inputbibex2}
	\end{minipage}
	\medskip
	
	\begin{minipage}[!t]{0.5\textwidth}
		\centering
		\includegraphics[width=\textwidth]{images/nesting/patterns/042nested}
		\subcaption{Nested result: given two \texttt{Author}s $\color{orange}a$ and $\color{orange}a'$, there exist two  \texttt{coauthorship} edges, $\color{blue}a\to a'$ and $\color{blue}a'\to a$ if and only if they share some authored paper contained respectively in $\epsilon({\color{blue}a\to a'})$ and $\epsilon({\color{blue}a'\to a})$. Moreover, each author $\color{orange}a$ is associated to the set of his authored papers $\epsilon({\color{orange}a})$. }
		\label{fig:outputnested}
	\end{minipage}
	\caption{Nesting a bibliographic network: the provenance information is nested within the original node. }
	\label{fig:bibex2}
\end{figure}


\begin{example}
	\label{ex:nestingbib}
	Given a bibliographic network containing (at least) \textsc{Author}s and \textsc{Paper}s as vertices, and where \textsc{authorOf} edges connect each author to the papers he has authored (Figure \ref{fig:inputbibex2}), we want to ``roll up'' the network into a coauthorship network, where each \textsc{Author} is connected by a \textsc{coAuthor} edge with another  \textsc{Author}(2) with which he has published some papers (Figure \ref{fig:outputnested}). In particular, for each resulting \textsc{Author}(2) vertex, nest inside it  its papers as vertices, and nest inside each \textsc{coAuthor} edge all the papers coauthored by  the source and destination \textsc{Author}s. We also want to exclude \textsc{coAuthor} hooks over the same vertex.
	
	
\begin{figure}[!t]
	\centering
	\begin{minipage}[!t]{0.5\textwidth}
		\centering
		\includegraphics[width=.6\textwidth]{images/nesting/patterns/00_vertex_pattern.pdf}
		\subcaption{Vertex summarization pattern ($g_V$). Author is the vertex grouping reference $\gamma_V$.}
		\label{fig:vertexPat}
	\end{minipage} \begin{minipage}[!t]{0.4\textwidth}
		\centering
		\includegraphics[width=1\textwidth]{00_path_pattern.pdf}
		\subcaption{Path summarization pattern ($g_E$). Author$_{src}$ and Author$_{dst}$ are respectively edge grouping references $\gamma_E^{src}$ and $\gamma_E^{dst}$.}
		\label{fig:pathPat}
	\end{minipage}
	\caption{Vertex and Path summarization patterns for the query expressed in Example \ref{ex:nestingbib}. Vertex and edge grouping references are marked by a light blue circled node. As we can see, the vertex grouping reference depicts the same property expressed by edge grouping references.}
	\label{fig:patterns}
\end{figure}
	Figure \ref{fig:patterns} represents the desired vertex ($g_V$) and path ($g_E$) summarization patterns: the former will create a nested \textsc{Author}(2) vertex and the latter will create a \textsc{coAuthor} nested edge. Given that $g_V$ appears twice in $g_E$, we may also pre-istantiate the pattern $g_V$ by visting $g_E$ once. The two patterns have different key roles: while the vertex summarization retrieves all the papers that one author has published and nest them within one single matched author, the path summarizations return all the papers authored by two different authors and creates an edge between the two previously nested vertices. %This construction implies that a join between the two paths must be carried out. 
	
	
	This problem can be solved by visiting the graph only once; %by visiting the graph starting from the vertices:
	if the current vertex is a \textsc{Paper}, traverse backwards all the \textsc{authorOf} edges, thus reaching all of its \textsc{Author}s, that are going to be \textsc{coAuthor} for at least the current paper. Instead of associating the nesting content at the end of the graph visiting process, I can incrementally define the subgraph to be nested by using a separated nesting index: by visiting the two distinct \textsc{Author} vertices adjacent to the current \textsc{Paper}, the latter one shall  be contained in both final \textsc{Author}(2) vertices, thus allowing the definition of a  \textsc{coAuthor} edge. %%The remaining types of vertices and edges %All the other vertices and edges 
	%%may be discarded. %as a starting point for the graph visiting process
	By doing this, only the edges are visited twice, but the vertices are visited only once. Hereby, with these patterns we reach the optimal solution by visiting the graph only once.
	
	%This pattern comparison remarks that, in order to reduce the graph time visit, we must start from visiting the \texttt{Paper}, which is shared among the two distinct patterns, and then keep going with the graph visit by exploring the source and target vertices. 
\end{example}

%There might be other possible patterns that can be optimized, but we're going to focus just on vertex and path summarization patterns where edge grouping references are connected to each other at a 2 edge step distance (Section \ref{sec:THOSPA}). We're also going to show how such optimizations can be detected beforehand by looking at the pattern representation.

The fulfilment of the former scenario is achieved by this paper via the following three contributions:

\begin{itemize}
	\item We implement the aforementioned solution into a nested graph data model, where the logical model differs from the physical one.
	\item \textbf{Graph Nesting Operator} (Section \ref{sec:nestingdef}); we provide a general definition of the nesting operator which combines the path  with the vertex summarization approaches to nesting graphs.
	\item \textsc{{Two HOp Separated Patterns}} \textbf{(THoSP) algorithm for a specific graph nesting task} (Section \ref{sec:THOSPA}): we
	compare it to its implementation
	over both graph  (SPARQL, Cypher), relational (SQL) and document oriented (AQL) query languages. The results of such experiments shows that the sum of both indexing and query evaluation time of our proposed solution outperforms by at least one order of magnitude the aforementioned solutions evaluated on such databases with their respective query languages (Section \ref{sec:nestexpeval}). Consequently, our data model also proved to be curcial in providing an enhanced implementation of the specific graph nesting task.
	%\item A general strategy on how to extend the THoSP algorithm for patterns having vertex and edge grouping references is provided (Section \ref{sec:optimizableClass}).
	%%\item By extending the concept of binary predicates into edges, Edge Joins are introduced as a preliminary step towards the definition of Graph Nesting (Chapter \ref{cha:nesting}).
\end{itemize}

The source code for THoSP is provided at \texttt{\color{red}[Link removed for double-blind review]}.

% !TeX root = 00_nesting_paper.tex
\section{Related Works}

\subsection{Nested Graph Representations}
Statecharts\index{statechsart} \cite{statecharts} were one of the first models representing nested graphs: they were used for representing complex systems at different abstraction levels, where each node represents a  state or ``configuration'' of the system, and each edge represents a transaction between two different states on a given event. Each vertex and edge is labelled, but  they do  not come with attribute-value associations because this model was not designed for data representations. In order to represent different nesting levels, each node can contain other states and edges connecting such states. As a consequence,  there is no distinction between (simple) states and states containing other states.
This model allows both \textbf{external edges}\index{edge!external} and \textbf{internal edges}: we say that edge $e$ is \textit{external} if its source (or target) is contained by the target (or source) but neither of them contains $e$; the edge is called \textit{internal}\index{edge!internal} when the containing vertex (either its source or target) also contains the edge. Besides of state representation purposes, this model was been even used for both modelling the evolution of \textit{pathophysiological} states and to describe the subsequent treatments to which the patient must undergo, where each treatment could be furtherly subdivided in smaller consequential steps to be followed \cite{NestedGlaucoma}.

This model was also adopted as a basis for the \textbf{hypernode} data model \cite{Poulovassilis1994}: even if hypernodes are subsequent to statecharts, they are less expressive than the former ones, because they do not label the edges and they only allow edges between vertices which are contained within the same vertex: the model neither represents external edges nor internal ones. As previously stated for statecharts, even this model does not allow to fully represent a property graph, since the attribute-value association must be necessarily expressed as a relation between two different vertices \cite{Poulovassilis1994}. Last, the fact that the vertex containments cannot overlap make such nested model affected from the same \textit{data replication} representation problem described for semistructured and nested data. A first extension of the hypernode model towards data representation is represented by CoGITaNT \cite{GenestS98}, where any type of edge (thus including internal and external ones) are included and data is firstly contained inside a node. Nested graphs are also supported by GraphML \cite{graphml} and GXL \cite{GXL}.

 Two different approaches have been used to extend current graph data model for supporting nesting operations:
the first ones try to overcome to the basic graph data structure limitations by simply extending the query language, while the other ones try to extend the data structures that are used for both input and intermediate computations. 


Among the first type of approaches, the one outlined in \cite{Etcheverry2012} propose to define a RDF\index{RDF} vocabulary over which the OLAP\index{OLAP} cube can be defined in RDF\index{graph!RDF}. On top of this ``structured'' RDF graph, an algorithm generates the SPARQL query that will allow to perform either the roll-up or the drill-down operation. This later approach implies that each possible computation has to be always recomputed on top of the row data like for classical ROLAP systems: as a consequence, this MOLAP approach does not benefit from the specific RDF representation, that is not able to represent different aggregation levels and to store intermediate computations.

The last type of approaches have been recently widely investigated, and seem to be more promising with respect to optimization techniques: in \cite{Tian20085,ChenYZHY08,Qu2011} graph data structures are associated with external graph indices, thus allowing to connect one graph  to its broader one with respect to the roll-up query. As a consequence, these solutions do not allow to freely expand any aggregated component at a time, but they can only backtrack the aggregation to a previous known state. %Some further details are going to be provided on Chapter \vref{cha:nesting}, where such operator will be implemented on a specific algorithm.
%As it will be showed in Chapter \ref{cha:graphsdef}, in order to meet such goals the nesting indices are going to be directly embedded within the definition of the nested (graph) data model, thus allowing to extend all the aforementioned approaches.


\subsection{Databases and Query Languages}\label{subsec:pathsumm}
We want now to discuss how current query languages can express graph nesting  within their data model of choice. In particular, we must select query languages that either support collections or nested representations allowing to express the same query presented in our running example. The associated proposed listings can be seen in the appendix.

For these reasons, we select PostgreSQL's dialect which, by extending the SQL-3 syntax with JSON data supports, allows to create arrays as a result of a \texttt{GROUP BY} query via \texttt{array\_agg}; therefore, instead of using function aggregating a collection into one single summarized value, we can list all the elements that have been aggregated. In particular, in PostgreSQL's dialect, we  implement our graph by  storing the triples defining an edge as the following relation:
\begin{center}
 \texttt{Edge}(\textit{edgeId},\;\textit{sourceId},\;\textit{edgeLabel},\;\textit{targetId})
\end{center}
Hereby, by grouping the edges by \textit{sourceId} and collecting all the target's ids we obtain a representation of nested vertices. Similarly, if we join the \texttt{Edge} relation with itself and group the join result by two distinct \textit{sourceId} and return the list of all the \textsc{Paper}s that they have in common, we can return the list of all the \textsc{Paper}s that they have coauthored. As showed in Listing \ref{SQLNesting}, the overall graph nesting cannot be created in one single SQL query, because by  the SQL language definition we cannot distinctively group the same dataset in different ways, but we must visit the same data twice and perform two distinct aggregations. 


All the other query languages are going to be affected by the same problem, SPARQL included (Listing \ref{SPARQLNesting}): despite the fact that this query language may represent the graph nesting query as a single statement, even in this case the \texttt{UNION} clause implies a separate visit for the two graph patterns. In particular, the first pattern allows to traverse those graph patterns matching the coauthorship statement in Figure \ref{fig:pathPat} so that they can be nested within the created \textsc{CoAuthorship} edge, while the second part returns all the associations of the \textsc{Paper} that have been authored by one single \textsc{Author}. In particular, the \texttt{OPTIONAL\dots FILTER(!bound(\dots))} syntax is adopted instead of \texttt{FILTER NOT EXISTS}, because the latter is only supported in SPARQL1.1, which is not supported by the current version of \textit{librdf} used to query Virtuoso in our benchmarks. In this case, the edge nesting is performed via the association of different \texttt{<http://cnt.io/nesting>} properties departing from the \texttt{?newedge}  \textsc{CoAuthorship} between two coauthors. Consequently, even in this case the graph visits two times the same graph patterns.


We also consider the AQL query language supported by ArangoDB, because ArangoDB is a NoSQL database relying on a document-oriented storage, which is hereby prone to both represent and return nested content. An example of how such graph nesting query can be carried out in AQL is presented in Listing \ref{AQLQueryNesting}: in this scenario we assume that we've previously loaded our graph data with the default format, where both vertices and edges are fully stored, and where the former are indexed by id, while the latter are also indexed by source and target vertex id. In particular, we can state that this algorithm provides the exact same result as the one produced by the SQL query, except that JSON documents are returned instead of relational tables containing JSON arrays.
%
%Let us consider the following query that will be express in our different languages:
%``\textit{Total number of orders handled by each employee. Only list employees that handled more than 100 orders}''
%
%At this point we immediately notice that NautiLOD and G could not express the summarization since those languages
%are graph traversal languages for pattern matching: consequently they could only select paths and subgraphs but
%they do not aggregate (summarize) nodes. In this case a bag of both employees and the number of the orders
%is returned. Since the result is a bag of values, we cannot obtain a graph as an output, and hence we cannot establish some new edges through the employee and the aggregated value of the sales.
%
%\begin{lstlisting}[caption={Summarization query in Gremlin},language=gremlin,frameround=fttt,frame=trBL]
%graph = TinkerGraph.open()
%graph.io(IoCore.gryo()).readGraph('/path/to/graph')
%g = graph.traversal()
%
%g.V().hasLabel("Employee").match(
%      __.as("emp").in("SalesEmployee").hasLabel("Sales").count()
%                                      .as("ordersByEmployee"),
%      __.as("ordersByEmployee").is(gt(100))
%).select("emp", "ordersByEmployee")
%\end{lstlisting}
%\medskip
%
%The SPARQL query returns a table with two attributes, where the first is the Employee ID and the
%second element is the number of its handled orders. In this case the output is expressed as a table.
%
%\begin{lstlisting}[caption={Summarization query in SPARQL: Table},language=sparql,frameround=fttt,frame=trBL]
%PREFIX ex:<http://example.it/Relations#>
%
%SELECT	 ?emp, (COUNT (?sales) AS ?ordersByEmployee)
%WHERE    {
%          ?sales a                 ex:Sales;
%                 ex:SalesEmployee  ?emp.
%          ?emp   a                 ex:Employee.    
%         }
%GROUP BY ?emp
%HAVING   COUNT(?orderNo) > 100
%\end{lstlisting}
%
%With the \texttt{\textbf{CONSTRUCT}} clause we could return the previous result inside an RDF Graph.
%\begin{lstlisting}[caption={Summarization query in SPARQL: Graph},language=sparql,frameround=fttt,frame=trBL]
%PREFIX ex:<http://example.it/Relations#>
%
%CONSTRUCT { ?ordersByEmployee ex:SalesEmployee ?emp. }
%WHERE {{
%	SELECT	 ?emp, (COUNT (?sales) AS ?ordersByEmployee)
%	WHERE    {
%	          ?sales a                 ex:Sales;
%	                 ex:SalesEmployee  ?emp.
%	          ?emp   a                 ex:Employee.    
%	}
%	GROUP BY ?emp
%	HAVING   COUNT(?orderNo) > 100
%}}
%\end{lstlisting}
%\medskip
%
%In Cypher we could formulate a similar query with a tabular result as follows:
%\begin{lstlisting}[caption={Summarization query in Cypher: Table},language=cypher,frameround=fttt,frame=trBL]
%MATCH  (sales:Sales)-[:SalesEmployee]->(empl:Employee)
%WITH   empl AS emp, COUNT(sales) AS ordersByEmployee
%WHERE  ordersByEmployee > 100
%RETURN empl, ordersByEmployee
%\end{lstlisting}
%
%With this language we could even return a new graph, where
%the whole information of the employee is returned and where the films are aggregated.
%\begin{lstlisting}[caption={Summarization query in Cypher: Graph},language=cypher,frameround=fttt,frame=trBL]
%MATCH  (sales:Sales)-[:SalesEmployee]->(empl:Employee)
%WITH   empl AS emp, COUNT(sales) AS ordersByEmployee
%WHERE  ordersByEmployee > 100
%CREATE p=
%(:Sales {count: ordersByEmployee})-[:SalesEmployee]->(empl:Employee)
%RETURN p
%\end{lstlisting}
%\medskip 
%
%Let us now focus on the BiQL language: firstly we cannot create new vertices that aggregate the results without
%updating the original database because the \texttt{\textbf{CREATE}} semantic has this precise meaning, secondly the
%\texttt{\textbf{CREATE}} clause does not allow to create multiple objects within the same query: this means that we 
%cannot create new edges while creating new vertices. By the way we could return all the employees and store the
%number of the sales inside each node.
%\begin{lstlisting}[caption={Summarization query in Cypher: Graph},language=biql,frameround=fttt,frame=trBL]
%SELECT <empl>{empl.*, ordersByEmployee: count(salesEdge)}
%FROM   Employee empl <- SalesEmployee salesEdge 
%WHERE  count(salesEdge) > 100
%\end{lstlisting}
%

Last, Listing \ref{Neo4JQuery} provides an example of Neo4J allowing to nest property graphs: even in this case the property graph model does not directly  nest the graphs inside one element. Similar to the previous approaches, we can group by all the graphs returned by the graph pattern by selecting the vertices of interest, and nesting the to-be-grouped remaining objects inside a collection. In particular, we can first match the vertex summarization pattern in Figure \ref{fig:vertexPat} and group by \textsc{Author}, and nest the collection of authored \textsc{Paper}s within the to-be-created nested vertex; similarly, we can first match the path summarization pattern presented in Figure \ref{fig:pathPat} by source and destination \textsc{Author}, and then create an edge between the previously created nested vertices by collecting all the coauthored \textsc{Paper}s appearing in the original graph. Moreover, the Neo4J property graph data model implies that we cannot create an edge if the vertices are not previously known beforehand and, therefore, we always must join the nested vertices with the original matched ones in order to reconstruct the original information and perform the actual matching operation. As it will be observed within the benchmarks, the   solution of not separating the elements' ids from their data quickly leads to an intractable solution. 
% !TeX root = 00_paper_entrypoint.tex

\section{Nested Graphs}\label{sec:model}
\begin{table}
\begin{adjustbox}{max width=.48\textwidth}
\begin{tabular}{c|l|c}
\toprule
\parbox[t]{2mm}{\multirow{6}{*}{\rotatebox[origin=c]{90}{$G$, NGDB}}} & $\mathcal{V},\mathcal{E}$ & vertex/edge indices in $\mathbb{N}^2$\\
                                 & $\lambda$ & edge to source-target vertices function\\
                                 & $\ell$ & vertex/edge multilabelling in $\Sigma^*$\\
                                 & $\omega$ & vertex/edge to tuple function\\
                                 & $\nu,\varepsilon$ & vertex/edge containment functions for $\mathcal{V}\cup\mathcal{E}$\\
                                 & $G_o$ & nested graph induced by vertex/edge $o\in\mathcal{V}\cup\mathcal{E}$\\
\midrule
\parbox[t]{2mm}{\multirow{10}{*}{\rotatebox[origin=c]{90}{$\eta$ operator dependencies}}} & $dt$ & index dovetailing function \\
                                 & $f\colon D\xrightarrow{f}C$ & function $f$ with domain $D$ and codomain $C$\\
                                 & $a\mapsto b$ & finite lambda function with domain $\{a\}$\\
                                 & $\oplus$ & finite function extension\\         
                                 & $\kappa$ & (graph) pattern, i.e. multilabelled graph classifier\\
                                 & $g_\kappa$ & nested graph classifier  over $\kappa$\\ 
                                 & $f_C$ & morphism from pattern to subgraph $G_C$\\
                                 & $\iota_{G_o}$ & indexing function for subgraphs $G_C\subseteq G_o$\\   
                                 & $\mu_\Omega$ & object user defined function\\  
                                 & $\mu_E$ & edge user defined function\\      
\bottomrule
\end{tabular}
\end{adjustbox}
\caption{Table Of Notations}
\vspace{-2em}
\end{table}
The term \textit{property graph}  \cite{angles12} usually refers to a directed, labelled and attributed multigraph. 
% In other words, if there is a schema, each vertex and edge will be represented by a relational tuple and without, by a document of key-value pairs. 
In a property graph a collection of \textit{labels} \cite{bergamimm17} is associated to every vertex and edge (e.g., \texttt{[}\mstr{Author}\texttt{]}) or \texttt{[}\mstr{coAuthorsip}\texttt{]}). Further on, vertices and edges may have arbitrary named attributes (\textit{properties}) in the form of key-value pairs (e.g., \texttt{\textbf{name}:Baldwin} or \texttt{\textbf{surname}:Oliver}). Property-value associations of vertices and edges can be represented by relational tuples; this is a common approach in literature used even when graphs have no fixed schema \cite{angles12}. We define the\textit{ nested (property) graph database} as the following extension of the property graph data model for nested information:

\begin{definition}[Nested Graph DataBase]
Given a set $\Sigma^*$ of strings,
	a \textbf{nested (property) graph database} $G$ is a tuple $G=\Braket{\VS, \ES, \lambda,\ell,\omega,\nestF,\prov}$, where $\VS$ and $\ES$ are disjoint sets, respectively referring to vertex and edge identifiers $o\equiv(c,i)\in\mathbb{N}^2$; $c$ is an incremental unique number associated to each graph. In particular, input data graphs have $c=0$ while  nested graphs created at nesting steps exhibit $c>0$. 
	
	A function $\lambda\colon \ES\to \VS^2$ maps each edge to its source and target vertex. Each vertex and edge is assigned to multiple possible labels through the labelling function $\ell:\VS\cup \ES\to \wp(\Sigma^*)$.  $\omega$ is a function mapping each vertex and edge into a relational tuple.
	
	In addition to the previous components defining a property graph, we also introduce functions representing \textit{vertex members} $\nestF\colon (\VS\cup \ES)\to\wp(\VS)$ and \textit{edge members} $\prov\colon(\VS\cup \ES)\to \wp(\ES)$. These functions induce the nesting by associating a set of vertices or edges to each vertex and edge. Each vertex or edge $o\in V\cup E$ induces a \textbf{nested (property) graph} as the following pair:
	\[G_o=\Braket{\nu(o),\Set{e\in\epsilon(o)|\lambda(e)\in (\cup_{n\geq 0}\;{\nu\epsilon}^{(n)}(\{o\}))^2}}\]
	where ${\nu\epsilon}$ returns the vertices contained in both vertices and edges ($\nu\epsilon(x)=\nu(x)\cup \nu(\epsilon(x))$). We denote $f(X){:=}\bigcup_{x\in X} f(x)$ when $X\subseteq \textup{dom}(f)$
\end{definition}


Since the member functions $\nu$ and $\epsilon$ induce the expansion of each single vertex or edge to a graph, we must avoid recursive nesting to support expanding operations.
% This condition is fundamental in order to define different levels as different abstractions over the data. 
Therefore, we additionally introduce the following constraints to be set at a nested property graph database level:

\begin{axiom}[Recursion Constraints]
	For each correctly nested property graph, each vertex $v\in \VS$ must not contain $v$ at any level of containment of $\nu$ and, any of its descendants $m$ must not contain $v$:
	\[\forall v\in \VS. \forall m\in \nu^+(v).\;\; m\neq v\wedge v\notin \cup_{n\geq 1}\;{\nu\epsilon}^{(n)}(m)\]
	Similarly to vertices, any edge shall not contain itself at any nesting level:
	\[\forall e\in \ES. \forall m\in \epsilon^+(e). m\neq e\wedge e\notin \cup_{n\geq 1}\;{\epsilon\nu}^{(n)}(m)\]
	where ${\epsilon\nu}$ returns the edges contained in both vertices and edges ($\epsilon\nu(x)=\epsilon(x)\cup \epsilon(\nu(x))$)
\end{axiom}

A vertex $v$ having a non-empty vertex or edge members is called \textbf{nested vertex}, while vertices with no members are simply referred to \textbf{simple vertices}. For edges, we respectively use the terms \textbf{nested edges} and \textbf{simple edges}. 

\begin{ex}[label=exImpl]
The property graph in Figure \ref{fig:inputbibex2} can be represented by the graph $G_{(0,11)}$, which is a nested vertex contained in the following nested graph database:
\[G=\Braket{\{(0,0),(0,1),\dots,(0,5),(0,11)\}, \{(0,6),\dots,(0,10)\},\lambda,\ell,\omega,\nu,\epsilon}\]

The nested vertex $(0,11)$  represents a \mstr{Bibliography} graph ($\ell(0,11)=[\mstr{Bibliography}]$), to which an empty tuple is associated ($\omega(0,11)=\{\}$). Its vertex ($\nu$) and edge ($\epsilon$) members are defined as follows:
\[\nu(0,11)=\{(0,0),\dots,(0,5)\}\quad\epsilon(0,11)=\{(0,6),\dots,(0,10)\}\]

The simple edge $6$ within the property graph in Figure \ref{fig:inputbibex2} ($\nu(0,6)=\epsilon(0,6)=\emptyset$) has now id $(0,6)$; it has one label, $\ell(0,6)=[\mstr{AuthorOf}]$, and it is associated to an empty tuple ($\omega(0,6)=\{\}$).
The source and target vertices are 
$\lambda(0,6)=\Braket{(0,0),\;(0,3)}$. Similar considerations can be carried out for each  remaining edge.

The simple vertex $0$ in the same Figure has id $(0,0)$ in the present example; such vertex refers to the \mstr{Author} \texttt{Abigail Conner}. This information is represented as follows:
\[\ell(0,0)=[\mstr{Author}]\quad\nu(0,0)=\epsilon(0,0)=\emptyset\] \[\omega(0,0)=\{\texttt{\textbf{name}}\colon\texttt{Abigail},\texttt{\textbf{surname}}\colon\texttt{Conner}\}\]
Similar considerations can be carried out for each remaining vertex.

\end{ex}

% !TeX root = 00_paper_entrypoint.tex


\begin{figure*}[!ht]
	
\begin{tabular}{c}
	{$AdjFile[o]\equiv\Braket{\textbf{\textsc{Header}},\nu(o),\epsilon(o),\Braket{\ell(o),\omega(o)},\texttt{outLen},\textit{\textsc{OutgoingEdges[]}},\texttt{inLen},\textit{\textsc{IngoingEdges[]}}}$} \\
	$\textit{\textsc{OutgoingEdges[}i\textsc{]}}\equiv\textit{\textsc{IngoingEdges[}i\textsc{]}}\equiv \textsc{\textbf{EdgeEntry}[}i\textsc{]}$ \\
	\linebreak \\
	{$\textbf{\textsc{Header}}:=\Braket{\texttt{length},\texttt{id},\texttt{hash},\texttt{ellOffset},\texttt{epsilonOffset},\texttt{contentOffset},\texttt{outOffset},\texttt{inOffset}}$}\\ $\textsc{\textbf{EdgeEntry}[}i\textsc{]}:=\Braket{\textsc{id[}i\textsc{]},\textsc{hash[}i\textsc{]},\textsc{adjVertexId[}i\textsc{]},\textsc{adjVertexHash[}i\textsc{]}}$\\
\end{tabular}
	\caption{Serialized data structure representing an extended adjacency list for one nested vertex $o$. The header contains some basic information (such the representation size of $o$, its id and associated hash) and the offsets to the remaining fields. $\nu$ and $\epsilon$ are empty when the serialized graph represents a basic property graph as the one in Figure \ref{fig:inputbibex2}.}\label{nestedGraphVertex}
\end{figure*}
\section{Graph Nesting}\label{sec:nestingdef}
We now define the graph nesting operation, which uses the outcome of the following partial and overlapping graph clustering operator (e.g., graph pattern matching) to provide structural aggregation:
%The graph nesting operator uses a classifier function grouping all the vertices and edges that shall appear as a member of a cluster $C$. 

\begin{definition}[Nested Graph Classifier, $g_\kappa$]
	Given a set of cluster labels $\;\mathcal{C}$, a \textbf{nested graph classifier} function $g_\kappa$ maps a nested graph $G_o$ into a nested  graph collection $\{G_C\}_{C\in\mathcal{C},G_C\neq \emptyset}$ of subgraphs of $G_o$. Such function uses a classifier function $\kappa\colon \VS\cup \ES\to \wp(\mathcal{C})$ mapping each vertex or edge in either no graph or at least one non-empty subgraph. Each nested graph $G_C$ is a pair $G_C=\Braket{\VS_C,\ES_C}$
	where $\VS_C$ (and $\ES_C$) is the set of all the vertices $v$ (and edges $e$) in $G_o$ having $C\in \kappa(v)$ (and $C\in \kappa(e)$). Therefore, the nested graph classifier is defined as follows:
	\[g_\kappa(G_o)=\Set{\Braket{\VS_C,\ES_C}|C\in\mathcal{C},(\VS_C\neq\emptyset\vee\ES_C=\emptyset)}\]
\end{definition}

When $\kappa$ is a pattern matching graph as in Figure  \ref{fig:patterns} as possible $\kappa$,   $\kappa\xrightarrow{f_C} G_C$ is the function $f_C$ associating  each vertex (and edge) in $\kappa$ to possibly more than one vertex (and one edge) in a subgraph $G_C\in g_\kappa(G_o)$. In order to represent the latter subgraphs  as either vertices and edges, we may use
the following \textsc{User-Defined Functions}:
\begin{definition}[User-Defined Functions]
	An \textbf{object user defined function} $\mu_\Omega$ maps each subgraph $G_C\in g_\kappa(G_o)$ into a pair $\mu_\Omega(G_C)=(L,t)$, where $L\in\wp(\Sigma^*)$ is a set of labels and $t$ is a relational tuple.
	
	An \textbf{edge user defined function} $\mu_E$ maps each subgraph $G_C\in g_\kappa(G_o)$ into a pair of identifiers $\mu_E(G_C)=(s,t)$ where $s,t\in\mathbb{N}$.
\end{definition}

\begin{ex}
Within our use case scenario, $\mu_\Omega$  must associate  the authors' informations to each nested vertex resulting from $g_{V}(G_{o})$ , and create nested edges with \textsc{coAuthorship} label and no associated tuple:
\[\mu_\Omega(G_C)=\begin{cases}
([\mstr{coAuthorship}],\;\emptyset) & G_C \in g_E(G_{o})\\
(\ell(f_C(\gamma_V))),\;\omega(f_C(\gamma_V))) & G_C \in g_V(G_{o})\\
\end{cases}\]
\end{ex}

While $\mu_\Omega$ may be used for transforming subgraphs to both vertices and edges, $\mu_E$ is only used to map subgraphs to edges.  
In order to complete such transformation, we have to map each graph in $g_\kappa(G)$ into a new id $(\textbf{c},\textbf{i})\notin \VS\cup \ES$, for which an indexing function $\iota_G$ over each $G_C$ has to be defined within our specific task. As we will see in the next section, our scenario provides some constraints on both patterns; this allows the definition of an indexing function  uniquely associating each matched subgraph  $G_C$  to the grouping references' ids.
The previous functions are involved in the definition of our general graph nesting operator, generalizing current literatures' graph summarization and roll-up operators:



\begin{definition}[Graph Nesting]
Given a nested graph $G_{(c,i)}$ within a nested graph database $G$, an object user defined function $\mu_\Omega$, an edge user defined function $\mu_E$ and an indexing function $\iota_G$, the graph nesting operator $\eta_{g_V,g_E,\mu_\Omega,\mu_E,\iota_G}^{\textbf{keep}}$ converts each subgraph in $G_C\in g_V(G_{(c,i)})$ (and $G_C\in g_E(G_{(c,i)})$) into a nested vertex (and nested edge) $\iota_G(G_C)$ and adds them in a newly-created nested vertex; vertices and edges in $G_{(c,i)}$ appearing neither in a nested vertex nor in a nested edge may be also returned if $\textbf{keep}$ is set to \texttt{true}. This operator returns the following nested graph:
\[\begin{split}
\eta&{}_{g_V,g_E,\mu_\Omega,\mu_E,\iota_G}^{\textbf{keep}}(G_{(c,i)})=G_{(\overline{c},dt(i))}=\\
&=\Big\langle \{v\in \nu(c,i) | V(v)=\emptyset\wedge\textbf{keep} \}\cup \iota_G(g_V(G_{(c,i)})),\\
&\qquad \{e\in \epsilon(c,i) | E(e)=\emptyset\wedge\textbf{keep} \}\cup \iota_G(g_E(G_{(c,i)}))\Big\rangle\\
\end{split}\]
where $\overline{c}=\max\{c|(c,i)\in\mathcal{V}\cup\mathcal{E}\}+1$. As a side effect of the graph nesting operation, the nesting graph database is updated using the nested graph classifier and user defined functions as follows:
	\begin{align*}
	\Big\langle&\VS\cup \iota_G(g_V(G_{(c,i)}))\cup\{(\overline{c},i)\},\quad \ES\cup \iota_G(g_E(G_{(c,i)})),\\
	& \lambda\oplus \bigoplus_{G_C\in g_E(G_{(c,i)})}\iota_G(G_C)\mapsto \mu_E(G_C),\\
	& \ell\oplus\bigoplus_{G_C\in g_E(G_{(c,i)})\cup g_V(G_{(c,i)})}\iota_G(G_C)\mapsto\texttt{fst}\;\mu_\Omega(G_C),\\
	& \omega\oplus\bigoplus_{G_C\in g_E(G_{(c,i)})\cup g_V(G_{(c,i)})}\iota_G(G_C)\mapsto\texttt{snd}\;\mu_\Omega(G_C),\\
	& \nu\oplus \bigoplus_{G_C\in g_E(G_{(c,i)})\cup g_V(G_{(c,i)})}\iota_G(G_C)\mapsto \mathcal{V}_C\\
	& \;\; \oplus (\overline{c},dt(i))\mapsto \{v\in \nu(c,i) | V(v)=\emptyset\wedge\textbf{keep} \}\cup \iota_G(g_V(G_{(c,i)})), \\
	& \epsilon\oplus \bigoplus_{G_C\in g_E(G_{(c,i)})\cup g_V(G_{(c,i)})}\iota_G(G_C)\mapsto \mathcal{E}_C\\
	& \;\; \oplus (\overline{c},dt(i))\mapsto\{e\in \epsilon(c,i) | E(e)=\emptyset\wedge\textbf{keep} \}\cup \iota_G(g_E(G_{(c,i)}))\Big\rangle\\
	\end{align*}
	where $(f\oplus g)(x)$ returns $g(x)$ if $x\in\textup{dom}(g)$ and $f(x)$ otherwise, and both $f$ and $g$ are finite domain functions. $a\mapsto b$ denotes a finite function, which domain contains only $a$.
\end{definition}

The following example describes the outcome of the graph nesting process.

\begin{ex}[label=exNest]
Figure \ref{fig:outputnested} provides the result of $\eta$ when the non-traversed vertices and edges are not preserved ($\textbf{keep}=\texttt{false}$) and where $V$ and $E$ are the ones represented in Figure \ref{fig:bibex2}. As showed by the former definition, the nesting operation updates the nested graph database by creating new nested vertices ($(1,0),(1,1),(1,2)$) and nested edges ($(1,3),(1,5),(1,7),(1,8)$). Such nested components are contained within the returned nested graph $G_{(1,11)}$, which is represented as a nested vertex with the following members:
\[\nu(1,11)=\{(1,0),(1,1),(1,2)\}\quad \epsilon(1,11)=\{(1,3),(1,5),(1,7),(1,8)\}\]
The nested graph database updated as a side effect of the graph nesting may be represented as follows:
\[\begin{split}
G'=\big\langle &\{(0,0),(0,1),\dots,(0,5),(0,11),(1,0),(1,1),(1,2),(1,11)\},\\
	  & \{(0,6),\dots,(0,10),(1,3),(1,5),(1,7),(1,8)\}, \\
	  & \lambda',\ell',\omega',\nu',\epsilon'\big\rangle\\
\end{split}\]
Let us now focus on the nested vertices and edges of $G_{(1,11)}$. As requested by the UDF functions, each resulting nested \textsc{Author}(2) preserves the original  vertices' tuple information, and its vertex members correspond to the \textsc{Paper}s authored by the corresponding \textsc{Author}(1). For easing the nested graph representation, we assume that each \textsc{Author}(2) has an associated  id $(1,i)$, which derives from a simple vertex with id $(0,i)$ in $G_{(0,11)}$. Therefore, vertex $(1,0)$ is represented as follows:
\[\ell'(1,0)=[\mstr{Author}]\quad\nu'(1,0)=\{(0,3)\}\quad \epsilon'(1,0)=\emptyset\] \[\omega'(1,0)=\{\texttt{\textbf{name}}\colon\texttt{Abigail},\texttt{\textbf{surname}}\colon\texttt{Conner}\}\]

Last, each resulting nested edge \textsc{coAuthorship} has a \mstr{coAuthorship} label, it has no tuple information and its vertex members correspond to the \textsc{Paper}s coauthored by source and target \textsc{Paper}.
For easing the nested graph representation, we assume that each \textsc{coAuthorship} edge $(1,a)\to (1,a')$ has an associated id $(1,\sum_{k=0}^{a+a'}k\;+\;a')$, which derives from the grouping references. Therefore, edge $0\to 2$ in Figure \ref{fig:outputnested} is represented as follows:
\[\ell'(1,5)=[\mstr{coAuthorship}]\;\nu'(1,5)=\{(0,3)\}\;\epsilon'(1,0)=\emptyset\; \omega'(1,0)=\{\}\] 
\end{ex}


\section{Two HOp Separated Patterns Algorithm} \label{sec:THOSPA}
Walking on the footsteps of relational databases, where algorithms were provided for specific instances of relational operators, THoSP provides an implementation for a specific graph nesting task. We now propose a sequential algorithm requiring a preliminary phase, where the operand is loaded into secondary memory using the \textbf{input data  representation}, and where no ancillary indexing data structures are serialized. Such representation is presented in Figure \ref{nestedGraphVertex}: it is an extension of the graph adjacency lists serialization previously proposed in \cite{bergamimm17} where each vertex $o$ has an associated \textbf{\textsc{Header}} containing its id ($o$), its associated hash and the offset pointing to other serialzied fields, such as the labelset and eventually its property-value representation ($\Braket{\ell(o),\omega(o)}$). Last, for each edge we store its id and hash value, as well as the hash and the id of the adjacent vertex. Hash values are used within the proposed THoSP algorithm to store the correspondences with the graph patterns in Figure \ref{fig:patterns}; therefore, each $\ell(o)$ is associated to a distinct hash value  $h(o)$. We also suppose that the input graph data to be serialized does not represent an exact adjacency list: for this reason, the graph is initially created in primary memory without the offset information and afterwards serialized into secondary memory. 

In order to solve our specific graph nesting problem as presented in Example \vref{ex2}, we have to formally determine the $\eta$ parameters representing THoSP.
We focus on vertex (and edge) summarization patterns which grouping references associate unique vertices to distinct matching subgraphs; they require that:
\[\forall G_C\in g_V(G_o).\neg\exists G_d\in g_V(G_o).\; G_c\neq G_d\wedge f_C(\gamma_V)=f_D(\gamma_V)\]
\[\begin{split}
\forall G_C\in g_E(G_o).\neg\exists &G_d\in g_E(G_o).\; G_c\neq G_d\\
 &\wedge f_C(\gamma_E^{src})=f_D(\gamma_E^{src})\wedge f_C(\gamma_E^{dst})=f_D(\gamma_E^{dst})\\
\end{split}\]
This requirement leads to a one-to-one mapping between subgraphs $G_C$ and vertices matched by vertex (or edge) grouping references, that can be expressed by the following indexing function:
	\[\iota_G(G_C)=\begin{cases}
(\overline{c},dt(\texttt{snd}f_C(\gamma_V))) & G_C \in g_V(G_{o})\\
(\overline{c},dt\left(\texttt{snd}f_C(\gamma_E^{src}),\;\texttt{snd}f_C(\gamma_E^{dst})\right)) & G_C \in g_E(G_{o})\\
\end{cases}\]
$dt$ is an arbitrary bijection associating a  $n$-tuple in $\mathbb{N}^n$ to one single integer $\mathbb{N}$ (see Appendix B).
This assumption permits a deterministic $\mu_E$ function, associating to each newly created nested edge from $G_C$  two nested vertices having $f_C(\gamma_E^{src})$ and $f_C(\gamma_E^{dst})$ as vertex grouping references:
	\[\mu_E(G_C)=\Braket{(\overline{c},dt(\texttt{snd}f_C(\gamma_E^{src}))),\;(\overline{c},dt(\texttt{snd}f_C(\gamma_E^{dst})))}\]
Please note that the former function provides such association without additional join costs. 

\begin{algorithm}[!t]
	\caption{Two HOp Separated Patterns Algorithm (THoSP)}\label{alg:THoSPAlgorithm}
	\begin{adjustbox}{max width=\textwidth}
		\begin{minipage}{1\linewidth}
			\algrenewcommand\algorithmicindent{1em}
			\begin{algorithmic}[1]
				\Procedure{doNest}{$Index,\textsc{pattern},f, \protect\overrightarrow{memb} =\{m_1,\dots,m_n\}$}
				\For{\textbf{each} $m_i\in \protect\overrightarrow{memb}$ \textbf{s.t.} \textsc{pattern}($\protect\overrightarrow{memb}$).doSerialize($m_i$)}\label{patternrequire}
				\State{$Index$.write($\Braket{f,m_i}$)} 
				\EndFor
				\EndProcedure
				\State
				
				\Procedure{$\eta^{\texttt{false}}_{g_{V}\;,g_{E},\dots}$}{$\nested$}
				\State \textsc{File} $AdjFile$ = \textsc{Open\_MemoryMap}($\nested$);\label{openMMAP}\Comment{Serialized operand}
				\State \textsc{File} $Nesting$ = \textsc{Open}(\textbf{new}); \Comment{$\nu \cup\epsilon$, member information}
				\State \textsc{Adjacency} $toSerialize$ = \par \textbf{new} \textsc{Map<Vertex,<Edge,Vertex>>}(); \Comment{Nested graph, adj. list}
				\State {$\alpha:=V\cap {E}\backslash(\gamma_V\cup\gamma_E^{src}\cup\gamma_E^{dst})$;}\Comment{Shared pattern} \label{restriction} 
				\For{\textbf{each vertex} $v'$ in $AdjFile$}\Comment{$v':=(c,v)$}\label{firstJoin}
				\If{$v'\vDash\alpha$}\label{vdashalpha}
				\For {\textbf{each} $ (u',e,v')\vDash V$}\Comment{$u':=(c',u)$} \label{substantiallyIs}
				\State{$\overline{u}:=(\overline{c},dt(u))$} \Comment{$\iota_G$, nested vertex}
				\State{\textsc{doNest}($Nesting$, V, $\overline{u},\{u',e,v'\}$)}
				\For {\textbf{each} $ (w',e,v')\vDash V$}\Comment{$w':=(c'',w)$} \label{substantiallyIs2}
				\If{$ (u',e,v',e',w')\vDash E$}\label{thereIsEdge}
				\State{$\overline{w}:=(\overline{c},dt(w))$}  \Comment{$\iota_G$, nested vertex}
				\State{$\overline{e}:=(\overline{c},dt(u,w))$} \Comment{$\iota_G$, nested edge} 
				\State{\textsc{doNest}($Nesting$, E, $\overline{e},\{u',e,v',e',w'\}$)}
				\State{$toSerialize$.put($\overline{u}$,$\Braket{\overline{e},\overline{w}}$)}
				\EndIf
				\EndFor
				\EndFor
				\EndIf
				\EndFor
				\State $AdjFile$.serialize($toSerialize$);
				\State \Return{($AdjFile$,$Nesting$)}\label{serialize}\Comment{Nested graph}
				
				\EndProcedure
			\end{algorithmic}
		\end{minipage}
	\vspace{-2em}
	\end{adjustbox}
\end{algorithm}

Algorithm \ref{alg:THoSPAlgorithm} provides the desired interpretation for the two pattern matching graphs returning the desired nested graph. 
%provides the desired implementation of the THoSP algorithm using the outcome of the previous preprocessing. We first restrict $\alpha$ to one single vertex and two edges 
 After opening the previously-loaded graph operand through memory mapping (Line \ref{openMMAP}), we must first identify a sub-pattern $\alpha$ (Line \ref{restriction}) that is going to be visited only once within the graph (Line \ref{vdashalpha}), after which either the vertex or the path summarization pattern can be visited in their entirety. We also perform some restrictions over these patterns enhancing such optimizations: for each vertex $v'$ matched by $\alpha$ (Line \ref{vdashalpha}) we know that we must (possibly) visit all the edges going from $v'$ towards the vertices $\gamma_E^{src}$ and $\gamma_E^{dst}$.  Therefore, having an edge as a constraint in $\alpha$ linking $v$ towards $\gamma_E^{src}$ or $\gamma_E^{dst}$ both in $E$ and $V$  reduces the graph visiting time to the actual edges traversed from $v'$ meeting the grouping references (Line \ref{thereIsEdge}). Therefore, we know when we finish  our patterns' instantiation after exhaustively matching all the elements within the pattern.
As a consequence, a ``path join'' is performed between the two nested patterns (Line \ref{firstJoin} with \ref{substantiallyIs2}): this is evident from the two vertex nested for loops appearing in the algorithm.  


\begin{table*}[!t]
	\centering
\begin{minipage}[!t]{0.47\textwidth}
	\begin{adjustbox}{max width=\textwidth}
		%\begin{minipage}[b]{\textwidth}
		\centering
		\begin{tabular}{@{}c|rrrr|r@{}}
			\toprule
			\multicolumn{1}{c}{\textbf{Operand}} & \multicolumn{5}{|c}{\textbf{Operand Loading and Indexing Time (C/C++)} (ms)}  \\
			$|V|$  & {PostgreSQL} & {Virtuoso} & {ArangoDB}  &  {Neo4J (Java)} & {\textbf{Nested Graphs}}  \\
			\midrule
			$10$   & 8.00$\cdot 10^0$ & 3.67$\cdot 10^0$ & 4.30$\cdot 10^1$ & 3.95$\cdot 10^3$  & \textbf{1.30}$\cdot 10^{-1}$\\
			$10^2$  & 1.80$\cdot 10^1$ & 6.86$\cdot 10^0$ & 2.67$\cdot 10^2$ &  4.12$\cdot 10^3$ & \textbf{3.30}$\cdot 10^{-1}$\\
			$10^3$  & 4.50$\cdot 10^1$ & 2.35$\cdot 10^1$ & 1.28$\cdot 10^3$ & 5.25$\cdot 10^3$ & \textbf{3.51}$\cdot 10^0$\\
			$10^4$   & 2.25$\cdot 10^2$ & 3.71$\cdot 10^2$ & 1.15$\cdot 10^4$ &  1.12$\cdot 10^4$ & \textbf{3.18}$\cdot 10^1$\\
			$10^5$   & 1.87$\cdot 10^3$ & 3.51$\cdot 10^3$ & 1.35$\cdot 10^5$ &  1.19$\cdot 10^6$ & \textbf{3.37}$\cdot 10^2$ \\
			$10^6$  & 1.91$\cdot 10^4$ & 3.46$\cdot 10^4$ & 1.36$\cdot 10^6$ & $>$1H & \textbf{3.69}$\cdot 10^3$\\
			$10^7$   & 1.84$\cdot 10^5$ & 3.64$\cdot 10^5$ & $>$1H & $>$1H & \textbf{4.40}$\cdot 10^4$\\
			$10^8$  & 1.98$\cdot 10^6$ & $>$1H & $>$1H & $>$1H & \textbf{5.18}$\cdot 10^5$\\
			\bottomrule
		\end{tabular}
	}
	%\end{minipage}
	\subcaption{\textit{Operand Loading and Indexing Time}. PostgreSQL and Neo4J have transactions, while Virtuoso and ArangoDB are transactionless. Nested Graphs are our transactionless proposed method.}
	\label{tab:storeevaluation}
\end{adjustbox}
\end{minipage}\quad \begin{minipage}[!t]{0.45\textwidth}
\begin{adjustbox}{max width=\textwidth}
	%\begin{minipage}[b]{\textwidth}
	\centering
	\begin{tabular}{@{}cr|rrrr|r@{}}
		\toprule
		\multicolumn{2}{c}{\textbf{Operands Size}} & \multicolumn{5}{|c}{\textbf{\textsc{Two HOp Separated Pattern} Time (C/C++)} (ms)}  \\
		$|V|$  & \#Subgraph  &  {SQL+JSON} & SPARQL & AQL  &  Cypher &{\textbf{THoSP}}  \\
		\midrule
		$10$ & $3$  & 2.10$\cdot 10^0$ &  1.10$\cdot 10^1$ & 3.89$\cdot 10^0$  & 6.81$\cdot 10^2$  & \textbf{1.10}$\cdot 10^{-1}$\\
		$10^2$ & $58$  & 9.68$\cdot 10^0$ &  6.30$\cdot 10^1$ & 1.23$\cdot 10^1$  &  1.94$\cdot 10^3$ & \textbf{1.40}$\cdot 10^{-1}$\\
		$10^3$ & $968$  & 1.80$\cdot 10^1$ & 6.30$\cdot 10^1$ & 1.50$\cdot 10^1$ & $>$1H & \textbf{4.60}$\cdot 10^{-1}$\\
		$10^4$ & $8,683$  & 6.92$\cdot 10^1$ & 3.64$\cdot 10^2$ & 4.67$\cdot 10^1$ &  $>$1H & \textbf{4.07}$\cdot 10^0$\\
		$10^5$ & $88,885$  & 2.94$\cdot 10^2$ & 4.15$\cdot 10^3$ & 5.09$\cdot 10^2$ &  $>$1H & \textbf{4.38}$\cdot 10^1$ \\
		$10^6$ & $902,020$  & 2.61$\cdot 10^3$ & 5.03$\cdot 10^4$ & 7.21$\cdot 10^3$ & $>$1H & \textbf{5.63}$\cdot 10^2$\\
		$10^7$ & $8,991,417$  & 2.57$\cdot 10^4$ & 6.72$\cdot 10^5$ & 9.22$\cdot 10^5$ & $>$1H & \textbf{8.20}$\cdot 10^3$\\
		$10^8$ & $89,146,891$  & 3.96$\cdot 10^5$ & $>$1H & $>$1H & $>$1H & \textbf{9.18}$\cdot 10^4$\\
		\bottomrule
	\end{tabular}
}
\subcaption{\textit{Graph Nesting Time}. The columns in this table provide the query languages used to benchmark }\label{tab:querytimeeval}
%\end{minipage}
\end{adjustbox}
\end{minipage}
\caption{Two HOp Separated Pattern. Experimental evaluations over gMark. We set a timeout at $1H=3.6\cdot 10^{6}$ ms.}
\vspace{-2em}
\end{table*}
\begin{table*}[!t]
	\centering
	\begin{minipage}[!t]{0.47\textwidth}
		\begin{adjustbox}{max width=\textwidth}
			%\begin{minipage}[b]{\textwidth}
			\centering
			\begin{tabular}{@{}c|rrrr|r@{}}
				\toprule
				\multicolumn{1}{c}{\textbf{Operand}} & \multicolumn{5}{|c}{\textbf{Operand Loading and Indexing Time (C/C++)} (ms)}  \\
				$|V|$  & {PostgreSQL} & {Virtuoso} & {ArangoDB}  &  {Neo4J (Java)} & {\textbf{Nested Graphs}}  \\
				\midrule
				$10$   & 2.59$\cdot 10^1$   & 1.21$\cdot 10^1$  & 2.43$\cdot 10^2$ &  2.93$\cdot 10^3$ & \textbf{3.49}$\cdot 10^{-1}$\\
				$10^2$  & 2.80$\cdot 10^1$  & 1.22$\cdot 10^1$  & 3.91$\cdot 10^2$ & 3.10$\cdot 10^3$  & \textbf{8.87}$\cdot 10^{-1}$\\
				$10^3$  & 2.96$\cdot 10^1$  & 7.86$\cdot 10^1$  & 2.67$\cdot 10^3$ & 4.65$\cdot 10^3$ & \textbf{6.53}$\cdot 10^0$\\
				$10^4$   & \textbf{4.00}$\cdot 10^1$ & 7.43$\cdot 10^2$  & 2.34$\cdot 10^4$ & 4.23$\cdot 10^4$ & 6.90$\cdot 10^1$\\
				$10^5$   & 3.44$\cdot 10^3$ & 2.11$\cdot 10^4$  & 6.08$\cdot 10^5$ & $>$1H & \textbf{1.58}$\cdot 10^3$ \\
				$10^6$  & 1.35$\cdot 10^4$  & 1.38$\cdot 10^5$  & $>$1H & $>$1H & \textbf{1.18}$\cdot 10^4$\\
				$10^7$   & \textbf{4.77}$\cdot 10^4$ &              $>$1H  & $>$1H & $>$1H & 1.05$\cdot 10^5$\\
				%$10^8$  & -- & $>$1H & $>$1H & $>$1H & 1.08$\cdot 10^6$\\
				\bottomrule
			\end{tabular}
		}
		%\end{minipage}
		\subcaption{\textit{Operand Loading and Indexing Time}. For this dataset, our proposed operand loading time is comparable with PostgreSQL's physical model.}
		\label{tab:storeevaluation2}
	\end{adjustbox}
\end{minipage}\quad \begin{minipage}[!t]{0.45\textwidth}
	\begin{adjustbox}{max width=\textwidth}
		%\begin{minipage}[b]{\textwidth}
		\centering
		\begin{tabular}{@{}cr|rrrr|r@{}}
			\toprule
			\multicolumn{2}{c}{\textbf{Operands Size}} & \multicolumn{5}{|c}{\textbf{\textsc{Two HOp Separated Pattern} Time (C/C++)} (ms)}  \\
			$|V|$  & \#Subgraph  &  {SQL+JSON} & SPARQL & AQL  &  Cypher &{\textbf{THoSP}}  \\
			\midrule
			$10$ & 19  & 1.69$\cdot 10^0$   &  3.4$\cdot 10^1$  & 6.57$\cdot 10^{-1}$  & 2.38$\cdot 10^3$    & \textbf{2.82}$\cdot 10^{-1}$\\
			$10^2$ & 255 & 1.75$\cdot 10^0$  & 3.22$\cdot 10^2$ & 2.51$\cdot 10^0$  & 1.01$\cdot 10^4$    & \textbf{3.46}$\cdot 10^{-1}$\\
			$10^3$ & 23,119  & 4.71$\cdot 10^1$ &  1.22$\cdot 10^3$ & 8.18$\cdot 10^1$  & $>$1H & \textbf{1.39}$\cdot 10^{1}$\\
			$10^4$ & 5,411,205  & 1.53$\cdot 10^4$ &  2.77$\cdot 10^5$ & 2.08$\cdot 10^4$  & $>$1H & \textbf{2.58}$\cdot 10^3$\\
			$10^5$ & 97,079,329  & 1.20$\cdot 10^6$ & $>$1H & {\color{red}OOM$^1$}  & $>$1H & \textbf{1.97}$\cdot 10^5$ \\
			$10^6$ & 241,448,529  & $>$1H &           $>$1H & {\color{red}OOM$^1$}  & $>$1H    & \textbf{6.22}$\cdot 10^5$\\
			$10^7$ & 361,759,509  & {\color{red}OOM$^2$} &      $>$1H & {\color{red}OOM$^1$}  & $>$1H      & \textbf{7.74}$\cdot 10^5$\\
			%$10^8$ & --  & -- & $>$1H & $>$1H & $>$1H & --\\
			\bottomrule
		\end{tabular}
	}
	\subcaption{\textit{Graph Nesting Time}. In some cases, an {\color{red}Out Of Memory} error (for either primary$^{\color{red}1}$ or secondary$^{\color{red}2}$ memory) interrupts the benchmarks before the 1H timeout.}\label{tab:querytimeeval2}
	%\end{minipage}
\end{adjustbox}
\end{minipage}
\caption{Two HOp Separated Pattern. Experimental evaluations over a subset of Microsoft Academic Graph \cite{Tang08,Sinha15}. Our solution clearly outperforms the default query plan implemented over those different graph query languages and databases.}
\vspace{-2em}
\end{table*}


Our physical data model differentiates the \textit{input data  represnetation} from the \textbf{query result} (Line \ref{serialize}). 
We suppose that the latter is only used by the user to read the outcome of the nesting process as in other query languages (such as SPARQL and SQL) and does not have to
produce ``materialised views''. Therefore, the result of the graph query itself can postpone the creation of a complete ``materialised view'', which will later use the same representation of the input data by using both the id information and the application of the User Defined Functions. In particular, the former $dt$ function is used to associate both the nested vertices, $\overline{u}$ and $\overline{w}$, and the nested edge $\overline{e}$ to their grouping references. This allows to easily reconstruct the complete nested graph information by using the inverse function of $dt$, thus allowing the postponed application of the user-defined functions.
Last, the \textsc{doNest} procedure performs the association between the nested vertices (and edges) $f$ and its members within the input graph $m_i$. When the pattern requires that $m_i$ should be a member of $f$ in the final nested graph,  \textsc{doNest} stores those membership associations as pairs $\Braket{f,m_i}$ in a $Nesting$ file. By doing so, we omit the \texttt{GROUP BY} cost which affects the previously seen query languages. 

%% TODO: Please note that if in $g_E$ there is no path connecting $\alpha$ to $\gamma_E^{src}$ or $\gamma_E^{dst}$, the problem may quickly become cubic with regard to the size of the vertices, because we must create all the possible permutations where $v'$ is present alongside another element matching $\gamma_E^{src}$ or $\gamma_E^{dst}$.

% !TeX root = 00_paper_entrypoint.tex


\section{Experimental Evaluations}\label{sec:nestexpeval}
Through the following experiments we want to show that our approach outperforms the same proposed coauthorship nesting scenario  on top of graph, relational, or document oriented databases. Therefore, we consider the time required to \textit{(i)}  serialize our data structure and \textit{(ii)} evaluate the query plan. In the former we compare the loading and indexing times (the time required to store and index the data structure), and in the latter  we time the query  over the previously-loaded operand. This twofold analysis is required because, in some cases, the costly creation of several indices may lead to a better query performance.

The lack of ancillary data attached to either vertices or edges ($\forall o. \omega(o)=\{\}$) allows a better comparison of query evaluation times, which are now independent from the values' representations and more tailored to evaluate both the  access time required for traversing the loaded operator and returning the nested representation.
%The query plans are evaluated with respect to the data representation where no additional data is attached to it: by doing so we compare the algorithms used within the data structures without considering the data representation costs. 
For our evaluations we choose a bibliography graph where vertices are only represented by  vertex ids and  label, and edges are represented only by both their label, and the source and target vertices' id. Such graph was generated by the gMark generator \cite{BBCFLA17} and by random sampling the Microsoft Academic Graph \cite{Tang08,Sinha15}. In the first case, a Zipf's Law distribution with parameter $2.5$ is associated to the ingoing distribution  of each \textsc{authorOf}, while a normal distribution between $0$ and $80$ is associated to its outgoing distribution; the generator was configured to generate $8$ experiments by incrementally creating a graph with vertices with a power of $10$, that is from $10$ to $10^8$. %For both datasets, each vertex represents either an \textsc{Author} or an authored \textsc{Paper} having distinct ids. The resulting graph is represented as a list of triplets: source id, edge label (author of) and target id. 
Microsoft Academic Graph operands were obtained by incrementally aggregating complete papers' egonets, so that each graph of vertex size $10^n$ is a subgraph of the  $10^m$-operands with $n<m$.


We performed our tests over a Lenovo ThinkPad P51 with a 3.00 GHz (until 4.00 GHz) Intel Xeon processor and 64 GB of RAM at 2.400 MHz. The tests were performed over a ferromagnetic Hard Disk at 5400 RPM with an NTFS File System. We evaluate THoSP using the two pattern matching queries provided in the running example. 
%Given that the secondary memory representation is a simple extension of the one used for nested graphs, we assume that our data serialization is always outperforming with repsect to graph libraries as discussed in our previous work on graph joins, where a similar graph data structure was adopted \cite{BergamiMM17}. 
We used default configurations for  \textbf{Neo4J 3.3.0}, \textbf{PostgreSQL 9.6.6} and \textbf{ArangoDB 3.2.7}, while we changed the cache buffer configurations for \textbf{Virtuoso 7.2} (as suggested in the configuration file) for 64 GB of RAM; we also kept  default multithreaded query execution plan. PostgreSQL queries were evaluated through the \texttt{psql} client and benchmarked using both \texttt{explain analyze} and \texttt{\textbackslash timing} commands; the former   allows to analyse SQL's query plans. Virtuoso was benchmarked through the Redland RDF library using directly the \texttt{librdf\_model\_query\_execute} function; SPARQL's associated query plan was analysed via Virtuoso's \texttt{explain} satement. 
%For Neo4J and PostgreSQL we kept the same default policies as depicted for  GCEA in Subsection \ref{sec:qplan}. Therefore, we must only describe the conditions under which we performed the time execution experiments for \textbf{ArangoDB}. 
AQL queries over {ArangoDB} were evaluated directly through the \texttt{arangosh} client and benchmarked using the \texttt{getExtra()} method; statements' \texttt{explain} method was used to analyse AQL's query plans. %Given that little documentation is provided with respect to the internal plan's implementations, we referred to the \texttt{explain} procedure provided by the shell itself. 
Cypher queries were evaluated using the Java API through the \texttt{execute} method of a \texttt{GraphDatabaseService} object; the \texttt{EX\-PLAIN} statement was used to analyse the query's associated query plan. Given that only binary database connections were used (e.g., no HTTP),
all the aforementioned conditions do not degradate the query evaluations. Last, given that all databases (except from Neo4J) are implemented in C/C++ and that Neo4J provided the worst overall performances, we implemented  serialization and THoSP only in C++.
Within the relational model the graph operand's edge information were  stored in one single relational table. Similar approaches are automatically used in Virtuoso for representing RDF triple stores over its relational engine. As opposed to our implementation, all the current databases do not serialize the resulting nested graph in secondary memory.



First of all, we must discuss the  loading and indexing time (Table \ref{tab:storeevaluation} and \ref{tab:storeevaluation2}). We begin with a comparison of Virtuoso and PostgreSQL, because they are both based on a traditional relational database engine using one single table to store a graph. Virtuoso  stores an RDF graph using its default format, while in PostgreSQL the graph was stored as described in Section 2.2. Given that Virtuoso is transactionless, it performed better at loading and index data for very small data sets (from $10$ to $10^3$) while, afterwards, the triple indexing time takes over on the overall performances. On the other hand, ArangoDB has not a relational data representation, and it  serializes vertices and edges as JSON objects with several primary (vertex and edge id) and secondary (edges' source and target id) indices. Given that the only data loaded into ArangoDB are the edges' labels, all the time required to store the data is the indexing time. Neo4J's serialization proves to be inefficient, mainly because there are no constraints for data duplication and we must always check if the to-be-inserted vertex already exists. Finally, our nested graph data structure creates adjacency lists directly when serializing the data, while primary indices are not used by our input data serialization, because the adjacency lists information is sufficient to join the edges in a two hop distance scenario.

Let us now consider the graph nesting time for the synthetic dataset (Table \ref{tab:querytimeeval}): albeit no specific triplet or key are associated to the stored graph, PostgreSQL appears to be more performant than Virtuoso on graph nesting. Please also note that the Virtuoso query engine rewrites the SPARQL query into SQL and, hereby, two SQL queries were performed in both cases. Since both data were represented in a similar way in secondary memory, the completely different performance between the two databases must be attributed  to an inefficient rewriting of the SPARQL query into SQL. In particular, the nested representation using JSON array for PostgreSQL proved to be more efficient than returning a full RDF graph represented by triplets, thus arguing in favour of document stores. The PostgreSQL's efficiency is attributable to the run-time indexing time of the relational tables, that is shared with ArangoDB, where the indices are created at loading time instead: in both cases a single join operation is performed, plus some (either runtime or stored) index access time. Both PostgreSQL and ArangoDB use \texttt{GROUP BY}-s to create collections of nested values, separately for both vertices and edges. As observed in the previous paragraph, no primary index is used while performing the THoSP query, and adjacency graphs are returned using the same data structure used for graph joins: one single vertex is returned alongside the set of outgoing edges. Moreover, the nesting result is not created by using \texttt{GROUP BY}-s, but by sparsely creating an index that associates the container to its members: as a result, our query plan does not generate an additional cost for sorting and collecting all the elements because the nesting is provided during the graph traversal phase. Thus, the choice of representing the nesting information as a separate index proves to be more efficient.

As a last step, we consider the graph nesting time for the real world Microsoft Academic Graph dataset (Table \ref{tab:querytimeeval2}). First of all, we observe an increased number of generated subgraphs for the same vertex operand size; the increased number of comparisons results in overall increased evaluation time. Nevertheless, the more significant discrepancy between  THoSP's evaluation time and the opposing solutions' remarks that our solution is more competitive when multiplicity increases: in particular, our solution benefits from the path join restrictions to the papers' coauthors and the absence of \texttt{GROUP BY} costs. The real data scenario also remarks that current databases and query languages have structural deficiencies concerning the implementations' spatial complexity. Except for Virtuoso which has a primary memory configuration, all the proposed solutions suffer from Out Of Memory errors: while Neo4J exhibits such problems after the 1H timeout while computing the desired result, all the remaining opponents quickly run out of memory during the indexing time. Our solution overcomes such limitations by both using an adjacency list representation in primary memory for the final result, and by avoiding \texttt{GROUP BY} costs.
% !TeX root = 00_nesting_paper.tex
\section{Conclusions}
This paper introduces an algorithm for graph nesting jointly with graph traversal queries. Moreover, a class of graph traversal queries that can be optimized is identified, among which a graph algorithm (THoSP) is proposed. THoSP generates nesting vertices of two graph vertices appearing at most at two edge distance within the input graph. By running our algorithm over our graph data structures, we also know that our choice of providing the containment as an external index provides better performances, thus avoiding an additional data grouping phase. Moreover, this algorithm shows that the definition of a separate index for providing the algorithmic result of graph nesting is beneficial, because it allows to return the nested graph as a simple graph having additional nesting informations. This solution was possible due to the assumptions of our proposed graph data model, where  it is possible to refer any time to the elements that are going to be created later on within the computation, by  deterministically knowing  the id belonging to the to-be-created nested vertex or edge.  Therefore, this paper shows that the representation of nested graph may lead to the solution of current graph querying problems in a tractable way. Nevertheless, we believe that further studies will have to be done on the class of  graph nesting problems, thus extending our work on THoSP.
\bibliographystyle{ACM-Reference-Format}
\bibliography{mbib} 
\appendix
\section{Query Benchmarks}
\begin{lstlisting}[caption={Graph Nesting in PostgreSQL's SQL dialect. Two distinct tables are created for both vertices and edges. },language=SQL,frameround=fttt,frame=trBL,mathescape=true,label=SQLNesting]
-- Nesting Vertices
SELECT distinct T.sourceId as src, 
       array_agg(distinct T.dst) as papers 
FROM edges-$i$ as T 
GROUP BY T.sourceId;
-- Nesting Edges
SELECT distinct T.sourceId as src, 
       T1.sourceId as dst, 
       array_agg(distinct T1.targetId) as papers
FROM edges-$i$ as T, edges-$i$ as T1 
WHERE T.targetId = T1.targetId 
AND   T.sourceId <> T1.sourceId 
GROUP BY T.sourceId, T1.sourceId;
\end{lstlisting}


\begin{lstlisting}[caption={Graph Nesting in SPARQL. We use properties to associate to either vertices and edges the nesting content.},language=SPARQL,frameround=fttt,frame=trBL,tabsize=2,mathescape=true,label=SPARQLNesting]
CONSTRUCT {
 ?autha ?newedge ?authb.
 ?newedge <http://cnt.io/nesting> ?paper1.
 ?authc <http://my.grph/edge> ?paper2.
} WHERE {
 {
  GRAPH <http://my.grph/g/$i$/> {
   ?autha <http://my.grph/g/edge> ?paper1.
   ?authb <http://my.grph/g/edge> ?paper1.
  }
  FILTER(?autha != ?authb).
  BIND(URI(CONCAT("http://my.grph/g/newedge/",STRAFTER(STR(?autha),"http://my.grph/g/id/"),"-",STRAFTER(STR(?authb),"http://my.grph/g/id/"))) AS  ?newedge).
  } UNION {
   GRAPH <http://my.grph/g/$i$/> {
    ?authc <http://my.grph/g/edge> ?paper2. 
   }
   OPTIONAL {
    ?authd <http://my.grph/g/edge> ?paper2.
    FILTER (?authd != ?authc)
   }
   FILTER(!bound(?authd))
  }
}
\end{lstlisting}

\begin{lstlisting}[caption={Graph Nesting in ArangoDB using AQL. All the fields marked with an underscore represent externally indexed structures.},language=AQL,frameround=fttt,frame=trBL,tabsize=2,mathescape=true,label=AQLQueryNesting]
-- Nesting vertices
FOR b IN authorOf 
COLLECT au = b._from INTO groups = [b._to] 
RETURN {"author" : au,  "papers": groups }
-- Nesting edges
FOR x IN authorOf 
FOR y IN authorOf 
FILTER x._to==y._to && x._from!=y._from 
COLLECT src = x._from, dst = y._from 
INTO groups = [ x._to ] 
RETURN {"src": src, "dst": dst, "contain": groups}
\end{lstlisting}

\begin{lstlisting}[caption={Graph Nesting in Neo4J using Cypher as a Query Language. Please note that the nested vertices must be created before creating the nested edges.},language=Cypher,frameround=fttt,frame=trBL,mathescape=true,label=Neo4JQuery]
MATCH (a1: Author)-->(p1:Paper) 
WITH a1, collect(p1.UID) AS papers1 
CREATE  p=(:Authors {authored: papers1, id:a1.UID})

MATCH (a1: Author)-->(p:Paper)<--(a2: Author), (a1p: Authors), (a2p: Authors)
WITH a1, a2, a1p, a2p, collect(p.UID) AS common
WHERE a1.UID = a1p.id AND a2.UID = a2p.id AND a1.UID <> a2.UID
CREATE p=(a1p)-[:Papers {coauthored: common}]->(a2p)
\end{lstlisting}

\section{$dt$ function}
Given a function $f(i,j)=\sum_{k=0}^{i+j}k\;+\;j$ associating each pair in $\mathbb{N}^2$ into one single integer $\mathbb{N}$ \cite{odifreddi1992},
the $dt$ function associating one $n$-tuple in $\mathbb{N}^n$ to one single integer in $\mathbb{N}$ as follows:
	\[dt(l)=\begin{cases}
f(0,0) & |l| = 0\\
f(1,i) & l=\{i\}\\
f(|l|,r(l)) & \textup{oth.}\\  
\end{cases}\quad r(l)=\begin{cases}
f(i,j) & l = \{i,j\}\\
f(h,r(t)) & l = \{h\}\cup t\\
\end{cases}\]


%\begin{lstfloat}[!t]
%
%\end{lstfloat}

\end{document}
