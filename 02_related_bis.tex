% !TeX root = 00_nesting_paper.tex
\section{Related Works}

\subsection{Nested Graph Representations}
Statecharts\index{statechsart} \cite{statecharts} were one of the first models representing nested graphs: they were used for representing complex systems at different abstraction levels, where each node represents a  state or ``configuration'' of the system, and each edge represents a transaction between two different states on a given event. Each vertex and edge is labelled, but  they do  not come with attribute-value associations because this model was not designed for data representations. In order to represent different nesting levels, each node can contain other states and edges connecting such states. As a consequence,  there is no distinction between (simple) states and states containing other states.
This model allows both \textbf{external edges}\index{edge!external} and \textbf{internal edges}: we say that edge $e$ is \textit{external} if its source (or target) is contained by the target (or source) but neither of them contains $e$; the edge is called \textit{internal}\index{edge!internal} when the containing vertex (either its source or target) also contains the edge. Besides of state representation purposes, this model was been even used for both modelling the evolution of \textit{pathophysiological} states and to describe the subsequent treatments to which the patient must undergo, where each treatment could be furtherly subdivided in smaller consequential steps to be followed \cite{NestedGlaucoma}.

This model was also adopted as a basis for the \textbf{hypernode} data model \cite{Poulovassilis1994}: even if hypernodes are subsequent to statecharts, they are less expressive than the former ones, because they do not label the edges and they only allow edges between vertices which are contained within the same vertex: the model neither represents external edges nor internal ones. As previously stated for statecharts, even this model does not allow to fully represent a property graph, since the attribute-value association must be necessarily expressed as a relation between two different vertices \cite{Poulovassilis1994}. Last, the fact that the vertex containments cannot overlap make such nested model affected from the same \textit{data replication} representation problem described for semistructured and nested data. A first extension of the hypernode model towards data representation is represented by CoGITaNT \cite{GenestS98}, where any type of edge (thus including internal and external ones) are included and data is firstly contained inside a node. Nested graphs are also supported by GraphML \cite{graphml} and GXL \cite{GXL}.

 Two different approaches have been used to extend current graph data model for supporting nesting operations:
the first ones try to overcome to the basic graph data structure limitations by simply extending the query language, while the other ones try to extend the data structures that are used for both input and intermediate computations. 


Among the first type of approaches, the one outlined in \cite{Etcheverry2012} propose to define a RDF\index{RDF} vocabulary over which the OLAP\index{OLAP} cube can be defined in RDF\index{graph!RDF}. On top of this ``structured'' RDF graph, an algorithm generates the SPARQL query that will allow to perform either the roll-up or the drill-down operation. This later approach implies that each possible computation has to be always recomputed on top of the row data like for classical ROLAP systems: as a consequence, this MOLAP approach does not benefit from the specific RDF representation, that is not able to represent different aggregation levels and to store intermediate computations.

The last type of approaches have been recently widely investigated, and seem to be more promising with respect to optimization techniques: in \cite{Tian20085,ChenYZHY08,Qu2011} graph data structures are associated with external graph indices, thus allowing to connect one graph  to its broader one with respect to the roll-up query. As a consequence, these solutions do not allow to freely expand any aggregated component at a time, but they can only backtrack the aggregation to a previous known state. %Some further details are going to be provided on Chapter \vref{cha:nesting}, where such operator will be implemented on a specific algorithm.
%As it will be showed in Chapter \ref{cha:graphsdef}, in order to meet such goals the nesting indices are going to be directly embedded within the definition of the nested (graph) data model, thus allowing to extend all the aforementioned approaches.


\subsection{Databases and Query Languages}\label{subsec:pathsumm}
We want now to discuss how current query languages can express graph nesting  within their data model of choice. In particular, we must select query languages that either support collections or nested representations allowing to express the same query presented in our running example. The associated proposed listings can be seen in the appendix.

For these reasons, we select PostgreSQL's dialect which, by extending the SQL-3 syntax with JSON data supports, allows to create arrays as a result of a \texttt{GROUP BY} query via \texttt{array\_agg}; therefore, instead of using function aggregating a collection into one single summarized value, we can list all the elements that have been aggregated. In particular, in PostgreSQL's dialect, we  implement our graph by  storing the triples defining an edge as the following relation:
\begin{center}
 \texttt{Edge}(\textit{edgeId},\;\textit{sourceId},\;\textit{edgeLabel},\;\textit{targetId})
\end{center}
Hereby, by grouping the edges by \textit{sourceId} and collecting all the target's ids we obtain a representation of nested vertices. Similarly, if we join the \texttt{Edge} relation with itself and group the join result by two distinct \textit{sourceId} and return the list of all the \textsc{Paper}s that they have in common, we can return the list of all the \textsc{Paper}s that they have coauthored. As showed in Listing \ref{SQLNesting}, the overall graph nesting cannot be created in one single SQL query, because by  the SQL language definition we cannot distinctively group the same dataset in different ways, but we must visit the same data twice and perform two distinct aggregations. 


All the other query languages are going to be affected by the same problem, SPARQL included (Listing \ref{SPARQLNesting}): despite the fact that this query language may represent the graph nesting query as a single statement, even in this case the \texttt{UNION} clause implies a separate visit for the two graph patterns. In particular, the first pattern allows to traverse those graph patterns matching the coauthorship statement in Figure \ref{fig:pathPat} so that they can be nested within the created \textsc{CoAuthorship} edge, while the second part returns all the associations of the \textsc{Paper} that have been authored by one single \textsc{Author}. In particular, the \texttt{OPTIONAL\dots FILTER(!bound(\dots))} syntax is adopted instead of \texttt{FILTER NOT EXISTS}, because the latter is only supported in SPARQL1.1, which is not supported by the current version of \textit{librdf} used to query Virtuoso in our benchmarks. In this case, the edge nesting is performed via the association of different \texttt{<http://cnt.io/nesting>} properties departing from the \texttt{?newedge}  \textsc{CoAuthorship} between two coauthors. Consequently, even in this case the graph visits two times the same graph patterns.


We also consider the AQL query language supported by ArangoDB, because ArangoDB is a NoSQL database relying on a document-oriented storage, which is hereby prone to both represent and return nested content. An example of how such graph nesting query can be carried out in AQL is presented in Listing \ref{AQLQueryNesting}: in this scenario we assume that we've previously loaded our graph data with the default format, where both vertices and edges are fully stored, and where the former are indexed by id, while the latter are also indexed by source and target vertex id. In particular, we can state that this algorithm provides the exact same result as the one produced by the SQL query, except that JSON documents are returned instead of relational tables containing JSON arrays.
%
%Let us consider the following query that will be express in our different languages:
%``\textit{Total number of orders handled by each employee. Only list employees that handled more than 100 orders}''
%
%At this point we immediately notice that NautiLOD and G could not express the summarization since those languages
%are graph traversal languages for pattern matching: consequently they could only select paths and subgraphs but
%they do not aggregate (summarize) nodes. In this case a bag of both employees and the number of the orders
%is returned. Since the result is a bag of values, we cannot obtain a graph as an output, and hence we cannot establish some new edges through the employee and the aggregated value of the sales.
%
%\begin{lstlisting}[caption={Summarization query in Gremlin},language=gremlin,frameround=fttt,frame=trBL]
%graph = TinkerGraph.open()
%graph.io(IoCore.gryo()).readGraph('/path/to/graph')
%g = graph.traversal()
%
%g.V().hasLabel("Employee").match(
%      __.as("emp").in("SalesEmployee").hasLabel("Sales").count()
%                                      .as("ordersByEmployee"),
%      __.as("ordersByEmployee").is(gt(100))
%).select("emp", "ordersByEmployee")
%\end{lstlisting}
%\medskip
%
%The SPARQL query returns a table with two attributes, where the first is the Employee ID and the
%second element is the number of its handled orders. In this case the output is expressed as a table.
%
%\begin{lstlisting}[caption={Summarization query in SPARQL: Table},language=sparql,frameround=fttt,frame=trBL]
%PREFIX ex:<http://example.it/Relations#>
%
%SELECT	 ?emp, (COUNT (?sales) AS ?ordersByEmployee)
%WHERE    {
%          ?sales a                 ex:Sales;
%                 ex:SalesEmployee  ?emp.
%          ?emp   a                 ex:Employee.    
%         }
%GROUP BY ?emp
%HAVING   COUNT(?orderNo) > 100
%\end{lstlisting}
%
%With the \texttt{\textbf{CONSTRUCT}} clause we could return the previous result inside an RDF Graph.
%\begin{lstlisting}[caption={Summarization query in SPARQL: Graph},language=sparql,frameround=fttt,frame=trBL]
%PREFIX ex:<http://example.it/Relations#>
%
%CONSTRUCT { ?ordersByEmployee ex:SalesEmployee ?emp. }
%WHERE {{
%	SELECT	 ?emp, (COUNT (?sales) AS ?ordersByEmployee)
%	WHERE    {
%	          ?sales a                 ex:Sales;
%	                 ex:SalesEmployee  ?emp.
%	          ?emp   a                 ex:Employee.    
%	}
%	GROUP BY ?emp
%	HAVING   COUNT(?orderNo) > 100
%}}
%\end{lstlisting}
%\medskip
%
%In Cypher we could formulate a similar query with a tabular result as follows:
%\begin{lstlisting}[caption={Summarization query in Cypher: Table},language=cypher,frameround=fttt,frame=trBL]
%MATCH  (sales:Sales)-[:SalesEmployee]->(empl:Employee)
%WITH   empl AS emp, COUNT(sales) AS ordersByEmployee
%WHERE  ordersByEmployee > 100
%RETURN empl, ordersByEmployee
%\end{lstlisting}
%
%With this language we could even return a new graph, where
%the whole information of the employee is returned and where the films are aggregated.
%\begin{lstlisting}[caption={Summarization query in Cypher: Graph},language=cypher,frameround=fttt,frame=trBL]
%MATCH  (sales:Sales)-[:SalesEmployee]->(empl:Employee)
%WITH   empl AS emp, COUNT(sales) AS ordersByEmployee
%WHERE  ordersByEmployee > 100
%CREATE p=
%(:Sales {count: ordersByEmployee})-[:SalesEmployee]->(empl:Employee)
%RETURN p
%\end{lstlisting}
%\medskip 
%
%Let us now focus on the BiQL language: firstly we cannot create new vertices that aggregate the results without
%updating the original database because the \texttt{\textbf{CREATE}} semantic has this precise meaning, secondly the
%\texttt{\textbf{CREATE}} clause does not allow to create multiple objects within the same query: this means that we 
%cannot create new edges while creating new vertices. By the way we could return all the employees and store the
%number of the sales inside each node.
%\begin{lstlisting}[caption={Summarization query in Cypher: Graph},language=biql,frameround=fttt,frame=trBL]
%SELECT <empl>{empl.*, ordersByEmployee: count(salesEdge)}
%FROM   Employee empl <- SalesEmployee salesEdge 
%WHERE  count(salesEdge) > 100
%\end{lstlisting}
%

Last, Listing \ref{Neo4JQuery} provides an example of Neo4J allowing to nest property graphs: even in this case the property graph model does not directly  nest the graphs inside one element. Similar to the previous approaches, we can group by all the graphs returned by the graph pattern by selecting the vertices of interest, and nesting the to-be-grouped remaining objects inside a collection. In particular, we can first match the vertex summarization pattern in Figure \ref{fig:vertexPat} and group by \textsc{Author}, and nest the collection of authored \textsc{Paper}s within the to-be-created nested vertex; similarly, we can first match the path summarization pattern presented in Figure \ref{fig:pathPat} by source and destination \textsc{Author}, and then create an edge between the previously created nested vertices by collecting all the coauthored \textsc{Paper}s appearing in the original graph. Moreover, the Neo4J property graph data model implies that we cannot create an edge if the vertices are not previously known beforehand and, therefore, we always must join the nested vertices with the original matched ones in order to reconstruct the original information and perform the actual matching operation. As it will be observed within the benchmarks, the   solution of not separating the elements' ids from their data quickly leads to an intractable solution. 