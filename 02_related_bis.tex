% !TeX root = 00_nesting_paper.tex
\section{Related Works}

\subsection{Nested Graph Representations}
\textbf{Statecharts} \cite{statecharts} represent one of the first applications of nested graphs to complex systems modelling. This choice allowed the representation of multiple abstraction levels at the same time: each node represents a  state or ``configuration'' of the system, and each edge represents a transaction between two different states on a given event. In order to represent different nesting levels, each node can contain other states and edges connecting such states. As a consequence,  there is no distinction between (simple) states and states containing other states. Given that this model was not designed for data representations, vertices and edges are labelled but cannot contain any property-value association. 
This model allows both \textit{external edges} and \textit{internal edges}: we say that edge  $e$ is \textit{external} if its source (or target) is contained by the target (or source) but neither of them contains $e$; the edge is called \textit{internal} when the containing vertex (either its source or target) also contains the edge. Besides of state representation purposes, this model has also been  used for both modelling the evolution of \textit{pathophysiological} states and to describe the clinical treatments to which the patient must undergo. This model also allowed to subdivide each treatment  into smaller and consequential procedures \cite{NestedGlaucoma}.

Statecharts were also adopted as a basis for the subsequent \textbf{hypernode} data model \cite{Poulovassilis1994}. Unlike statecharts, hypernodes allows neither edge labelling nor external and internal edges. As previously stated for statecharts, even this model does not allow to fully represent a property graph, since the attribute-value association must be necessarily expressed as a relation between two different vertices.  A first extension of the hypernode model towards data representation is represented by CoGITaNT \cite{GenestS98}, where any type of edge (thus including internal and external ones) are included and data is firstly contained inside a node. Nested graphs are also supported by GraphML \cite{graphml} and GXL \cite{GXL}.

 Two different approaches have been used to extend current graph data model for supporting nesting operations:
the first ones try to overcome to the basic graph data structure limitations by simply extending the query language, while the other ones try to extend the data structures that are used for both input and intermediate computations. Among the first type of approaches, \cite{Etcheverry2012} proposes the definition of a RDF  vocabulary over which the OLAP  cube can be defined in RDF. On top of this ``structured'' RDF graph, an algorithm generates the SPARQL query that will allow to perform either the roll-up or the drill-down operation. This implies that each possible computation over the data view has to be always recomputed on top of the raw data like for classical ROLAP systems, thus thwarting the benefits of updating the intermediate query result. On the other hand, the last type of approaches have been recently widely investigated, and seem to be more promising with respect to optimization techniques. Graph data structures are now \cite{Tian20085,ChenYZHY08,Qu2011} associated with external graph indices, thus allowing to connect one graph to its broader one with respect to the roll-up query. As a consequence, these solutions do not allow to freely expand any aggregated component at a time, but they can only backtrack the aggregation to a previous known state. %Some further details are going to be provided on Chapter \vref{cha:nesting}, where such operator will be implemented on a specific algorithm.
%As it will be showed in Chapter \ref{cha:graphsdef}, in order to meet such goals the nesting indices are going to be directly embedded within the definition of the nested (graph) data model, thus allowing to extend all the aforementioned approaches.


\subsection{Databases and Query Languages}\label{subsec:pathsumm}
We want now to discuss how current query languages can express graph nesting  within their data model of choice. In particular, we must select query languages that either support collections or nested representations allowing to express the same query presented in our running example. 

For these reasons, we select PostgreSQL's dialect which, by extending the SQL-3 syntax with JSON data supports, the \texttt{array\_agg} aggregation function allows to group the result-set into arrays.  We  represent our graph by  storing the triples defining an edge as the following relation:
\begin{center}
 \texttt{Edge}(\textit{edgeId},\;\textit{sourceId},\;\textit{edgeLabel},\;\textit{targetId})
\end{center}
By grouping the edges by \textit{sourceId} and collecting all the target's ids we obtain a representation of nested vertices. Similarly, if we join the \texttt{Edge} relation with itself and group the join result by two distinct \textit{sourceId} and return the list of all the \textsc{Paper}s that they have in common, we can return the list of all the \textsc{Paper}s that they have coauthored. The overall graph nesting cannot be created in one single SQL query, because we cannot distinctively group the same dataset in different ways, but we must perform two distinct aggregations (see Listing \ref{SQLNesting} in Appendix). 


Despite the fact that SPARQL  may represent the graph nesting query as a single statement, a \texttt{UNION} clause implies a separate visit for the two graph patterns. The first pattern presented in Listing \ref{SPARQLNesting} (see Appendix) allows to traverse those  patterns matching the coauthorship statement in Figure \ref{fig:pathPat} so that they can be nested within the created \textsc{CoAuthorship} edge, while the second part returns all the associations of the \textsc{Paper} that have been authored by one single \textsc{Author}. In particular, the \texttt{OPTIONAL\dots FILTER(!bound(\dots))} syntax is adopted instead of \texttt{FILTER NOT EXISTS}, because the latter is not supported by \textit{librdf} which is used to query Virtuoso in our benchmarks. In this case, the edge nesting is performed via the association of different \texttt{<http://cnt.io/nesting>} properties departing from one single  \textsc{coAuthorship} edge (\texttt{?newedge}). 

Given that ArangoDB is a NoSQL database relying on a document-oriented storage, which is hereby prone to both represent and return nested content, we also consider its query language, AQL. An example of how such graph nesting query can be carried out in AQL is presented in Listing \ref{AQLQueryNesting}: in this scenario we assume that we've previously loaded our graph data with the default ArangoDB format, where  vertices are indexed by id while edges are also indexed by source and target vertex id. Even though AQL returns JSON documents instead of relational table, We can state that the resulting query plan  provides the exact same result as the one produced by the SQL query.
%
%Let us consider the following query that will be express in our different languages:
%``\textit{Total number of orders handled by each employee. Only list employees that handled more than 100 orders}''
%
%At this point we immediately notice that NautiLOD and G could not express the summarization since those languages
%are graph traversal languages for pattern matching: consequently they could only select paths and subgraphs but
%they do not aggregate (summarize) nodes. In this case a bag of both employees and the number of the orders
%is returned. Since the result is a bag of values, we cannot obtain a graph as an output, and hence we cannot establish some new edges through the employee and the aggregated value of the sales.
%
%\begin{lstlisting}[caption={Summarization query in Gremlin},language=gremlin,frameround=fttt,frame=trBL]
%graph = TinkerGraph.open()
%graph.io(IoCore.gryo()).readGraph('/path/to/graph')
%g = graph.traversal()
%
%g.V().hasLabel("Employee").match(
%      __.as("emp").in("SalesEmployee").hasLabel("Sales").count()
%                                      .as("ordersByEmployee"),
%      __.as("ordersByEmployee").is(gt(100))
%).select("emp", "ordersByEmployee")
%\end{lstlisting}
%\medskip
%
%The SPARQL query returns a table with two attributes, where the first is the Employee ID and the
%second element is the number of its handled orders. In this case the output is expressed as a table.
%
%\begin{lstlisting}[caption={Summarization query in SPARQL: Table},language=sparql,frameround=fttt,frame=trBL]
%PREFIX ex:<http://example.it/Relations#>
%
%SELECT	 ?emp, (COUNT (?sales) AS ?ordersByEmployee)
%WHERE    {
%          ?sales a                 ex:Sales;
%                 ex:SalesEmployee  ?emp.
%          ?emp   a                 ex:Employee.    
%         }
%GROUP BY ?emp
%HAVING   COUNT(?orderNo) > 100
%\end{lstlisting}
%
%With the \texttt{\textbf{CONSTRUCT}} clause we could return the previous result inside an RDF Graph.
%\begin{lstlisting}[caption={Summarization query in SPARQL: Graph},language=sparql,frameround=fttt,frame=trBL]
%PREFIX ex:<http://example.it/Relations#>
%
%CONSTRUCT { ?ordersByEmployee ex:SalesEmployee ?emp. }
%WHERE {{
%	SELECT	 ?emp, (COUNT (?sales) AS ?ordersByEmployee)
%	WHERE    {
%	          ?sales a                 ex:Sales;
%	                 ex:SalesEmployee  ?emp.
%	          ?emp   a                 ex:Employee.    
%	}
%	GROUP BY ?emp
%	HAVING   COUNT(?orderNo) > 100
%}}
%\end{lstlisting}
%\medskip
%
%In Cypher we could formulate a similar query with a tabular result as follows:
%\begin{lstlisting}[caption={Summarization query in Cypher: Table},language=cypher,frameround=fttt,frame=trBL]
%MATCH  (sales:Sales)-[:SalesEmployee]->(empl:Employee)
%WITH   empl AS emp, COUNT(sales) AS ordersByEmployee
%WHERE  ordersByEmployee > 100
%RETURN empl, ordersByEmployee
%\end{lstlisting}
%
%With this language we could even return a new graph, where
%the whole information of the employee is returned and where the films are aggregated.
%\begin{lstlisting}[caption={Summarization query in Cypher: Graph},language=cypher,frameround=fttt,frame=trBL]
%MATCH  (sales:Sales)-[:SalesEmployee]->(empl:Employee)
%WITH   empl AS emp, COUNT(sales) AS ordersByEmployee
%WHERE  ordersByEmployee > 100
%CREATE p=
%(:Sales {count: ordersByEmployee})-[:SalesEmployee]->(empl:Employee)
%RETURN p
%\end{lstlisting}
%\medskip 
%
%Let us now focus on the BiQL language: firstly we cannot create new vertices that aggregate the results without
%updating the original database because the \texttt{\textbf{CREATE}} semantic has this precise meaning, secondly the
%\texttt{\textbf{CREATE}} clause does not allow to create multiple objects within the same query: this means that we 
%cannot create new edges while creating new vertices. By the way we could return all the employees and store the
%number of the sales inside each node.
%\begin{lstlisting}[caption={Summarization query in Cypher: Graph},language=biql,frameround=fttt,frame=trBL]
%SELECT <empl>{empl.*, ordersByEmployee: count(salesEdge)}
%FROM   Employee empl <- SalesEmployee salesEdge 
%WHERE  count(salesEdge) > 100
%\end{lstlisting}
%

Last, Listing \ref{Neo4JQuery} provides an example of Neo4J: even if Neo4J's property graph model does not directly nest the graphs inside one vertex or edges, we can associate to each vertex or edge the vertices' and edges' ids that it may contain. This solution can be achieved by  first matching the vertex summarization pattern in Figure \ref{fig:vertexPat} and then perform an \textsc{Author} group by (\texttt{with}), and then nest the collection of authored \textsc{Paper}s via \texttt{collect}. Last, we match the path summarization pattern presented in Figure \ref{fig:pathPat} and group it by source and destination \textsc{Author}, then we create the \textsc{coAuthorship} edge containing the co-authored \textsc{Paper}'s id. As it will be observed within the benchmarks, the solution of not separating the elements' ids from their data quickly leads to an intractable solution. 