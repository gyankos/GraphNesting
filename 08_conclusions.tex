% !TeX root = 00_nesting_paper.tex
\section{Conclusions}
We presented an algorithm for graph nesting jointly with graph traversal queries. Moreover, we identified optimizations for a class of graph traversal queries combining both vertex summarization and edge summarization patterns. Our proposal, THoSP, focuses on the use case where vertex patterns are separated within the edge pattern by a two hop distance. By running our algorithm over the proposed physical nested graph data model, we showed that the choice of separating the adjacency graph representation from the nesting index is beneficial, because it allows to return the nested graph as a simple graph having extra nesting informations. This solution was possible due to the assumptions of our logical nested graph database model, where  it is possible to refer any time to the elements that are going to be created later on within the computation, by  using the vertex and edge id information.  Therefore, this paper shows that the representation of nested graph may lead to the solution of current graph querying problems in a tractable way. Nevertheless, we believe that further studies will have to be done to extend the class of optimizable graph nesting patterns, thus extending our work on THoSP.
