% !TeX root = 00_nesting_paper.tex
\section{Conclusions}
This paper introduces an algorithm for graph nesting jointly with graph traversal queries. Moreover, a class of graph traversal queries that can be optimized is identified, among which a graph algorithm (THoSP) is proposed whenever the nesting vertices are generated by two graph vertices appearing at most at two steps within the original graph operand. By comparing the execution of our algorithm with our graph data structures, we also know that our choice of providing the containment as an external index provides better performances, thus avoiding an additional data grouping phase. Moreover, this algorithm shows that the definition of a separate index for providing the algorithmic result of graph nesting is beneficial, because it allows to return the nested graph as a simple graph having additional nesting informations. This solution was possible due to the assumptions of our proposed graph data model, where it is showed that it is possible to refer any time to the elements that are going to be created later on within the computation, by simply deterministically knowing which the id belonging to the element that is going to be created.  Therefore, this paper proves that the representation of nested graph may lead to the solution of current graph querying problems in a tractable way. Nevertheless, we believe that further studies will have to be done on the class of  GROQ problems, thus extending our work on THoSP.