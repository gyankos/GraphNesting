% !TeX root = 00_nesting_paper.tex
\section{Conclusions}
We presented an algorithm for graph nesting jointly with graph traversal queries. Moreover, we identified optimizations for a class of graph traversal queries and proposed THoSP, and efficient algorithm for graph nesting. THoSP generates nesting vertices of two graph vertices appearing at most at two edge distance within the input graph \textbf{[REFORMULATE SENTENCE: I don't get the meaning at all...]}. By running our algorithm over our graph data structures, we also know that our choice of providing the containment as an external index provides better performances, thus avoiding an additional data grouping phase. Moreover, this algorithm shows that the definition of a separate index for providing the algorithmic result of graph nesting is beneficial, because it allows to return the nested graph as a simple graph having additional nesting informations. This solution was possible due to the assumptions of our proposed graph data model, where  it is possible to refer any time to the elements that are going to be created later on within the computation, by  deterministically knowing  the id belonging to the to-be-created nested vertex or edge.  Therefore, this paper shows that the representation of nested graph may lead to the solution of current graph querying problems in a tractable way. Nevertheless, we believe that further studies will have to be done on the class of  graph nesting problems, thus extending our work on THoSP.
