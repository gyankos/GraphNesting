% !TeX root = 00_paper_entrypoint.tex

\section{Conclusions}
To the best of our knowledge, this paper proposed for the first time an algorihtm (THoSP) which adds structural aggregation to an input graph. The final outcome of this process is a nested graph, which contains vertices and edges that may contain subgraphs of the original input graph. Such result is obtained by jointly visiting two graph patterns, the vertex and the pattern summarization, respectively leading to the creation of nested vertices and nested edges. The reason why such algorithm outperforms equivalent implementations over graph, relational and document based competitors is twofold: first, while their query plans force one graph visit per pattern, our solution allows to visit such graph only once; last, by detaching the graph representation from the membership information in the \textit{query result} we can avoid the cost of performing an additional \texttt{GROUP BY} operation. This paper also provides a nested graph data model, allowing the definition of a generic graph nesting operator.
We believe that further studies will have to be done to extend the class of optimizable graph nesting patterns, and that a more general data model may provide a cleaner definition of the proposed graph nesting operator. 
