% !TeX root = 00_paper_entrypoint.tex

\section{Query Benchmarks}
\begin{lstlisting}[caption={Graph Nesting in PostgreSQL's SQL dialect. Two distinct tables are created for both vertices and edges. },language=SQL,frameround=fttt,frame=trBL,mathescape=true,label=SQLNesting]
-- Nesting Vertices
SELECT distinct T.sourceId as src, 
       array_agg(distinct T.dst) as papers 
FROM edges-$i$ as T 
GROUP BY T.sourceId;
-- Nesting Edges
SELECT distinct T.sourceId as src, 
       T1.sourceId as dst, 
       array_agg(distinct T1.targetId) as papers
FROM edges-$i$ as T, edges-$i$ as T1 
WHERE T.targetId = T1.targetId 
AND   T.sourceId <> T1.sourceId 
GROUP BY T.sourceId, T1.sourceId;
\end{lstlisting}


\begin{lstlisting}[caption={Graph Nesting in SPARQL. We use properties to associate to either vertices and edges the nesting content.},language=SPARQL,frameround=fttt,frame=trBL,tabsize=2,mathescape=true,label=SPARQLNesting]
CONSTRUCT {
 ?autha ?newedge ?authb.
 ?newedge <http://cnt.io/nesting> ?paper1.
 ?authc <http://my.grph/edge> ?paper2.
} WHERE {
 {
  GRAPH <http://my.grph/g/$i$/> {
   ?autha <http://my.grph/g/edge> ?paper1.
   ?authb <http://my.grph/g/edge> ?paper1.
  }
  FILTER(?autha != ?authb).
  BIND(URI(CONCAT("http://my.grph/g/newedge/",STRAFTER(STR(?autha),"http://my.grph/g/id/"),"-",STRAFTER(STR(?authb),"http://my.grph/g/id/"))) AS  ?newedge).
  } UNION {
   GRAPH <http://my.grph/g/$i$/> {
    ?authc <http://my.grph/g/edge> ?paper2. 
   }
   OPTIONAL {
    ?authd <http://my.grph/g/edge> ?paper2.
    FILTER (?authd != ?authc)
   }
   FILTER(!bound(?authd))
  }
}
\end{lstlisting}

\begin{lstlisting}[caption={Graph Nesting in ArangoDB using AQL. All the fields marked with an underscore represent externally indexed structures.},language=AQL,frameround=fttt,frame=trBL,tabsize=2,mathescape=true,label=AQLQueryNesting]
-- Nesting vertices
FOR b IN authorOf 
COLLECT au = b._from INTO groups = [b._to] 
RETURN {"author" : au,  "papers": groups }
-- Nesting edges
FOR x IN authorOf 
FOR y IN authorOf 
FILTER x._to==y._to && x._from!=y._from 
COLLECT src = x._from, dst = y._from 
INTO groups = [ x._to ] 
RETURN {"src": src, "dst": dst, "contain": groups}
\end{lstlisting}

\begin{lstlisting}[caption={Graph Nesting in Neo4J using Cypher as a Query Language. Please note that the nested vertices must be created before creating the nested edges.},language=Cypher,frameround=fttt,frame=trBL,mathescape=true,label=Neo4JQuery]
MATCH (a1: Author)-->(p1:Paper) 
WITH a1, collect(p1.UID) AS papers1 
CREATE  p=(:Authors {authored: papers1, id:a1.UID})

MATCH (a1: Author)-->(p:Paper)<--(a2: Author), (a1p: Authors), (a2p: Authors)
WITH a1, a2, a1p, a2p, collect(p.UID) AS common
WHERE a1.UID = a1p.id AND a2.UID = a2p.id AND a1.UID <> a2.UID
CREATE p=(a1p)-[:Papers {coauthored: common}]->(a2p)
\end{lstlisting}

\section{$dt$ function}
Given a function $f(i,j)=\sum_{k=0}^{i+j}k\;+\;j$ associating each pair in $\mathbb{N}^2$ into one single integer $\mathbb{N}$ \cite{odifreddi1992},
the $dt$ function associating one $n$-tuple in $\mathbb{N}^n$ to one single integer in $\mathbb{N}$ as follows:
	\[dt(l)=\begin{cases}
f(0,0) & |l| = 0\\
f(1,i) & l=\{i\}\\
f(|l|,r(l)) & \textup{oth.}\\  
\end{cases}\quad r(l)=\begin{cases}
f(i,j) & l = \{i,j\}\\
f(h,r(t)) & l = \{h\}\cup t\\
\end{cases}\]


%\begin{lstfloat}[!t]
%
%\end{lstfloat}
