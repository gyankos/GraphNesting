\usetheme[numbering=fraction]{metropolis}           % Use metropolis theme

\usefonttheme{professionalfonts}
\usepackage[T1]{fontenc}
\usepackage[type1]{libertine}
\renewcommand{\ttdefault}{cmtt}

\usepackage{booktabs}
\usepackage{braket}
\usepackage{multirow}




%% Do not cound backup slides
\usepackage{appendixnumberbeamer}

% English Format, please!
\usepackage[UKenglish]{babel}% http://ctan.org/pkg/babel
\usepackage[UKenglish]{isodate}% \year=2018
\day=10
\month=6

\title{THoSP: an Algorithm for Nesting Property Graphs}
\date{\today}
\author[G. Bergami et al.]{Giacomo Bergami \textsuperscript{1} \and Andr\'e Petermann \textsuperscript{2} \and Danilo Montesi \textsuperscript{1}\\
1\textsuperscript{st} Joint  GRADES-NDA International Workshop, 2018}
\institute[Bologna \& Leipzig Uni.]{ Universit\`a di Bologna\textsuperscript{1},  Universit\"at Leipzig\textsuperscript{2}}

%% Time Arrow
\usepackage{smartdiagram}

%%joins
\def\ojoin{\setbox0=\hbox{$\bowtie$}%
	\rule[0.2ex]{.27em}{.4pt}\llap{\rule[1ex]{.27em}{.4pt}}}
\def\leftouterjoin{\mathbin{\ojoin\mkern-2mu\Join}}
\def\rightouterjoin{\mathbin{\Join\mkern-3.4mu\ojoin}}
\def\fullouterjoin{\mathbin{\ojoin\mkern-7mu\Join\mkern-7mu\ojoin}}

%%Boxes
\usepackage{tcolorbox}

%%Animations
\usepackage{eulervm}
\usepackage[osf,sc]{mathpazo}
\usepackage{inconsolata}
\usetikzlibrary{matrix,fit,calc}
\definecolor{webgreen}{rgb}{0,.5,0}
\usepackage{tikz}
\usetikzlibrary{tikzmark,positioning}
\newcommand*{\yellowemph}[1]{%
	\tikz[baseline=(X.base)] \node[rectangle, fill=yellow, rounded corners, inner sep=0.3mm] (X) {#1};%
}
\newcommand*{\greenemph}[1]{%
	\tikz[baseline=(X.base)] \node[rectangle, fill=green, rounded corners, inner sep=0.3mm] (X) {#1};%
}
\tikzstyle{none}=[inner sep=0pt]
\tikzstyle{new}=[circle,fill=white,draw=black]
\tikzstyle{smallnode}=[circle, inner sep=0mm, outer sep=0mm, minimum size=2mm, draw=black, fill=black]
\tikzstyle{unselected}=[circle,draw=black,dashed]
\usetikzlibrary{decorations.pathreplacing}
\usepackage{preview}
\usetikzlibrary{automata,arrows,backgrounds,positioning,fit,calc}
\PreviewEnvironment{tikzpicture}

\tikzstyle{every node}=[font=\small]
\tikzset{>=latex}
\tikzstyle{tuple} = [draw,rounded corners,align=center,label={above right:\{#1\}}]
\tikzstyle{user}[red] = [draw=#1,rounded corners,align=center,label={above:\{User\}}]
\tikzstyle{proj}[blue] = [draw=#1,rounded corners,align=center,label={above:\{Paper\}}]
\tikzstyle{user2}[red] = [draw=#1,rounded corners,align=center,label={above:\{User\}}]
\tikzstyle{user3}[red] = [draw=#1,rounded corners,align=center,fill=blue!30,label={above:\{User\}}]
\tikzstyle{proj2}[blue] = [draw=#1,rounded corners,align=center,label={above:\{Paper\}}]
\tikzstyle{proj3}[blue] = [draw=#1,rounded corners,align=center,fill=blue!30,label={above:\{Paper\}}]
\tikzstyle{result}[blue] = [draw=#1,rounded corners,align=center,label={above:\{User,Paper\}}]
\tikzstyle{result2}[blue] = [draw=#1,rounded corners,align=center,fill=blue!30,label={above:\{User,Paper\}}]

\newcommand{\nbd}[2]{\draw[red,very thick] ($(#1)-(0.15,-0.28)$) rectangle ($(#2)+(0.3,-0.1)$);}
\newcommand{\nbl}[2]{\draw[red,very thick] ($(#1)-(0.15,-0.28)$) rectangle ($(#2)+(1.5,-0.15)$);}
\newcommand{\nbm}[2]{\draw[red,very thick] ($(#1)-(0.25,-0.28)$) rectangle ($(#2)+(0.85,-0.15)$);}
\newcommand{\nbmc}[1]{\draw[blue,very thick,fill=cyan, fill opacity=0.2] ($(#1)-(0.15,-0.28)$) rectangle ($(#1)+(0.85,-0.15)$);}
\newcommand{\nblc}[1]{\draw[blue,very thick,fill=cyan, fill opacity=0.2] ($(#1)-(0.15,-0.28)$) rectangle ($(#1)+(1.5,-0.15)$);}
\newcommand{\propval}[2]{\textsc{#1}=\textit{#2}}
\newcommand{\user}[2]{\propval{Name}{#1}}
\newcommand{\paper}[2]{\propval{Title}{#1}\\ \propval{Conference}{#2}}
\newcommand{\proj}[2]{\propval{Title}{#2}\\ \propval{1Author}{#1}}
\newcommand{\result}[2]{\propval{Title}{#2}\\ \propval{1Author}{#1}\\ \propval{Name}{#1}}

\usetikzlibrary{chains}
\usetikzlibrary{calc}
\usetikzlibrary{shapes}
\usetikzlibrary{shapes.multipart}
\usetikzlibrary{arrows}
\usetikzlibrary{fit}

%% Colors
\newcommand\kk[1]{\textcolor{RoyalBlue}{\text{\textup{\textbf{\texttt{#1}}}}}}
\newcommand\cc[1]{\textcolor{Sepia}{\text{\textup{\textbf{\texttt{#1}}}}}}
\newcommand{\remark}[1]{\textbf{\alert{#1}}}

%% Definition boxes
\tcbset{
	colback=red!5!white,
	boxrule=0.1pt,
	colframe=red!75!black,
	fonttitle=\bfseries
}

%% Objectifing elements
\usepackage{environ}
%%% Defining a slide, which is also a section
\NewEnviron{lucido}[1][]{%
	\subsection{#1}
	\begin{frame}{\insertsection\;--\;#1}
	\BODY
	\end{frame}
}

%% Defining a slide consisting of multiple subparts
\NewEnviron{multilucido}[1][]{%
	\subsection{#1}
	\BODY
	\setcounter{contalucido}{0}
}

\newcounter{contalucido}
\NewEnviron{sottolucido}[1][\thecontalucido]{%
	\refstepcounter{contalucido}
	\begin{frame}{\insertsection~--~\insertsubsection~(#1)}
		\BODY
	\end{frame}
}

%%Code
\usepackage{listings}
\lstdefinelanguage{pseudi}{
	 basicstyle=\tiny,
	 showstringspaces=false,
	 literate={:}{{{\color{RoyalBlue}:}}}1,
	keywords={for, each, in},
	ndkeywords={Map},
	sensitive=false,
	comment=[l]{//},
}
\NewEnviron{sottofragile}[1][\thecontalucido]{%
	\refstepcounter{contalucido}
	\begin{frame}[fragile]{\insertsection~--~\insertsubsection~(#1)}
		\BODY
	\end{frame}
}




\definecolor{webgreen}{rgb}{0,.5,0}
\newcommand{\mstr}[1]{\textup{\color{webgreen}``#1''}}

\newcommand{\nested}{\ensuremath{G_o}}
\newcommand{\ngraph}{{g}}
\DeclareMathOperator{\dom}{dom}
\DeclareMathOperator{\cod}{cod}
\newcommand{\ONTA}{\mstr{onta}}
\newcommand{\RELA}{\mstr{rela}}
\newcommand{\SRC}{\mstr{src}}
\newcommand{\DST}{\mstr{dst}}
\newcommand{\VS}{\mathcal{V}}
\newcommand{\ES}{\mathcal{E}}
\newcommand{\Keys}{K}
\newcommand{\Val}{V}
\newcommand{\valF}{F}
\newcommand{\nestF}{\nu}
\newcommand{\prov}{\epsilon}

\titlegraphic{\begin{tikzpicture}\node[inner sep=0pt,opacity=0.1] (russell) at (4,2) {\includegraphics[width=1.3\textwidth]{seal.png}};\end{tikzpicture}}
\addtobeamertemplate{title page}{}{}
