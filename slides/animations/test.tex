\documentclass[hyperref={pdfpagelabels=false}]{beamer}
\usepackage{lmodern}

\usepackage[utf8]{inputenc} % this is needed for german umlauts
\usepackage[ngerman]{babel} % this is needed for german umlauts
\usepackage[T1]{fontenc}    % this is needed for correct output of umlauts in pdf

\usepackage{verbatim}
\usepackage{tikz}
\usetikzlibrary{arrows,shapes}

% Define some styles for graphs
\tikzstyle{vertex}=[circle,fill=black!25,minimum size=20pt,inner sep=0pt]
\tikzstyle{selected vertex} = [vertex, fill=red!24]
\tikzstyle{blue vertex} = [vertex, fill=blue!24]
\tikzstyle{edge} = [draw,thick,-]
\tikzstyle{weight} = [font=\small]
\tikzstyle{selected edge} = [draw,line width=5pt,-,red!50]
\tikzstyle{ignored edge} = [draw,line width=5pt,-,black!20]

% see http://deic.uab.es/~iblanes/beamer_gallery/index_by_theme.html
\usetheme{Frankfurt}
\usefonttheme{professionalfonts}

% disables bottom navigation bar
\beamertemplatenavigationsymbolsempty

% http://tex.stackexchange.com/questions/23727/converting-beamer-slides-to-animated-images
\setbeamertemplate{navigation symbols}{}%

\begin{document}
\pgfdeclarelayer{background}
\pgfsetlayers{background,main}

\begin{frame}
    \begin{figure}
        \begin{tikzpicture}[->,scale=1.8, auto,swap]
            % Draw the vertices. First you define a list.
            \foreach \pos/\name in {{(0,0)/a}, {(0,2)/b}, {(1,2)/c},
                                    {(1,0)/d}, {(2,1)/e}, {(3,1)/f}, 
                                    {(4,2)/g}, {(5,2)/h}, {(4,0)/i},
                                    {(5,0)/j}}
                \node[vertex] (\name) at \pos {$\name$};

            % Connect vertices with edges and draw weightstikz add shades to nodes in animations
            \foreach \source/ \dest /\pos in {a/b/, b/c/, c/d/, d/a/,
                                        c/e/bend left, d/e/, e/c/,
                                        e/f/, f/g/, f/i/,
                                        g/f/bend right, i/f/bend left,
                                        g/h/, h/j/, j/i/, i/g/}
                \path (\source) edge [\pos] node {} (\dest);

            % Start animating the edge selection. 
            % For convenience we use a background layer to 
            % highlight edges. This way we don't have to worry about 
            % the highlighting covering weight labels. 
            \begin{pgfonlayer}{background}
                \foreach \source / \dest / \fr / \pos in {d/a/1/,
                        a/b/2/, b/c/3/, c/d/4/, d/e/5/, e/c/6/, 
                        c/e/7/bend left, e/f/8/, f/g/9/,
                        g/f/10/bend right, f/i/11/, i/g/12/, g/h/13/, tikz add shades to nodes in animations
                        h/j/14/, j/i/15/, i/f/16/bend left}
                    \path<\fr->[selected edge] (\source.center) edge 
                                        [\pos] node {} (\dest.center);
            \end{pgfonlayer}
        \end{tikzpicture}
    \end{figure}
    Lorem ipsum dolor sit amet, consetetur sadipscing elitr, 
    sed diam nonumy eirmod tempor invidunt ut labore et dolore
    magna aliquyam erat, sed diam voluptua. At vero eos et 
    accusam et.
\end{frame}
\end{document}
