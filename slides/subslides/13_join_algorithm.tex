%%%%%%%%%%%%%%%%%%%%%%
%%%%%%%%%%%%%%%%%%%%%%
%%%%%%%%%%%%%%%%%%%%%%
\begin{lucido}[Graph Conjunctive Equals Algorithm]
The GCEA Algorithm consists of three phases:
\begin{enumerate}[<+->]
	\item Generating the hashing function $h$ from the vertex $\theta$ predicate.
	\item \alert{Partitioning}: associating to each vertex id an hash value.
	\item \alert{Serialization}: storing the partitioned operands in secondary memory.
	\item \alert{(Partition) Sorted Join}: actual graph join from \texttt{mmap}-ped serialized operands.
\end{enumerate}
\end{lucido}

%%%%%%%%%%%%%%%%%%%%%%%%%%%%%
%%%%%%%%%%%%%%%%%%%%%%%%%%%%%
%%%%%%%%%%%%%%%%%%%%%%%%%%%%%
\begin{multilucido}[Serialization]
	\begin{sottolucido}
		\begin{center}
			\begin{columns}[T]
				\begin{column}{.7\textwidth}
					\centering
					\resizebox{.8\textwidth}{!}{\begin{tikzpicture}
\clip (0.3,-2.5) rectangle (7.7, 1);
\node[user] (a1) at (2,.25) {\user{Alice}{$t_1$}};
\node[user] (a3) at (2,-2.25) {\user{Carl}{$t_1$}};
\node[user] (a2) at (6.3,.25) {\user{Bob}{$t_2$}};
\node[user] (a4) at (6.3,-2.25) {\user{Dan}{$t_2$}};

\path[->,orange] (a1.north east) edge node [anchor=south,sloped] {\{Follows\}} (a2.north west);
\path[->,orange] (a2.south east) edge  node [anchor=south,sloped] {\{Follows\}\qquad} (a4.north east);
\path[<-,orange] (a3.north west) edge node [anchor=north,sloped] {\{Follows\}\qquad} (a1.south west);
\path[<-,orange] (a3.east) edge node [anchor=south,sloped] {\{Follows\}\qquad} (a4);
\end{tikzpicture}
}
					\begin{itemize}
	\footnotesize
	\item  {The left array represent the hash primary index.}
	\item The central array contains the vertex values:
	\begin{itemize}
		\item Headers are omitted for brevity
		\item Triples $(o,h,e)$ identify the outgoing vertex id, its hash value
			and the corresponding outgoing edge.
	\end{itemize}
	\item The right array represent the vertex id secondary index.
\end{itemize}
				\end{column}
				\begin{column}{.3\textwidth}
					\begin{tikzpicture}
					\begin{scope}[start chain=going left,node distance=0pt]
					\def\width{9ex}
					\tikzstyle{frame}=[
					font=\tiny,
					draw,
					rectangle split,
					rectangle split part align={center, left},
					rectangle split empty part height=0ex,
					rectangle split empty part depth=0ex,
					rectangle split empty part width=0ex,
					]
					
					%%MAIN
					\node (main) [frame,
					rectangle split parts=12, %% teal
					rectangle split part fill={yellow,teal!30,white,white,yellow,teal!30,white,yellow,teal!30,yellow,teal!30,white} 
					] at (-1,0) {%
						\nodepart{one} Alice
						\nodepart{two} 2
						\nodepart{three} (3,2,1)
						\nodepart{four} (2,3,2)
						\nodepart{five} Bob
						\nodepart{six} 1
						\nodepart{seven} (4,4,3)
						\nodepart{eight} Carl
						\nodepart{nine} 0
						\nodepart{ten} Dan
						\nodepart{eleven} 1
						\nodepart{twelve} (2,3,4)
					};
					
					%% SORT
					\node (sort) [draw,font=\tiny,
					rectangle split,
					rectangle split parts=12,
					text centered,
					align=center,
					rectangle split part fill={Fuchsia!30,Fuchsia!30,brown!30,Fuchsia!30,Fuchsia!30,brown!30,Fuchsia!30,Fuchsia!30,brown!30,Fuchsia!30,Fuchsia!30,brown!30} 
					] at (0.5,0) {%
						\nodepart{one} \kk{1}
						\nodepart{two} \cc{1}
						\nodepart{three} 
						
						\nodepart{four} \kk{2}
						\nodepart{five} \cc{3}
						\nodepart{six} 
						
						\nodepart{seven} \kk{3}
						\nodepart{eight} \cc{2}
						\nodepart{nine} 
						
						\nodepart{ten} \kk{4}
						\nodepart{eleven} \cc{4}
						\nodepart{twelve} 
						
					};
					
					\node (qsort1) [draw,font=\tiny,
					rectangle split,
					rectangle split parts=8,
					text centered,
					align=center,
					rectangle split part fill={Fuchsia!30,brown!30,Fuchsia!30,brown!30,Fuchsia!30,brown!30,Fuchsia!30,brown!30} 
					] at (-2.3,0) {%
						\nodepart{one}   \cc{\textbf 1}
						\nodepart{two}  
						\nodepart{three}  \cc{\textbf 2}
						\nodepart{four}  
						\nodepart{five} \cc{\textbf 3}
						\nodepart{six} 
						\nodepart{seven} \cc{\textbf 4}
						\nodepart{eight} 
					};
					\draw[-latex] (qsort1.two east) -- (main.one west);
					\draw[-latex] (qsort1.four east) -- (main.five west);
					\draw[-latex] (qsort1.six east) -- (main.eight west);
					\draw[-latex] (qsort1.eight east) -- (main.ten west);
					
					
					\draw[-latex] (sort.three west) -- (main.one east);
					\draw[-latex] (sort.six west) -- (main.eight east);
					\draw[-latex] (sort.nine west) -- (main.five east);
					\draw[-latex] (sort.twelve west) -- (main.ten east);
					\end{scope}
					\end{tikzpicture}
				\end{column}
			\end{columns}
		\end{center}
	\end{sottolucido}

	\begin{sottolucido}
		\begin{center}
			\begin{columns}[T]
				\begin{column}{.7\textwidth}
					\centering
					\resizebox{.8\textwidth}{!}{\begin{tikzpicture}
\clip (0.3,-3) rectangle (10.1, 1.7);
\node[proj] (a1) at (2,0) {\proj{Alice}{Graphs}};
\node[proj] (a1b) at (5,0) {\proj{Alice}{Join}};
\node[proj] (a3) at (5,-2) {\proj{Carl}{Projection}};
\node[proj] (a2) at (8.3,0) {\proj{Bob}{OWL}};
\node[proj] (a4) at (8.3,-2) {\proj{Dan}{$\mu$-calc}};

\path[->,purple] (a1.north east) edge [bend left=40] node [anchor=south,sloped] {\{Cites\}} (a2.north west);
\path[->,purple] (a2.south west) edge  node [anchor=south,sloped] {\{Cites\}\qquad} (a3.north east);
\path[<-,purple] (a3.north west) edge node [anchor=north,sloped] {\{Cites\}\qquad} (a1b.south west);
\path[->,purple] (a3.south) edge [bend right=30] node [anchor=south,sloped] {\{Cites\}} (a4.south);
\end{tikzpicture}}
					\begin{itemize}
	\footnotesize
	\item  {The left array represent the hash primary index.}
	\item The central array contains the vertex values:
	\begin{itemize}
		\item Headers are omitted for brevity
		\item Triples $(o,h,e)$ identify the outgoing vertex id, its hash value
			and the corresponding outgoing edge.
	\end{itemize}
	\item The right array represent the vertex id secondary index.
\end{itemize}
				\end{column}
				\begin{column}{.3\textwidth}
					\begin{tikzpicture}
				\begin{scope}[start chain=going left,node distance=0pt]
				\def\width{9ex}
				\tikzstyle{frame}=[
				font=\tiny,
				draw,
				rectangle split,
				rectangle split part align={center, left},
				rectangle split empty part height=0ex,
				rectangle split empty part depth=0ex,
				rectangle split empty part width=0ex,
				]
				
				%%MAIN
				\node (main) [frame,
				rectangle split parts=14, %% teal
				rectangle split part fill={yellow,teal!30,white,yellow,teal!30,white,yellow,teal!30,white,yellow,teal!30,white,yellow,teal!30} 
				] at (-1,0) {%
					\nodepart{one} Graphs, Alice
					\nodepart{two} 1
					\nodepart{three} (4,2,1)
					\nodepart{four} Join, Alice
					\nodepart{five} 1
					\nodepart{six} (3,3,2)
					\nodepart{seven} OWL, Bob
					\nodepart{eight} 1
					\nodepart{nine} (3,3,3)
					\nodepart{ten} Project, Carl
					\nodepart{eleven} 1
					\nodepart{twelve} (5,4,4)
					\nodepart{thirteen} $\mu$-calc, Dan
					\nodepart{fourteen} 0
				};
				
				%% SORT
				\node (sort) [draw,font=\tiny,
				rectangle split,
				rectangle split parts=15,
				text centered,
				align=center,
				rectangle split part fill={Fuchsia!30,Fuchsia!30,brown!30,Fuchsia!30,Fuchsia!30,brown!30,Fuchsia!30,Fuchsia!30,brown!30,Fuchsia!30,Fuchsia!30,brown!30,Fuchsia!30,Fuchsia!30,brown!30} 
				] at (0.5,0) {%
					\nodepart{one} \kk{1}
					\nodepart{two} \cc{1}
					\nodepart{three} 
					
					\nodepart{four} \kk{2}
					\nodepart{five} \cc{1}
					\nodepart{six} 
					
					\nodepart{seven} \kk{3}
					\nodepart{eight} \cc{3}
					\nodepart{nine} 
					
					\nodepart{ten} \kk{4}
					\nodepart{eleven} \cc{2}
					\nodepart{twelve} 
					
					\nodepart{thirteen} \kk{5}
					\nodepart{fourteen} \cc{4}
					\nodepart{fifteen} 
					
				};
				
				\node (qsort1) [draw,font=\tiny,
				rectangle split,
				rectangle split parts=8,
				text centered,
				align=center,
				rectangle split part fill={Fuchsia!30,brown!30,Fuchsia!30,brown!30,Fuchsia!30,brown!30,Fuchsia!30,brown!30} 
				] at (-2.3,0) {%
					\nodepart{one}   \cc{\textbf 1}
					\nodepart{two}  
					\nodepart{three}  \cc{\textbf 2}
					\nodepart{four}  
					\nodepart{five} \cc{\textbf 3}
					\nodepart{six} 
					\nodepart{seven} \cc{\textbf 4}
					\nodepart{eight} 
				};
				\draw[-latex] (qsort1.two east) -- (main.one west);
				\draw[-latex] (qsort1.four east) -- (main.seven west);
				\draw[-latex] (qsort1.six east) -- (main.ten west);
				\draw[-latex] (qsort1.eight east) -- (main.thirteen west);
				
				
				\draw[-latex] (sort.three west) -- (main.one east);
				\draw[-latex] (sort.six west) -- (main.four east);
				\draw[-latex] (sort.nine west) -- (main.ten east);
				\draw[-latex] (sort.twelve west) -- (main.seven east);
				\draw[-latex] (sort.fifteen west) -- (main.thirteen east);
				\end{scope}
				\end{tikzpicture}
				\end{column}
			\end{columns}
		\end{center}
	\end{sottolucido}
\end{multilucido}


%%%%%%%%%%%%%%%%%%%%%%%%%%%%%%%%%%%%%%%
%%%%%%%%%%%%%%%%%%%%%%%%%%%%%%%%%%%%%%%
%%%%%%%%%%%%%%%%%%%%%%%%%%%%%%%%%%%%%%%
\subsubsection{(Partition) Sorted Join} 
\begin{frame}{GCEA Algorithm: (Partition) Sorted Join}
	\begin{columns}[T]
		\begin{column}{.4\textwidth}
			
			\begin{itemize}[<only@+>]
				\item \alert{Intersection of the primary index through a linear scan}
				\item Select the nodes with the first common hash \cc{1}.
				\item Select the nodes with the first common hash \cc{1}. In particular, node \kk{1} from the left operand matches with node
				\kk{1} from the right operand. Moreover, outgoing nodes \kk{3} left and \kk{4} right also match.
				\item Select the nodes with the first common hash \cc{1}. In particular, node \kk{1} from the left operand matches with node
				\kk{2} from the right operand. Moreover, outgoing nodes \kk{2} left and \kk{3} right also match.
				\item Select the nodes with the first common hash \cc{2}. In particular, node \kk{3} from the left operand matches with node
				\kk{4} from the right operand. No edges and outgoing vertices match.
				\item On the next steps, \kk{2} left matches with \kk{3} right, but \kk{2} has no outgoing edges. Similarly, \kk{4} left matches with \kk{5} right, but \kk{5} has no outgoing edges.
			\end{itemize}
			\onslide<1->
		\end{column}
		%%%%%%%%%%%%%%%%%%%%%%%%%%%%%%%
		%%%%%%%%%%%%%%%%%%%%%%%%%%%%%%%
		%%%%%%%%%%%%%%%%%%%%%%%%%%%%%%%
		%%%%%%%%%%%%%%%%%%%%%%%%%%%%%%%
		\begin{column}{.3\textwidth}
			\begin{tikzpicture}
			\begin{scope}[start chain=going left,node distance=0pt]
			\def\width{9ex}
			\tikzstyle{frame}=[
			font=\tiny,
			draw,
			rectangle split,
			rectangle split part align={center, left},
			rectangle split empty part height=0ex,
			rectangle split empty part depth=0ex,
			rectangle split empty part width=0ex,
			]
			
			%%MAIN
			\node (main) [frame,
			rectangle split parts=12, %% teal
			rectangle split part fill={yellow,teal!30,white,white,yellow,teal!30,white,yellow,teal!30,yellow,teal!30,white} 
			] at (-1,0) {%
				\nodepart{one} Alice
				\nodepart{two} 2
				\nodepart{three} (3,2,1)
				\nodepart{four} (2,3,2)
				\nodepart{five} Bob
				\nodepart{six} 1
				\nodepart{seven} (4,4,3)
				\nodepart{eight} Carl
				\nodepart{nine} 0
				\nodepart{ten} Dan
				\nodepart{eleven} 1
				\nodepart{twelve} (2,3,4)
			};
			
			%% SORT
			\node (sort) [draw,font=\tiny,
			rectangle split,
			rectangle split parts=12,
			text centered,
			align=center,
			rectangle split part fill={Fuchsia!30,Fuchsia!30,brown!30,Fuchsia!30,Fuchsia!30,brown!30,Fuchsia!30,Fuchsia!30,brown!30,Fuchsia!30,Fuchsia!30,brown!30} 
			] at (0.5,0) {%
				\nodepart{one} \kk{1}
				\nodepart{two} \cc{1}
				\nodepart{three} 
				
				\nodepart{four} \kk{2}
				\nodepart{five} \cc{3}
				\nodepart{six} 
				
				\nodepart{seven} \kk{3}
				\nodepart{eight} \cc{2}
				\nodepart{nine} 
				
				\nodepart{ten} \kk{4}
				\nodepart{eleven} \cc{4}
				\nodepart{twelve} 
				
			};
			
			\node (qsort1) [draw,font=\tiny,
			rectangle split,
			rectangle split parts=8,
			text centered,
			align=center,
			rectangle split part fill={Fuchsia!30,brown!30,Fuchsia!30,brown!30,Fuchsia!30,brown!30,Fuchsia!30,brown!30} 
			] at (-2.3,0) {%
				\nodepart{one}   \cc{\textbf 1}
				\nodepart{two}  
				\nodepart{three}  \cc{\textbf 2}
				\nodepart{four}  
				\nodepart{five} \cc{\textbf 3}
				\nodepart{six} 
				\nodepart{seven} \cc{\textbf 4}
				\nodepart{eight} 
			};
			\draw[-latex] (qsort1.two east) -- (main.one west);
			\draw[-latex] (qsort1.four east) -- (main.five west);
			\draw[-latex] (qsort1.six east) -- (main.eight west);
			\draw[-latex] (qsort1.eight east) -- (main.ten west);
			
			
			\draw[-latex] (sort.three west) -- (main.one east);
			\draw[-latex] (sort.six west) -- (main.eight east);
			\draw[-latex] (sort.nine west) -- (main.five east);
			\draw[-latex] (sort.twelve west) -- (main.ten east);
			\onslide<2-4>
			\nbd{qsort1.one}{qsort1.two}
			\nbm{main.one}{main.four}
			\onslide<3>
			\nbmc{main.three}
			\onslide<4>
			\nbmc{main.four}
			\onslide<5>
			\nbd{qsort1.three}{qsort1.four}
			\nbm{main.five}{main.seven}
			\onslide<1->
			\end{scope}
			\end{tikzpicture}
			
		\end{column}
		%%%%%%%%%%%%%%%%%%%%%%%%%%%%%%%
		%%%%%%%%%%%%%%%%%%%%%%%%%%%%%%%
		%%%%%%%%%%%%%%%%%%%%%%%%%%%%%%%
		%%%%%%%%%%%%%%%%%%%%%%%%%%%%%%%
		\begin{column}{.3\textwidth}
			\begin{tikzpicture}
			\begin{scope}[start chain=going left,node distance=0pt]
			\def\width{9ex}
			\tikzstyle{frame}=[
			font=\tiny,
			draw,
			rectangle split,
			rectangle split part align={center, left},
			rectangle split empty part height=0ex,
			rectangle split empty part depth=0ex,
			rectangle split empty part width=0ex,
			]
			
			%%MAIN
			\node (main) [frame,
			rectangle split parts=14, %% teal
			rectangle split part fill={yellow,teal!30,white,yellow,teal!30,white,yellow,teal!30,white,yellow,teal!30,white,yellow,teal!30} 
			] at (-1,0) {%
				\nodepart{one} Graphs, Alice
				\nodepart{two} 1
				\nodepart{three} (4,2,1)
				\nodepart{four} Join, Alice
				\nodepart{five} 1
				\nodepart{six} (3,3,2)
				\nodepart{seven} OWL, Bob
				\nodepart{eight} 1
				\nodepart{nine} (3,3,3)
				\nodepart{ten} Project, Carl
				\nodepart{eleven} 1
				\nodepart{twelve} (5,4,4)
				\nodepart{thirteen} $\mu$-calc, Dan
				\nodepart{fourteen} 0
			};
			
			%% SORT
			\node (sort) [draw,font=\tiny,
			rectangle split,
			rectangle split parts=15,
			text centered,
			align=center,
			rectangle split part fill={Fuchsia!30,Fuchsia!30,brown!30,Fuchsia!30,Fuchsia!30,brown!30,Fuchsia!30,Fuchsia!30,brown!30,Fuchsia!30,Fuchsia!30,brown!30,Fuchsia!30,Fuchsia!30,brown!30} 
			] at (0.5,0) {%
				\nodepart{one} \kk{1}
				\nodepart{two} \cc{1}
				\nodepart{three} 
				
				\nodepart{four} \kk{2}
				\nodepart{five} \cc{1}
				\nodepart{six} 
				
				\nodepart{seven} \kk{3}
				\nodepart{eight} \cc{3}
				\nodepart{nine} 
				
				\nodepart{ten} \kk{4}
				\nodepart{eleven} \cc{2}
				\nodepart{twelve} 
				
				\nodepart{thirteen} \kk{5}
				\nodepart{fourteen} \cc{4}
				\nodepart{fifteen} 
				
			};
			
			\node (qsort1) [draw,font=\tiny,
			rectangle split,
			rectangle split parts=8,
			text centered,
			align=center,
			rectangle split part fill={Fuchsia!30,brown!30,Fuchsia!30,brown!30,Fuchsia!30,brown!30,Fuchsia!30,brown!30} 
			] at (-2.3,0) {%
				\nodepart{one}   \cc{\textbf 1}
				\nodepart{two}  
				\nodepart{three}  \cc{\textbf 2}
				\nodepart{four}  
				\nodepart{five} \cc{\textbf 3}
				\nodepart{six} 
				\nodepart{seven} \cc{\textbf 4}
				\nodepart{eight} 
			};
			\draw[-latex] (qsort1.two east) -- (main.one west);
			\draw[-latex] (qsort1.four east) -- (main.seven west);
			\draw[-latex] (qsort1.six east) -- (main.ten west);
			\draw[-latex] (qsort1.eight east) -- (main.thirteen west);
			
			
			\draw[-latex] (sort.three west) -- (main.one east);
			\draw[-latex] (sort.six west) -- (main.four east);
			\draw[-latex] (sort.nine west) -- (main.ten east);
			\draw[-latex] (sort.twelve west) -- (main.seven east);
			\draw[-latex] (sort.fifteen west) -- (main.thirteen east);
			
			%% First Hash Code
			\onslide<2-4>
			\nbd{qsort1.one}{qsort1.two}
			\nbl{main.one}{main.six}
			\onslide<3>
			\nblc{main.three}
			\onslide<4>
			\nblc{main.six}
			\onslide<5>
			\nbd{qsort1.three}{qsort1.four}
			\nbl{main.seven}{main.nine}
			\onslide<1->
			\end{scope}
			\end{tikzpicture}
		\end{column}
	\end{columns}
	
	\begin{tikzpicture}
	\begin{scope}[start chain,node distance=0cm,yshift=70,xshift=-23,minimum height=1.5em]
	\onslide<3->
	\node[label=above:{\textit{V}},draw,fill=yellow!30,on chain] (v1) {1$\oplus$1};
	\node[label=above:{\textit{E}},draw,fill=teal!30,on chain] (hx) {1$\oplus$1};
	\onslide<4->
	\node[label=above:{\textit{V}},draw,fill=yellow!30,on chain] (v1) {1$\oplus$2};
	\node[label=above:{\textit{E}},draw,fill=teal!30,on chain] (hx) {2$\oplus$2};
	\onslide<5->
	\node[label=above:{\textit{V}},draw,fill=yellow!30,on chain] (v1) {3$\oplus$4};
	\onslide<6->
	\node[label=above:{\textit{V}},draw,fill=yellow!30,on chain] (v1) {2$\oplus$3};
	\onslide<7->
	\node[label=above:{\textit{V}},draw,fill=yellow!30,on chain] (v1) {4$\oplus$5};
	\end{scope};
	
	\onslide<1->
	\end{tikzpicture}
	
\end{frame}


%%%%%%%%%%%%%%%%%%%%%%%%%%%%%%%
%%%%%%%%%%%%%%%%%%%%%%%%%%%%%%%
%%%%%%%%%%%%%%%%%%%%%%%%%%%%%%%
%%%%%%%%%%%%%%%%%%%%%%%%%%%%%%%
%%%%%%%%%%%%%%%%%%%%%%%%%%%%%%%
%%%%%%%%%%%%%%%%%%%%%%%%%%%%%%%
%%%%%%%%%%%%%%%%%%%%%%%%%%%%%%%
%%%%%%%%%%%%%%%%%%%%%%%%%%%%%%%
%%%%%%%%%%%%%%%%%%%%%%%%%%%%%%%
%%%%%%%%%%%%%%%%%%%%%%%%%%%%%%%
%%%%%%%%%%%%%%%%%%%%%%%%%%%%%%%
%%%%%%%%%%%%%%%%%%%%%%%%%%%%%%%

