%%%%%%%%%%%%%%%%%%%%%%
%%%%%%%%%%%%%%%%%%%%%%
%%%%%%%%%%%%%%%%%%%%%%
\begin{lucido}[Example (Conjunctive EquiJoin)]
	This is an example of a possible graph join query:


\begin{tcolorbox}[title=A Graph Join Query]
		Consider an on-line service such as \yellowemph{ResearchGate}
		where researchers can \textup{follow} each others' work, and a \yellowemph{citation graph}.
		\yellowemph{Return the paper graph} where \remark{a paper cites another one iff. the \greenemph{first
				author  of the first paper \textit{follows} the first author of} \greenemph{the second}}.
	\end{tcolorbox}
	
	\begin{itemize}
		\item \yellowemph{Binary operator returning just one graph}.
		\item \greenemph{Vertex conditions are ($\theta$-)join conditions}.
		\item \remark{A way to combine the edges is determined}.
	\end{itemize}
\end{lucido}

%%%%%%%%%%%%%%%%%%%%%%
%%%%%%%%%%%%%%%%%%%%%%
%%%%%%%%%%%%%%%%%%%%%%
\begin{lucido}[Operands in Property Graph Data Model]
	\begin{tikzpicture}
\clip (0.3,-2.5) rectangle (7.7, 1);
\node[user] (a1) at (2,.25) {\user{Alice}{$t_1$}};
\node[user] (a3) at (2,-2.25) {\user{Carl}{$t_1$}};
\node[user] (a2) at (6.3,.25) {\user{Bob}{$t_2$}};
\node[user] (a4) at (6.3,-2.25) {\user{Dan}{$t_2$}};

\path[->,orange] (a1.north east) edge node [anchor=south,sloped] {\{Follows\}} (a2.north west);
\path[->,orange] (a2.south east) edge  node [anchor=south,sloped] {\{Follows\}\qquad} (a4.north east);
\path[<-,orange] (a3.north west) edge node [anchor=north,sloped] {\{Follows\}\qquad} (a1.south west);
\path[<-,orange] (a3.east) edge node [anchor=south,sloped] {\{Follows\}\qquad} (a4);
\end{tikzpicture}

	\begin{tikzpicture}
\clip (0.3,-3) rectangle (10.1, 1.7);
\node[proj] (a1) at (2,0) {\proj{Alice}{Graphs}};
\node[proj] (a1b) at (5,0) {\proj{Alice}{Join}};
\node[proj] (a3) at (5,-2) {\proj{Carl}{Projection}};
\node[proj] (a2) at (8.3,0) {\proj{Bob}{OWL}};
\node[proj] (a4) at (8.3,-2) {\proj{Dan}{$\mu$-calc}};

\path[->,purple] (a1.north east) edge [bend left=40] node [anchor=south,sloped] {\{Cites\}} (a2.north west);
\path[->,purple] (a2.south west) edge  node [anchor=south,sloped] {\{Cites\}\qquad} (a3.north east);
\path[<-,purple] (a3.north west) edge node [anchor=north,sloped] {\{Cites\}\qquad} (a1b.south west);
\path[->,purple] (a3.south) edge [bend right=30] node [anchor=south,sloped] {\{Cites\}} (a4.south);
\end{tikzpicture}
\end{lucido}

%%%%%%%%%%%%%%%%%%%%%%
%%%%%%%%%%%%%%%%%%%%%%
%%%%%%%%%%%%%%%%%%%%%%
\begin{lucido}[Conjunctive EquiJoin]
	\begin{tikzpicture}
	%% Operand Left
	\begin{scope}[scale=0.6,yshift=5.5cm,xshift=0cm,every node/.style={transform shape}]
	\node[user2] (a1) at (2,.25) {\user{Alice}{$t_1$}};
	\node[user2] (a3) at (2,-2.25) {\user{Carl}{$t_1$}};
	\node[user2] (a2) at (6.3,.25) {\user{Bob}{$t_2$}};
	\node[user2] (a4) at (6.3,-2.25) {\user{Dan}{$t_2$}};
	
	\path[->,orange] (a1.north east) edge node [anchor=south,sloped] {\{Follows\}} (a2.north west);
	\path[->,orange] (a2.south east) edge  node [anchor=south,sloped] {\{Follows\}\qquad} (a4.north east);
	\path[<-,orange] (a3.north west) edge node [anchor=north,sloped] {\{Follows\}\qquad} (a1.south west);
	\path[<-,orange] (a3.east) edge node [anchor=south,sloped] {\{Follows\}\qquad} (a4);
	
	\onslide<2>
	\node[user3] (a1) at (2,.25) {\user{Alice}{$t_1$}};
	\onslide<3>
	\node[user3] (a2) at (6.3,.25) {\user{Bob}{$t_2$}};
	\onslide<4>
	\node[user3] (a3) at (2,-2.25) {\user{Carl}{$t_1$}};
	\onslide<5>
	\node[user3] (a4) at (6.3,-2.25) {\user{Dan}{$t_2$}};
	\onslide<6>
	\path[->,orange,line width=1mm] (a1.north east) edge node [anchor=south,sloped] {\{Follows\}} (a2.north west);
	\path[<-,orange,line width=1mm] (a3.north west) edge node [anchor=north,sloped] {\{Follows\}\qquad} (a1.south west);
	\onslide<1->
	\end{scope}
	
	%% Operand Right
	\begin{scope}[scale=0.6,yshift=-4cm,xshift=0cm,every node/.style={transform shape}]
	\node[proj2] (a1) at (2,0) {\proj{Alice}{Graphs}};
	\node[proj2] (a1b) at (5,0) {\proj{Alice}{Join}};
	\node[proj2] (a3) at (5,-2) {\proj{Carl}{Projection}};
	\node[proj2] (a2) at (8.3,0) {\proj{Bob}{OWL}};
	\node[proj2] (a4) at (8.3,-2) {\proj{Dan}{$\mu$-calc}};
	
	\path[->,purple] (a1.north east) edge [bend left=40] node [anchor=south,sloped] {\{Cites\}} (a2.north west);
	\path[->,purple] (a2.south west) edge  node [anchor=south,sloped] {\{Cites\}\qquad} (a3.north east);
	\path[<-,purple] (a3.north west) edge node [anchor=north,sloped] {\{Cites\}\qquad} (a1b.south west);
	\path[->,purple] (a3.south) edge [bend right=30] node [anchor=south,sloped] {\{Cites\}} (a4.south);
	
	\onslide<2>
	\node[proj3] (a1) at (2,0) {\proj{Alice}{Graphs}};
	\node[proj3] (a1b) at (5,0) {\proj{Alice}{Join}};
	\onslide<3>
	\node[proj3] (a2) at (8.3,0) {\proj{Bob}{OWL}};
	\onslide<4>
	\node[proj3] (a3) at (5,-2) {\proj{Carl}{Projection}};
	\onslide<5>	
	\node[proj3] (a4) at (8.3,-2) {\proj{Dan}{$\mu$-calc}};
	\onslide<6>
	\path[->,purple,line width=1mm] (a1.north east) edge [bend left=40] node [anchor=south,sloped] {\{Cites\}} (a2.north west);
	\path[<-,purple,line width=1mm] (a3.north west) edge node [anchor=north,sloped] {\{Cites\}\qquad} (a1b.south west);
	\onslide<1->
	\end{scope}
	
	%% Result
	\begin{scope}[scale=0.6,yshift=2.8cm,xshift=12cm,every node/.style={transform shape}]		
	
	\onslide<2>
	\node[result2] (a1) at (1,0) {\result{Alice}{Graphs}};
	\node[result2] (a2) at (5.5,0) {\result{Alice}{Join}};
	\onslide<3>
	\node[result2] (a3) at (1,-7.4) {\result{Bob}{OWL}};
	\onslide<4>
	\node[result2] (a4) at (1,-3.7) {\result{Carl}{Projection}};
	\onslide<5>
	\node[result2] (a5) at (5.5,-7.4) {\result{Dan}{$\mu$-calc}};
	\onslide<3->
	\node[result] (a1) at (1,1) {\result{Alice}{Graphs}};
	\node[result] (a2) at (5.5,1) {\result{Alice}{Join}};
	\onslide<4->
	\node[result] (a3) at (1,-7.4) {\result{Bob}{OWL}};
	\onslide<5->
	\node[result] (a4) at (1,-3.7) {\result{Carl}{Projection}};
	\onslide<6->
	\node[result] (a5) at (5.5,-7.4) {\result{Dan}{$\mu$-calc}};
	\path[->,purple!60!orange,very thick,dotted] (a1.north west) edge [bend right=33] node [anchor=south,sloped] {\{Follows,Cites\}} (a3.north west);
	\path[->,purple!60!orange,very thick,dotted] (a2) edge  node [anchor=north,sloped] {\{Follows,Cites\}} (a4.north east);
	\end{scope}
	
	
	\node<2-5> at (3,0) {\textbf{1. Vertex Join}};
	\node<6-> at (3,0) {\textbf{2. Combining Edges}};
	
	\end{tikzpicture}
\end{lucido}


%%%%%%%%%%%%%%%%%%%%%%
%%%%%%%%%%%%%%%%%%%%%%
%%%%%%%%%%%%%%%%%%%%%%
\begin{lucido}[Disjunctive EquiJoin]
		\begin{tikzpicture}
%% Operand Left
\begin{scope}[scale=0.6,yshift=5.5cm,xshift=0cm,every node/.style={transform shape}]
\node[user2] (a1) at (2,.25) {\user{Alice}{$t_1$}};
\node[user2] (a3) at (2,-2.25) {\user{Carl}{$t_1$}};
\node[user2] (a2) at (6.3,.25) {\user{Bob}{$t_2$}};
\node[user2] (a4) at (6.3,-2.25) {\user{Dan}{$t_2$}};
\path[->,orange] (a1.north east) edge node [anchor=south,sloped] {\{Follows\}} (a2.north west);
\path[->,orange] (a2.south east) edge  node [anchor=south,sloped] {\{Follows\}\qquad} (a4.north east);
\path[<-,orange] (a3.north west) edge node [anchor=north,sloped] {\{Follows\}\qquad} (a1.south west);
\path[<-,orange] (a3.east) edge node [anchor=south,sloped] {\{Follows\}\qquad} (a4);
\onslide<2>
\path[->,orange,line width=1mm] (a1.north east) edge node [anchor=south,sloped] {\{Follows\}} (a2.north west);
\onslide<3>
\path[<-,orange,line width=1mm] (a3.north west) edge node [anchor=north,sloped] {\{Follows\}\qquad} (a1.south west);
\onslide<4>
\path[->,orange,line width=1mm] (a2.south east) edge  node [anchor=south,sloped] {\{Follows\}\qquad} (a4.north east);
\onslide<5>
\path[<-,orange,line width=1mm] (a3.east) edge node [anchor=south,sloped] {\{Follows\}\qquad} (a4);
\onslide<1->
\end{scope}

%% Operand Right
\begin{scope}[scale=0.6,yshift=-4cm,xshift=0cm,every node/.style={transform shape}]
\node[proj2] (a1) at (2,0) {\proj{Alice}{Graphs}};
\node[proj2] (a1b) at (5,0) {\proj{Alice}{Join}};
\node[proj2] (a3) at (5,-2) {\proj{Carl}{Projection}};
\node[proj2] (a2) at (8.3,0) {\proj{Bob}{OWL}};
\node[proj2] (a4) at (8.3,-2) {\proj{Dan}{$\mu$-calc}};

\path[->,purple] (a1.north east) edge [bend left=40] node [anchor=south,sloped] {\{Cites\}} (a2.north west);
\path[->,purple] (a2.south west) edge  node [anchor=south,sloped] {\{Cites\}\qquad} (a3.north east);
\path[<-,purple] (a3.north west) edge node [anchor=north,sloped] {\{Cites\}\qquad} (a1b.south west);
\path[->,purple] (a3.south) edge [bend right=30] node [anchor=south,sloped] {\{Cites\}} (a4.south);
\onslide<2>
\path[->,purple,line width=1mm] (a1.north east) edge [bend left=40] node [anchor=south,sloped] {\{Cites\}} (a2.north west);
\onslide<3>
\path[<-,purple,line width=1mm] (a3.north west) edge node [anchor=north,sloped] {\{Cites\}\qquad} (a1b.south west);
\onslide<6>
\path[->,purple,line width=1mm] (a3.south) edge [bend right=30] node [anchor=south,sloped] {\{Cites\}} (a4.south);
\onslide<7>
\path[->,purple,line width=1mm] (a2.south west) edge  node [anchor=south,sloped] {\{Cites\}\qquad} (a3.north east);
\onslide<1->
\end{scope}

%% Result
\begin{scope}[scale=0.6,yshift=2.8cm,xshift=12cm,every node/.style={transform shape}]		

\node[result] (a1) at (1,1) {\result{Alice}{Graphs}};
\node[result] (a2) at (5.5,1) {\result{Alice}{Join}};
\node[result] (a3) at (1,-7.4) {\result{Bob}{OWL}};
\node[result] (a4) at (1,-3.7) {\result{Carl}{Projection}};
\node[result] (a5) at (5.5,-7.4) {\result{Dan}{$\mu$-calc}};
\onslide<2>
\path[->,purple!60!orange,line width=1mm,dotted] (a1.north west) edge [bend right=33] node [anchor=south,sloped] {\{Follows,Cites\}} (a3.north west);
\path[->,orange,line width=1mm] (a2) edge [bend left] node [anchor=south,sloped] {\{Follows\}} (a3);
\onslide<3>
\path[->,orange,line width=1mm] (a1.south west) edge node [anchor=north,sloped] {\{Follows\}} (a4.north west);
\path[->,purple!60!orange,line width=1mm,dotted] (a2) edge  node [anchor=north,sloped] {\{Follows,Cites\}} (a4.north east);
\onslide<4>
\path[->,orange,line width=1mm] (a3.south east) edge node [anchor=south,sloped] {\{Follows\}} (a5.south west);
\onslide<5>
\path[->,orange,line width=1mm] (a5.north east) edge [bend right] node [anchor=south,sloped,pos=0.2] {\{Follows\}} (a4);
\onslide<6>
\path[->,purple,line width=1mm] (a4.south east) edge node [anchor=south,sloped] {\{Cites\}} (a5.north west);	
\onslide<7>
\path[->,purple,line width=1mm] (a3) edge  node [anchor=north,sloped] {\{Cites\}} (a4);

\onslide<3->
\path[->,purple!60!orange,very thick,dotted] (a1.north west) edge [bend right=33] node [anchor=south,sloped] {\{Follows,Cites\}} (a3.north west);
\path[->,orange] (a2) edge [bend left] node [anchor=south,sloped] {\{Follows\}} (a3);
\onslide<4->
\path[->,orange] (a1.south west) edge node [anchor=north,sloped] {\{Follows\}} (a4.north west);
\path[->,purple!60!orange,very thick,dotted] (a2) edge  node [anchor=north,sloped] {\{Follows,Cites\}} (a4.north east);
\onslide<5->
\path[->,orange] (a3.south east) edge node [anchor=south,sloped] {\{Follows\}} (a5.south west);
\onslide<6->
\path[->,orange] (a5.north east) edge [bend right] node [anchor=south,sloped,pos=0.2] {\{Follows\}} (a4);
\onslide<7->
\path[->,purple] (a4.south east) edge node [anchor=south,sloped] {\{Cites\}} (a5.north west);
\onslide<8->
\path[->,purple] (a3) edge  node [anchor=north,sloped] {\{Cites\}} (a4);	
\end{scope}
\end{tikzpicture}
\end{lucido}
