%\section{Physical Model}

\section{THoSP Algorithm}
\begin{lucido}[Physical Model]
Motivations:
\begin{enumerate}
	\setbeamertemplate{enumerate items}[circle]
	\item Reduce the number of graph visiting times by visiting the subpattern first, and then extending the visit to the remaining patterns.
	\item Represent the nested graph as an adjacency list enriched with an external nesting index.
\end{enumerate}
The algorithm uses the same principles that were adopted for implementing graph joins:
\begin{itemize}
	\setbeamertemplate{itemize items}[square]
	\item Use memory mapping (OS buffering).
	\item Serialized graphs represent vertices  associated to both ingoing and outgoing edges.
	\item No additional indexing structures are exploited.
\end{itemize}
\end{lucido}

\begin{lucido}[Example]
	\tikzstyle{selected edge} = [draw,line width=5pt,-,green!50]
\tikzset{
	entity/.code={
		\tikzset{
			rounded corners,             
			name=#1,
			inner sep=2pt,
			every entity/.try,
		}%
		\def\entityname{#1}%
	},
	elabel/.style = {
		above,midway,sloped
	},
	eprop/.style = {
		draw=black, text width=2.2cm, below, midway, sloped
	},
	entity anchor/.style={matrix anchor=#1},
	every entity/.style={
		draw,
	},
	every property/.style={
		inner xsep=0.20cm, inner ysep=0.075cm, anchor=west, text width=1.75in
	}
}
\def\property#1{\node[name=\entityname-#1, every property/.try]{\propertysplit#1;};}
\def\properties{\begingroup\catcode`\_=11\relax\processproperties}
\def\processproperties#1{\endgroup%
	\gdef\propertycode{}%
	\foreach \p in {#1}{%
		\expandafter\expandafter\expandafter\gdef\expandafter\expandafter\expandafter\propertycode%
		\expandafter\expandafter\expandafter{\expandafter\propertycode\expandafter\property\expandafter{\p}\\}%
	}%
	\propertycode%
}
\def\propertysplit#1:#2;{\textbf{#1}:#2}

\def\entitynamenode{%
	\node[every entity name/.try] (\entityname-name) {\textbf{\entityname}};
	\draw (\entityname-name.south west) -- (\entityname-name.south east);
	\\[1ex]
}
\tikzset{
	every entity name/.style={every propertred!50y/.try, align=center}
}

\resizebox{\textwidth}{!}{\begin{tikzpicture}[every node/.style={font=\ttfamily\Large}, node distance=0.5in]
	\matrix [entity=Author,fill=orange!20,label={above right:0}]  (a1) {
	\entitynamenode
	\properties{
		name : Abigail,        
		surname : Conner
	}
};
\matrix [entity=Author,fill=orange!20,label={above right:1}] at (0,-6) (a2) {
	\entitynamenode
	\properties{
		name : Baldwin,        
		surname : Oliver
	}
};
\matrix [entity=Author,fill=orange!20,label={above right:2}] at (0,-3) (a3) {
	\entitynamenode
	\properties{
		name : Cassie,        
		surname : Norman
	}
};

\matrix [entity=Paper,fill=green!10,label={above right:3}] at (9,0) (s1) {
	\entitynamenode
	\properties{
		title : On Joining Graphs
	}
};
\matrix [entity=Paper,fill=green!10,label={above right:4}] at (9,-3) (s3) {
	\entitynamenode
	\properties{
		title : Object Databases
	}
};
\matrix [entity=Paper,fill=green!10,label={above right:5}] at (9,-6) (s2) {
	\entitynamenode
	\properties{
		title : On Nesting Graphs
	}
};

\draw[->,very thick] (a1) -- (s1) node [elabel,label={below:6}] {AuthorOf};
\draw[->,very thick] (a3) -- (s1) node [elabel,label={below:7}] {AuthorOf};
\draw[->,very thick] (a3) -- (s3) node [elabel,label={below:8}] {AuthorOf};
\draw[->,very thick] (a2) -- (s3) node [elabel,label={below:9}] {AuthorOf};
\draw[->,very thick] (a2) -- (s2) node [elabel,label={below:10}] {AuthorOf};
\node at (30,-17) {};

\begin{pgfonlayer}{background}
\node<2-5>[inner sep=10pt,rounded corners,fill=green!50,fit=(s1)]{};

\node<3,5>[inner sep=10pt,rounded corners,fill=green!50,fit=(a1)]{};
\draw<3,5>[->,selected edge] (a1) -- (s1) {};

\draw<4-5>[->,selected edge] (a3) -- (s1) {};
\node<4-5>[inner sep=10pt,rounded corners,fill=green!50,fit=(a3)]{};

\node<6-8>[inner sep=10pt,rounded corners,fill=green!50,fit=(s3)]{};
\node<6,8>[inner sep=10pt,rounded corners,fill=green!50,fit=(a3)]{};
\node<7-8>[inner sep=10pt,rounded corners,fill=green!50,fit=(a2)]{};
\draw<6,8>[->,selected edge] (a3) -- (s3) {};
\draw<7-8>[->,selected edge] (a2) -- (s3) {};

\node<9>[inner sep=10pt,rounded corners,fill=green!50,fit=(s2)]{};
\node<9>[inner sep=10pt,rounded corners,fill=green!50,fit=(a2)]{};
\draw<9>[->,selected edge] (a2) -- (s2) {};
\end{pgfonlayer}
	
\matrix<2-> [entity=Paper,fill=green!10,label={above right:3}] at (10,-16) (s1p) {
	\entitynamenode
	\properties{
		title : On Joining Graphs
	}
};
\matrix<6-> [entity=Paper,fill=green!10,label={above right:4}] at (19,-16) (s3p) {
	\entitynamenode
	\properties{
		title : Object Databases
	}
};

\node<8->[inner sep=25pt,rounded corners,draw=blue!80,fit=(s3p),label=above right :{$\epsilon({\color{blue}\texttt{0}\to\texttt{1}}),\epsilon({\color{blue}\texttt{1}\to\texttt{0}})$}] (12cp){
};

\matrix<9-> [entity=Paper,fill=green!10,label={above right:5}] at (26,-16) (s2p) {
	\entitynamenode
	\properties{
		title : On Nesting Graphs
	}
};
\matrix<3-> [entity=Author,fill=orange!20,label={above right:0}] at (11,-10) (a1p) {
	\entitynamenode
	\properties{
		name : Abigail,        
		surname : Conner
	}
};

\node<3->[inner sep=10pt,rounded corners,draw=orange!80,fit=(s1p),
label={above left:$\epsilon(\texttt{\color{orange}0})$}] (s1cp){
};
\draw<3->[color=orange!80] (a1p.south) -- (s1cp.north west);
\draw<3->[color=orange!80] (a1p.south) -- (s1cp.north east);
\path<3->[top color=orange!50, bottom color=orange!10,opacity=0.2] (a1p.south) -- (s1cp.north west) -- (s1cp.north east) -- (a1p.south);

\node<5->[inner sep=25pt,rounded corners,draw=blue!80,fit=(s1p),
label={above:$\epsilon({\color{blue}\texttt{0}\to\texttt{2}}),\epsilon({\color{blue}\texttt{2}\to\texttt{0}})\qquad \qquad\qquad\qquad\qquad $ }
%,label=below :{$\epsilon(\texttt{\color{blue}20})$}
] (02cp){
};


\node<6->[rounded corners,draw=orange!80,fit=(s1p) (s3p),
label={above right:$\epsilon(\texttt{\color{orange}2})$}] (c22p) {
};



\matrix<7-> [entity=Author,fill=orange!20,label={above right:1}] at (26,-10) (a2p) {
	\entitynamenode
	\properties{
		name : Baldwin,        
		surname : Oliver
	}
};
\node<7-8>[inner sep=10pt,rounded corners,draw=orange!80,fit=(s3p),
label={above right:$\epsilon(\texttt{\color{orange}1})$}] (s3pft){
};
\draw<7-8>[color=orange!80] (a2p.south east) -- (s3p.north west);
\draw<7-8>[color=orange!80] (a2p.south east) -- (s3p.north east);
\path<7-8>[top color=orange!50, bottom color=orange!10,opacity=0.2] (a2p.south east) -- (s3p.north west) -- (s3p.north east) -- (a2p.south east);
\matrix<4-> [entity=Author,fill=orange!20,label={above right:2}] at (20,-10) (a3p) {
	\entitynamenode
	\properties{
		name : Cassie,        
		surname : Norman
	}
};
\draw<4-5>[color=orange!80] (a3p.south) -- (s1cp.north west);
\draw<4-5>[color=orange!80] (a3p.south) -- (s1cp.north east);
\path<4-5>[top color=orange!50, bottom color=orange!10,opacity=0.2] (a3p.south) -- (s1cp.north west) -- (s1cp.north east) -- (a3p.south);
\draw<6->[color=orange!80] (a3p.south west) -- (c22p.north);
\draw<6->[color=orange!80] (a3p.south west) -- (c22p.north east);
\path<6->[top color=orange!50, bottom color=orange!10,opacity=0.2] (a3p.south west) -- (c22p.north) -- (c22p.north east) -- (a3p.south west);


\draw<5->[latex-latex] (a1p) edge [bend left] node  [elabel,sloped] (c02p) {\color{blue}coauthorship} (a3p) ;
\draw<5->[color=blue!80] (c02p.south) -- (02cp.north);
\draw<5->[color=blue!80] (c02p.south) -- (02cp.north east);

\draw<8->[latex-latex] (a3p.south) edge [bend right] node [elabel,sloped] (c21p) {\color{blue}coauthorship} (a2p.south) ;
\draw<8->[color=blue!80] (c21p.south) -- (12cp.north);
\draw<8->[color=blue!80] (c21p.south) -- (12cp.north east);



\node<9->[inner sep=20pt,rounded corners,draw=orange!80,fit=(s2p) (s3p),
label={above right:$\qquad\qquad\qquad \epsilon(\texttt{\color{orange}1})$}] (c24p) {
};
\draw<9->[color=orange!80] (a2p.south east) -- ($ (c24p.north east) - (1,0) $);
\draw<9->[color=orange!80] (a2p.south east) -- (c24p.north east);
\path<9->[top color=orange!50, bottom color=orange!10,opacity=0.2] (a2p.south east) -- ($ (c24p.north east) - (1,0) $) -- (c24p.north east) -- (a2p.south east);
	\end{tikzpicture}  }
\end{lucido}
