\begin{multilucido}[GSQL Operators]
	\begin{sottolucido}
		\centering
		\smartdiagram[constellation diagram]{GSQL Operators, Create, Map, Elect, Disjoint, Fold}
	\end{sottolucido}

	\begin{sottolucido}
		It is possible to define a query language expressing both traversal queries and algebraic operators:
		\begin{description}
			\item[Create] Creates a new object within the GSM database (atoms, objects, collections).
			
			\item[Map] Transforms all the objects within the GSM databas. Allows to express projections, ``select''s and to nest existing elements.
			
			\item[Elect] Elects over which object the n-ary operation has to be performed (Similar to SQL's from).
			
			\item[Disjoint] This operator jointly with map and elect allows to express all the n-ary relational operators. 
			
			\item[Fold] Iteration over an arbitrary \alert{finite} set. Allows iteratively apply the expression's evaluation.  
			
		\end{description}
	\end{sottolucido}
\end{multilucido}

\begin{lucido}[GSQL: Future Works]
\begin{itemize}
	\item Find equivalence rules.
	\item Find optimization by grouping the atomic statements.
\end{itemize}
\end{lucido}
