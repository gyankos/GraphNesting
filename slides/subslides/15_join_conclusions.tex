\begin{lucido}[Results]
	\begin{itemize}
		\item The implementation of this graph operator showed the inefficiency of current graph libraries and database to support graph hashings. Moreover, adjacency list provided a better representation for one-hop distances.
		\item  This motivates the construction of a novel model of graph database based on an adjacency list representation, and ``nested data'' (each vertex contains the outgoing vertices' informations).
	\end{itemize}
\end{lucido}

\begin{lucido}[Further Results]
\begin{description}
	\item[Properties:] Conjunctive and disjunctive joins are commutative and associative.
	\item[Less-Equal Join:] The sort-hash join technique allows to answer less-equal predicates.
	\item[Disjunctive Algorithm:] The disjunctive algorithm can be written as an extension of GCEA.
\end{description}
\end{lucido}

\begin{lucido}[Future Work]
\begin{description}
	\item[Parallelization:] Hashing and bucketing  provide a straightforward parallelization after the operand serialization phase.
	\item[Data Integration:] If alignments are represented as $\theta$ predicates, full disjunctive joins allow to merge graph-schema.
\end{description}
\end{lucido}

\begin{multilucido}[Graph Full Join for schema integration]
	\begin{sottolucido}
		\begin{columns}[c, onlytextwidth]
			\begin{column}{.6\textwidth}
				\begin{center}
				\includegraphics[height=.9\textheight]{../fig/03joins/ontologyLeftRight.pdf}
				\end{center}
			\end{column}
		\begin{column}{.4\textwidth}
			Left and right graphs provide a graphical representation of two nested data operands. Edges within the operands represent the containment-container relations. Edges among the operand represent the alignments, that can be expressed as a $\theta$ predicate. 
		\end{column}
	\end{columns}
	\end{sottolucido}

	\begin{sottolucido}
		\begin{columns}[c, onlytextwidth]
			\begin{column}{.6\textwidth}
				\begin{center}
				\includegraphics[height=.9\textheight]{../fig/03joins/ontologyFullJoin.pdf}
			\end{center}
			\end{column}
			\begin{column}{.4\textwidth}
				The result of performing the graph full $\theta$-join over the disjunctive semantics.
			\end{column}
		\end{columns}
	\end{sottolucido}
\end{multilucido}