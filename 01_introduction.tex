% !TeX root = 00_nesting_paper-Wall-Pedantic.tex

\section{Introduction}
Graph data management systems play an increasing role in present data analytics, because they allow flexible analyses of relationships among data objects. In fact, graphs have been already used as a fundamental data structure for representing data within different contexts, such as corporate data \cite{success,Park2016355}, social networks \cite{xie,BrodkaK14}  and linked data \cite{Vasilyeva13}. 
Despite the increasing production of graph data, a general operator aggregating with a roll-up fashion one single graph is missing. %partitions which the aforementioned vertices represent. 
 Given one single input graph $g$, the desired graph nesting operator should both preserve the vertices and edges that are not involved in the roll-up operation, 
Such operator must create a new graph containing \textbf{nested vertices} and \textbf{nested edges}, containing subgraphs of the original input graph $g$; we would also like that the nested created components may be freely unnested, such that the original graph may be obtained back again. The edges connecting the nested vertices do not necessarily represent a partition of the input graph, but the presence of an edge between two nested vertices may represent the existence (within the input graph) of an edge (or more generally a path) between the subgraphs of $g$ contained in the aforementioned nested vertices. On the other hand,  current literature disembodies these two steps into two distinct approaches, which are \textit{vertex summarization} and \textit{path summarization}, thus not permitting the definition of one single algorithm evaluating both patterns at once. Before outlining our proposed algorithmic solution, we must necessarily discuss these two alternatives.


The \textit{vertex summarization} strategies  group the vertices as in relational \texttt{Group By} strategies, and then aggregate the edges accordingly \cite{JunghannsPR17}. These class of operations restrict the ability of aggregating by paths, that are limited to the set of edges connecting the vertices belonging to two (different) class of vertices. It follows that such class of aggregations   do not allow to freely aggregate the edges into final nested edges. Since most of such vertex summarization techniques are graph-partition based, most of them fail to aggregate graphs that share some vertices and edges in common \cite{yin,Tian20085,jakawat}. Among all those graph summarization techniques, only HEIDS \cite{ChengJQ16} and Graph Cubes \cite{Zhao11} perform graph summarizations of one single graph over a graph collection of non pairwise disjoint subgraphs. On the other hand, these last two techniques require that the set union of such graph collections must be equivalent to the original graph, while data integration and social networks scenarios even consider the possibility of outliers within the data, that are hereby represented in no subgraph belonging to the aforementioned graph collections. %To overcome to this graph operation limitation, this thesis proposes the \textbf{graph nesting} operator, thus providing a general graph summarization technique.
%A straightforward implementation  proves to be inefficient, because the visit of such graph collections of size $k$ (to be nested within the graph operand) implies to perform, in the worst case scenario, $|g|^k$ visits of the graph operand $g$: this results into an exponential algorithm, because the size of $k$ may vary, while $|g|$ is fixed. This implies that the graph must be always visited more than once, even if this may not be required. Even though this general operator proves to be inefficient in practice, it allows to detect a broader class of problems and of optimizable algorithms.
%In order to reduce the graph visiting cost from $|g|^k$ to $O(|g|)$, we could use a graph traversal approach: instead of pre-computing $k$ subgraphs of $g$ that are going to be later on used to nest $g$, we can directly perform the graph nesting while visiting the graph, thus allowing  not to perform additional costs for comparing the resulting graphs in a later step. The following example shows how such queries can be efficiently formulated and implemented.
On the other hand, \textit{path summarizations} techniques  allow the aggregation of multiple paths  among source and target vertices that share the same properties. At the time of the writing, such approaches can be performed only over (graph) query languages.


%The problem with both path and vertex summarizations is that no general class of either source and target vertices can be used as an outcome of a previous community detection \cite{xie,berlingerio11} or data cleaning and alignment phase \cite{ALIEH17} without rewriting the previously extracted data into an explicit query, thus requiring an additional pre-processing step and thus making such approach not as flexible as required by data integration scenarios. %%initial query each time after different vertex data is extracted, thus not allowing to use such query definition for general data integration scenarios. 
%This problem is also reflected by 
Such query languages permit executing queries that represent both these approaches, even though  the vertex summarization phase must be necessarily separated from the path summarization one. This constraint thwarts the advantages of performing vertex and path summarizations contemporarily: while Cypher queries can perform distinct aggregations only within distinct \texttt{MATCH} clauses, SPARQL 1.1 implementations of the query plan combine these two phases with a \texttt{UNION} operator, thus forcing the algorithm to visit the same graph twice. As a result, the query plan optimizers of such query languages do not allow to avoid to visit one same graph more than once whether possible.  

This paper shows that such query language limitations can be reduced by using a graph nesting operator, which embodies both vertex and path summarization queries, which can be performed contemporarily with one single graph visiting processing. %The following example provides an example of how such query can be formulated and performed. --> Old example
To that end, we study the existence of specific graph patterns prone to algorithmic optimizations over a specific physical model. Hereby, we show that such algorithmic approaches allow to reduce the time complexity of the visiting and nesting problem, which is going to performed over a data structure that requires less indexing time than its own competitors.

\begin{figure}[!t]
	\begin{minipage}[!t]{0.5\textwidth}
		\centering
		\includegraphics[width=.8\textwidth]{images/nesting/patterns/04bibliography}
		\subcaption{Input bibliographical network.}
		\label{fig:inputbibex2}
	\end{minipage}
	\medskip
	
	\begin{minipage}[!t]{0.5\textwidth}
		\centering
		\includegraphics[width=\textwidth]{images/nesting/patterns/042nested}
		\subcaption{Nested result: given two \texttt{Author}s $\color{orange}a$ and $\color{orange}a'$, there exist two  \texttt{coauthorship} edges, $\color{blue}a\to a'$ and $\color{blue}a'\to a$ if and only if they share some authored paper contained respectively in $\epsilon({\color{blue}a\to a'})$ and $\epsilon({\color{blue}a'\to a})$. Moreover, each author $\color{orange}a$ is associated to the set of his authored papers $\epsilon({\color{orange}a})$. }
		\label{fig:outputnested}
	\end{minipage}
	\caption{Nesting a bibliographic network: the provenance information is nested within the original node. }
	\label{fig:bibex2}
\end{figure}


\begin{example}
	\label{ex:nestingbib}
	Given a bibliographic network containing (at least) \textsc{Author}s and \textsc{Paper}s as vertices, and where \textsc{authorOf} edges connect each author to the papers he has authored (Figure \ref{fig:inputbibex2}), we want to ``roll up'' the network into a coauthorship network, where each \textsc{Author} is connected by a \textsc{coAuthor} edge with another  \textsc{Author}(2) with which he has published some papers (Figure \ref{fig:outputnested}). In particular, for each resulting \textsc{Author}(2) vertex, nest inside it  its papers as vertices, and nest inside each \textsc{coAuthor} edge all the papers coauthored by  the source and destination \textsc{Author}s. We also want to exclude \textsc{coAuthor} hooks over the same vertex.
	
	
\begin{figure}[!t]
	\centering
	\begin{minipage}[!t]{0.5\textwidth}
		\centering
		\includegraphics[width=.6\textwidth]{images/nesting/patterns/00_vertex_pattern.pdf}
		\subcaption{Vertex summarization pattern ($g_V$). Author is the vertex grouping reference $\gamma_V$.}
		\label{fig:vertexPat}
	\end{minipage} \begin{minipage}[!t]{0.4\textwidth}
		\centering
		\includegraphics[width=1\textwidth]{00_path_pattern.pdf}
		\subcaption{Path summarization pattern ($g_E$). Author$_{src}$ and Author$_{dst}$ are respectively edge grouping references $\gamma_E^{src}$ and $\gamma_E^{dst}$.}
		\label{fig:pathPat}
	\end{minipage}
	\caption{Vertex and Path summarization patterns for the query expressed in Example \ref{ex:nestingbib}. Vertex and edge grouping references are marked by a light blue circled node. As we can see, the vertex grouping reference depicts the same property expressed by edge grouping references.}
	\label{fig:patterns}
\end{figure}
	Figure \ref{fig:patterns} represents the desired vertex ($g_V$) and path ($g_E$) summarization patterns: the former will create a nested \textsc{Author}(2) vertex and the latter will create a \textsc{coAuthor} nested edge. Given that $g_V$ appears twice in $g_E$, we may also pre-istantiate the pattern $g_V$ by visting $g_E$ once. The two patterns have different key roles: while the vertex summarization retrieves all the papers that one author has published and nest them within one single matched author, the path summarizations return all the papers authored by two different authors and creates an edge between the two previously nested vertices. %This construction implies that a join between the two paths must be carried out. 
	
	
	This problem can be solved by visiting the graph only once; %by visiting the graph starting from the vertices:
	if the current vertex is a \textsc{Paper}, traverse backwards all the \textsc{authorOf} edges, thus reaching all of its \textsc{Author}s, that are going to be \textsc{coAuthor} for at least the current paper. Instead of associating the nesting content at the end of the graph visiting process, I can incrementally define the subgraph to be nested by using a separated nesting index: by visiting the two distinct \textsc{Author} vertices adjacent to the current \textsc{Paper}, the latter one shall  be contained in both final \textsc{Author}(2) vertices, thus allowing the definition of a  \textsc{coAuthor} edge. %%The remaining types of vertices and edges %All the other vertices and edges 
	%%may be discarded. %as a starting point for the graph visiting process
	By doing this, only the edges are visited twice, but the vertices are visited only once. Hereby, with these patterns we reach the optimal solution by visiting the graph only once.
	
	%This pattern comparison remarks that, in order to reduce the graph time visit, we must start from visiting the \texttt{Paper}, which is shared among the two distinct patterns, and then keep going with the graph visit by exploring the source and target vertices. 
\end{example}

%There might be other possible patterns that can be optimized, but we're going to focus just on vertex and path summarization patterns where edge grouping references are connected to each other at a 2 edge step distance (Section \ref{sec:THOSPA}). We're also going to show how such optimizations can be detected beforehand by looking at the pattern representation.

The fulfilment of the former scenario is achieved by this paper via the following three contributions:

\begin{itemize}
	\item We implement the aforementioned solution into a nested graph data model, where the logical model differs from the physical one.
	\item \textbf{Graph Nesting Operator} (Section \ref{sec:nestingdef}); we provide a general definition of the nesting operator which combines the path  with the vertex summarization approaches to nesting graphs.
	\item \textsc{{Two HOp Separated Patterns}} \textbf{(THoSP) algorithm for a specific graph nesting task} (Section \ref{sec:THOSPA}): we
	compare it to its implementation
	over both graph  (SPARQL, Cypher), relational (SQL) and document oriented (AQL) query languages. The results of such experiments shows that the sum of both indexing and query evaluation time of our proposed solution outperforms by at least one order of magnitude the aforementioned solutions evaluated on such databases with their respective query languages (Section \ref{sec:nestexpeval}). Consequently, our data model also proved to be curcial in providing an enhanced implementation of the specific graph nesting task.
	%\item A general strategy on how to extend the THoSP algorithm for patterns having vertex and edge grouping references is provided (Section \ref{sec:optimizableClass}).
	%%\item By extending the concept of binary predicates into edges, Edge Joins are introduced as a preliminary step towards the definition of Graph Nesting (Chapter \ref{cha:nesting}).
\end{itemize}

The source code for THoSP is provided at \texttt{\color{red}[Link removed for double-blind review]}.
